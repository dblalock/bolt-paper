
% Recall that our task is to construct functions $g(\cdot)$, $h(\cdot)$, and $f(\cdot)$ such that
% \begin{align}
%     \norm{f(g(\A), h(\B)) - \A\B}_F < \eps(\tau) \norm{\A\B}_F
% \end{align}
% for the smallest $\eps(\tau)$ possible.

% In this section, we discuss the vector quantization (VQ) approach to approximate matrix multiplication. Because these methods approximate matrix products by approximating individual dot products between rows of $\A$ and columns of $\B$, we do so by examining how they approximate a single dot product $\a^\top \b$ between one row of $\A$ and one column of $\B$. For simplicity, we will also begin by focusing on Product Quantization (PQ) \cite{pq}, the classic algorithm on which most others are based.

To lay the groundwork for our own method, we begin by reviewing Product Quantization (PQ) \cite{pq}. PQ is a classic vector quantization algorithm for approximating inner products and Euclidean distances and serves as the basis for nearly all vector quantization methods similar to our own. % This not only lays necessary groundwork for explaining our own method, but also affords the opportunity to discuss

% to both lay the groundwork  the classic vector quantization algorithm for approximating inner products on which most others are based.

The basic intuition behind PQ is that $\a^\top \b \approx \hat{\a}^\top \b$, where $\norm{\hat{\a} - \a}$ is small but $\hat{\a}$ has special structure allowing the product to be computed quickly. This structure consists of $\hat{\a}$ being formed by concatenating learned prototypes in disjoint subspaces; one obtains a speedup by precomputing the dot products between $\b$ and the prototypes once, and then reusing these values across many $\a$ vectors. The $\a$ vectors here are the (transposed) rows of $\A$ and the $\b$ vectors are the columns of $\B$.

 % \cite{pq}, a classic approach to this problem that serves as a conceptual foundation for our own method. There are three basic steps behind PQ:
In somewhat more detail, PQ consists of the following:
\vspace{-3mm}
% \begin{enumerate}\itemsep.5mm
\begin{enumerate}\itemsep2mm
    % \item asdf
    % \item $g(\A)$
    % \item Replacing each row of $\A$ with a prototype chosen from a predefined set.%
    \item \textbf{Prototype Learning} - In an initial, offline training phase,  cluster the rows of $\A$ (or a training set $\tilde{\A}$) using K-means to create prototypes. A separate K-means is run in each of $C$ disjoint subspaces to produce $C$ sets of $K$ protoypes. %The prototypes consist of the cartesian product of prototypes within disjoint subspaces.
    % \item $g(\A)$ - Replace each row of $\A$ with the most similar prototype.
    % \item $h(\B)$ - Precompute the dot products between each column of $\B$ and each prototype.
    \item \textbf{Encoding Function}, $g(\a)$ - Determine the most similar prototype to $\a$ in each subspace. Store these assignments as integer indices using $C \log_2(K)$ bits.
    \item \textbf{Table Construction}, $h(\B)$ - Precompute the dot products between $\b$ and each prototype in each subspace. Store these partial dot products in $C$ lookup tables of size $K$.
    \item \textbf{Aggregation}, $f(\cdot,\cdot)$ - Use the indices and tables to \textit{lookup} the estimated partial $\a^\top \b$ in each subspace, then sum the results across all $C$ subspaces. %$A[i]^\top B[:, j]$\footnote{Because we will frequently need to refer to slices of rank-3 tensors and above, we employ Python-style slicing notation rather than traditional superscripts and subscripts.} rather than compute it with a series of multiply-adds.
\end{enumerate}
\vspace{-3mm}
PQ is depicted for a single pair of vectors $\a$ and $\b$ in Figure~\ref{fig:pq}. We elaborate upon each of these steps below.
% \vspace{-2mm}

% ------------------------------------------------
% \subsection{Prototype Learning}
\vspace{-2mm}
\paragraph{Prototype Learning:}
% ------------------------------------------------

Let $\tilde{A} \in \R^{N \times D}$ be a training set, $K$ be a number of prototypes per subspace, $C$ be a number of subspaces, and $\{\mathcal{J}^{(c)}\}_{c=1}^C$ be the (mutually exclusive and collectively exhaustive) sets of indices associated with each subspace. The training-time task of PQ is to learn $C$ sets of prototypes $\mat{P}^{(c)} \in \R^{K \times |\mathcal{J}^{(c)}|}$ and assignments $\vec{z}^{(c)} \in \R^{N}$ such that:
% \vspace{-1mm}
\begin{align}
    \sum_{i=1}^N \sum_{c=1}^C \sum_{j\in \mathcal{J}^{(c)}} \left( \tilde{\A}_{ij} - \mat{P}^{(c)}_{z^{(c)},j} \right)^2
\vspace*{-.5mm}
\end{align}
is minimized. It does this by running K-means separately in each subspace $\mathcal{J}^{(c)}$ and using the resulting centroids and assignments to populate $\mat{P}^{(c)}$ and $\vec{z}^{(c)}$.

\begin{figure}[t]
\begin{center}
\includegraphics[width=\linewidth]{amm/pq}
\caption{Product Quantization. The $g(\cdot)$ function returns the index of the most similar prototype to the data vector a in each subspace. The $h(\cdot)$ function computes a lookup table of dot products between the query vector b and each protoype in each subspace. The aggregation function $f(\cdot,\cdot)$ sums the table entries corresponding to each index.}
% $\vec{a}$
\label{fig:pq}
\end{center}
\end{figure}


% Perhaps the simplest means of constructing prototypes would be to run a clustering algorithm like K-means on the rows of $\A$. This would be effective, but would require using a large number of prototypes in order to have each row closely match its prototype. What would be preferable is some means of having a huge number of prototypes without having to pay so large a time and space cost.

% PQ achieves this goal by using as prototypes the cartesian product of prototypes within disjoint subspaces. Concretely, PQ runs K-means with $K$ centroids in each of $C$ disjoint (usually contiguous) sets of dimensions. This results in $K^C$ possible combinations of centroid assignments for each vector, with each unique combination corresponding to a unique overall prototype.

% vectors into disjoint ``subvectors'' (each corresponding to a unique set of dimensions) and runs k-means within each subspace.
%  If there are $K$ prototypes per subspace and $C$ subspaces, this results in $K^C$ possible combinations of prototype assignments for each vector. Viewed differently, this creates $K^C$ overall prototypes whose entries

% ------------------------------------------------
% \subsection{Encoding Function - $g(\A)$}
\vspace{-2.5mm}
\paragraph{Encoding Function, $\bm{g(\A)}$:}
% ------------------------------------------------

Given the learned prototypes, PQ replaces each row $\a$ of $\A$ with the concatenation of its $C$ K-means centroid assignments in each of the $C$ subspaces. Formally:
\vspace{-.5mm}
\begin{align} \label{eq:pqLoss}
    g^{(c)}(\a) \triangleq \argmin_k \sum_{j\in \mathcal{J}^{(c)}} \left( \a_j - \mat{P}^{(c)}_{k,j} \right)^2.
\vspace*{-.5mm}
\end{align}
We will refer to the resulting sequence of indices as the \textit{encoding} of $\a$ and the set of $K$ centroids as a \textit{codebook}. For convenience, we will also refer to the vector $\a^{(c)} \triangleq \langle a_j \rangle, j \in \mathcal{J}^{(c)}$ as the \textit{subvector} of $\a$ in subspace $c$. %This encoding function has complexity $\Theta(KD)$ for each $\a$, since there are $K$ centroids per subspace and the sum of the number of indices in all subspaces is $D$.

% ------------------------------------------------
% \subsection{Table Construction - $h(\B)$}
\vspace{-2.5mm}
\paragraph{Table Construction, $\bm{h(\B)}$:}
% ------------------------------------------------

Using these same prototypes, PQ constructs a lookup table $h^{(c)}(\b) \in \R^K$ in each of the $C$ subspaces for each column $\b$ of $\B$, where
% \vspace{-.5mm}
\begin{align}
    h^{(c)}(\b)_k \triangleq \sum_{j\in \mathcal{J}^{(c)}} \b_j \mat{P}^{(c)}_{k,j}.
\end{align}
Existing work has shown that setting $K = 16$ and quantizing the lookup tables to 8 bits can offer enormous speedups compared to larger $K$ and/or floating-point tables \cite{bolt, quickAdc, quickerAdc}. This is because 16 1-byte entries can be stored in a SIMD register, allowing 16 or more table lookups to be performed in parallel in a single instruction. Since the table entries naturally occupy more than 8 bits even for 8-bit data, some means of quantizing these entries is necessary. This can easily be done by subtracting off the minimum entry in each table and linearly rescaling such that the maximum entry in any table is at most 255. Ignoring rounding error, this affine transform is invertible, and is reflected by the constants $\alpha$ and $\beta$ in equation~\ref{eq:objective}. See Appendix~\ref{sec:lutQuantize} for additional details.

% \vspace{-.5mm}
%This has complexity $\Theta(KD)$ for each $\b$, since it is essentially the same as $g(\cdot)$ except without the $\argmin$ operation.

% ------------------------------------------------
% \subsection{Aggregation - $f(\cdot,\cdot)$}
\vspace{-2mm}
\paragraph{Aggregation, $\bm{f(\cdot,\cdot)}$:}
% ------------------------------------------------

Given the encoding of $\a$ and the lookup tables for $\b$, the product can be approximated as
\vspace*{-2mm}
\begin{align}
    \a^\top \b = \sum_{c=1}^C \a^{(c)\top} \b^{(c)} \approx \sum_{c=1}^C h^{(c)}(\b)_k, \text{ }k = g^{(c)}(\a).
\end{align}
% \vspace{-.5mm}

%This has complexity $\Theta(C)$ for each pair of vectors, or $\Theta(NMC)$ for the full matrix product $\A\B$. Since $C \ll D$, this provides a large speedup when $N$ and $M$ are large enough to amortize the costs of $g(\cdot)$ and $h(\cdot)$.

% % ------------------------------------------------
% \subsection{Complexity and Extensions}
% % ------------------------------------------------

% The total time complexity of PQ is equal to $\Theta(NDK + DKC + NMC)$. When $N$ and $M$ are large, the last term dominates and the complexity of PQ is essentially $\Theta(NMC)$.
% Assuming one does not use Strassen's or other algorithms that are never used in practice in modern numerical libraries, full matrix multiplication has complexity $\Theta(NDM)$. Thus, PQ offers a speedup of $\Theta(\frac{D}{C})$.

% To increase this ratio, many authors have attempted to construct more elaborate $g(\cdot)$ and $h(\cdot)$ functions. One line of work seeks to preprocess $\a$ and/or $\b$ with rotation matrices \cite{opq,cartesianKmeans,lopq}. Another line of work seeks to relax the restriction that the codebooks use disjoint subspaces; several authors \cite{aq,cq,otq,sq,grvq,stackedQuantizers} have observed that equation~\ref{eq:pqLoss} is a special case of
% \begin{align} \label{eq:nonOrthogonal}
%     \sum_{i=1}^N min_{\{z^{(1)},\ldots,z^{(C)}\}} \sum_{j=1}^D \left( \tilde{A}_{ij} - \sum_{c=1}^C \mat{P}^{(c)}_{z^{(c)},j} \right) ^2.
% \end{align}
% That is, all $C$ codebooks can contribute to approximating any dimension of $\a$, instead of only one codebook being responsible for it. This is strictly more expressive, but also slows down $g(\cdot)$ greatly and complicates training. There has been effort to reduce these burdens by carefully ordering update and assignment operations \cite{stackedQuantizers}, adding sparsity \cite{sq}, or structuring overlap between codebooks so that optimal assignments remain possible \cite{otq}. TODO sentence about optimizing LUTs directly.
