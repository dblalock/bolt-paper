%-------------------------------------------------------------- Includes

\usepackage{bbm}  % who knows?

\usepackage{amsmath}          % basic math
\usepackage{amssymb} 			    % math symbols
% \usepackage{amsthm}           % theorems
\usepackage{textcomp}

% \usepackage{float}            % make figures work
% \usepackage{cite}             % citations would be nice
\usepackage{url}              % better urls; magically keeping doc from breaking due to urls in refs

% \usepackage{tabu}
\usepackage{array}            % multiline table cells; somehow
\usepackage{tabularx}         % better tables
% \usepackage[table]{xcolor}    % colored table cells  % incompatible with sigconf
\usepackage{colortbl}
\newcolumntype{Y}{>{\centering\arraybackslash}X}	% centered column type for tabularx

% stuff for piecewise functions
\usepackage{mathtools}          %loads amsmath as well
\DeclarePairedDelimiter\Floor\lfloor\rfloor
\DeclarePairedDelimiter\Ceil\lceil\rceil

% \DeclareMathOperator*{\argmin}{arg\,min} % argmin
% \DeclareMathOperator*{\argmax}{arg\,max} % argmax
\DeclareMathOperator*{\argmin}{argmin} % argmin
\DeclareMathOperator*{\argmax}{argmax} % argmax

\usepackage{pbox}   % for trick to force linebreaks in table cells

\usepackage{setspace}

\usepackage{bm}  % bold math

\usepackage[dvipsnames]{xcolor}

% \usepackage{hyperref}

% \color{ForestGreen}

% \usepackage{setspace}

% \usepackage{booktabs} % acm recommended, and for pandas latex tables

%-------------------------------------------------------------- Algorithm setup

% commented out for sysml/icml

% \usepackage{algorithm}
% \usepackage{algorithmicx}
% \usepackage[noend]{algpseudocode} % I think this removes trailing "end {if,for,while}"

% % % \algnewcommand{\LineComment}[1]{\State \(\triangleright\) #1} % left-aligned comments
% % % \algnewcommand{\SideComment}[1]{\(//\) #1}
% \algnewcommand{\COMMENT}[2][.5\linewidth]{\leavevmode\hfill\makebox[#1][l]{//~#2}}
% \algnewcommand{\LineComment}[1]{\State \(//\) #1}	% left-aligned comments
% \algnewcommand\RETURN{\State \textbf{return} }


% sane comments when forced to use plain algorithmic instead of algorithmicx
% \renewcommand{\algorithmiccomment}[1]{\bgroup\hfill//~#1\egroup}
    % {\color{Gray}
\renewcommand{\algorithmiccomment}[1]{
    {\color{OliveGreen}
        \bgroup\hfill//~#1\egroup
    }
}
\newcommand{\LINECOMMENT}[1]{
    {\color{OliveGreen}
        //~#1
    }
}


%-------------------------------------------------------------- Figures setup

% \usepackage[pdftex]{graphicx}
\usepackage{graphicx}
\usepackage[space]{grffile}   % allow spaces in file names
% declare the path(s) where your graphic files are
\graphicspath{{../figs/}}
% and their extensions so you won't have to specify these
\DeclareGraphicsExtensions{.pdf,.jpeg,.jpg,.png}


%\textfloatsep: space between last top float or first bottom float and the text (default = 20.0pt plus 2.0pt minus 4.0pt).
%\intextsep : space left on top and bottom of an in-text float (default = 12.0pt plus 2.0pt minus 2.0pt).
\setlength{\textfloatsep}{4pt}
\setlength{\intextsep}{4pt}

% \usepackage{caption}
% \usepackage[font={small,it}]{caption}
\usepackage[font={bf}]{caption}
\setlength{\abovecaptionskip}{1pt} % less space between captions and figures
% \setlength{\abovecaptionskip}{-2pt} % less space between captions and figures
% \setlength{\abovecaptionskip}{-5pt}	% less space between captions and figures
% \setlength{\belowcaptionskip}{-13pt}  % less space below captions
% \setlength{\belowcaptionskip}{-10pt}  % less space below captions
\setlength{\belowcaptionskip}{-4pt}	% less space below captions

\usepackage{paralist}
\setdefaultleftmargin{10pt}{10pt}{}{}{}{}

% \usepackage{changepage}
% \usepackage{tabulary}

\usepackage{outlines}
\usepackage{enumitem}
\newcommand{\ItemSpacing}{0mm}
\newcommand{\ParSpacing}{0mm}
\setenumerate[1]{itemsep={\ItemSpacing},parsep={\ParSpacing},label=\arabic*.}
% \setenumerate[2]{itemsep={\ItemSpacing},parsep={\ParSpacing},label=\arabic*.}
\setenumerate[2]{itemsep={\ItemSpacing},parsep={\ParSpacing}}

% \usepackage[linewidth=1pt]{mdframed}
% \mdfsetup{frametitlealignment=\center, skipabove=0, innertopmargin=1mm,
% innerleftmargin=2mm, leftmargin=0mm, rightmargin=0mm}

%-------------------------------------------------------------- Miscellaneous setup

% \usepackage{enumitem}
% \setlist{nolistsep}
% \setlist{first=itemsep1em}
% \setlist[1]{labelindent=\parindent}

% \newlength\myindent
% \setlength\myindent{2em}
% \newcommand\bindent{%
%   \begingroup
%   \setlength{\itemindent}{\myindent}
%   \addtolength{\algorithmicindent}{\myindent}
% }
% \newcommand\eindent{\endgroup}

% remove unwanted space between paragraphs;
% it's set by the IEEE conference format, but no papers from this conference have it
% \parskip 0ex plus 0.2ex minus 0.1ex

% make vectors be bold instead of with arrows
% \renewcommand{\vec}[1]{\mathbf{#1}}
\renewcommand{\vec}[1]{\bm{#1}}
% add 'mat' command to make matrices bold
% \newcommand{\mat}[1]{\mathbf{#1}}
\newcommand{\mat}[1]{\bm{#1}}


\newcommand{\cs}[0]{\text{, }}

% \DeclarePairedDelimiter\ceil{\lceil}{\rceil}
% \DeclarePairedDelimiter\floor{\lfloor}{\rfloor}
\providecommand{\ceil}[1]{\left \lceil #1 \right \rceil }
\providecommand{\floor}[1]{\left \lfloor #1 \right \rfloor }

% \newtheorem{Definition}{Definition}[section]
% \newtheorem{Theorem}{Theorem}[section]
% \newtheorem{P}{Definition}[section]

% % ------------------------ convenience commands
% \newcommand\eps\varepsilon
% \DeclareMathOperator{\infimum}{inf}
% \renewcommand\inf\infty
% \DeclareMathOperator{\erfc}{erfc}

% \renewcommand{\c}{\vec{c}}
% \newcommand{\q}{\vec{q}}
% \renewcommand{\r}{\vec{r}}
% \renewcommand{\v}{\vec{v}}
% \newcommand{\vhat}{\hat{\vec{v}}}
% \newcommand{\x}{\vec{x}}
% \newcommand{\xhat}{\hat{\vec{x}}}
% \newcommand{\y}{\vec{y}}
% \newcommand{\yhat}{\hat{\vec{y}}}
% \newcommand{\z}{\vec{z}}
% \newcommand{\zhat}{\hat{\vec{z}}}

% \newcommand{\lam}{\lambda}
% \newcommand{\sig}{\sigma}
% \newcommand{\Sig}{\Sigma}

% \newcommand{\E}{\mathop{{}\mathbb{E}}}

% \newcommand{\A}{\mat{A}}
% \newcommand{\Ahat}{\mat{\hat{A}}}
% \renewcommand{\a}{\vec{a}}
% \newcommand{\ahat}{\vec{\hat{a}}}
% \newcommand{\B}{\mat{B}}
% \newcommand{\Bhat}{\mat{\hat{B}}}
% \renewcommand{\b}{\vec{b}}
% \newcommand{\bhat}{\vec{\hat{b}}}
% \newcommand{\V}{\mat{V}}

% \renewcommand{\H}{\mathcal{H}}
% \newcommand{\R}{\mathbb{R}}
% % \newcommand{\V}{\mathcal{V}}
% \newcommand{\Xs}{\mathcal{X}}
% % \newcommand{\Z}{\mathcal{Z}}
% % \renewcommand{\S}{\mathcal{S}}
% % \newcommand{\Nb}{\mathcal{N}_b}

% \newcommand{\onehalf}{\frac{1}{2}}

% \newcommand{\Xcal}{\mathcal{X}}
% \newcommand{\Ycal}{\mathcal{Y}}

% \newcommand{\pie}[1]{\frac{\pi}{#1}}
% % \newcommand{\pitwo}{\frac{\pi}{2}}
% % \newcommand{\pifour}{\frac{\pi}{4}}

% % \newcommand{\norm}[1]{\left\lVert #1 \right\rVert}
% \DeclarePairedDelimiter\abs{\lvert}{\rvert}%
% \DeclarePairedDelimiter\norm{\lVert}{\rVert}%

% \DeclareMathOperator{\Beta}{Beta}
% \DeclareMathOperator{\Normal}{\mathcal{N}}
% \DeclareMathOperator{\erf}{erf}
% \DeclareMathOperator{\Var}{Var}

% % \newcommand{\iid}{\stackrel{i.i.d.}{\sim}}
% \newcommand{\iid}{\stackrel{iid}{\sim}}

% \newcommand{\Lcal}{\mathcal{L}}
% \newcommand{\Rcal}{\mathcal{R}}
% \newcommand{\Scal}{\mathcal{S}}
% \newcommand{\Dcal}{\mathcal{D}}
% \newcommand{\Fcal}{\mathcal{F}}

% % make *all* text 10pt
% % \renewcommand{\footnotesize}{\normalsize}
% % \renewcommand{\footnotesize}{\small}
% % \renewcommand{\small}{\normalsize}

% ------------------------ convenience commands
\newcommand\eps\varepsilon
\DeclareMathOperator{\infimum}{inf}
\renewcommand\inf\infty
\DeclareMathOperator{\erfc}{erfc}

\renewcommand{\c}{\vec{c}}
\newcommand{\q}{\vec{q}}
\renewcommand{\r}{\vec{r}}
\newcommand{\w}{\vec{w}}
\renewcommand{\v}{\vec{v}}
\newcommand{\vhat}{\hat{\vec{v}}}
\newcommand{\x}{\vec{x}}
\newcommand{\xhat}{\hat{\vec{x}}}
\newcommand{\y}{\vec{y}}
\newcommand{\yhat}{\hat{\vec{y}}}
\newcommand{\z}{\vec{z}}
\newcommand{\zhat}{\hat{\vec{z}}}
\newcommand{\X}{\mat{X}}

\newcommand{\lam}{\lambda}
\newcommand{\sig}{\sigma}
\newcommand{\Sig}{\Sigma}
\newcommand{\Sigm}{\mat{\Sigma}}

% \newcommand{\E}{\mathop{{}\mathbb{E}}}
\newcommand{\E}{\mathbb{E}}

\newcommand{\A}{\mat{A}}
\newcommand{\Ahat}{\mat{\hat{A}}}
\renewcommand{\a}{\vec{a}}
\newcommand{\ahat}{\vec{\hat{a}}}
\newcommand{\B}{\mat{B}}
\newcommand{\Bhat}{\mat{\hat{B}}}
\renewcommand{\b}{\vec{b}}
\newcommand{\g}{\vec{g}}
\newcommand{\bhat}{\vec{\hat{b}}}
\newcommand{\U}{\mat{U}}
\newcommand{\Ut}{\mat{U}^{\top}}
\newcommand{\V}{\mat{V}}
\newcommand{\Vt}{\mat{V}^{\top}}
\newcommand{\W}{\mat{W}}
\newcommand{\Y}{\mat{Y}}
\newcommand{\Z}{\mat{Z}}
\newcommand{\I}{\mat{I}}
\renewcommand{\P}{\mat{P}}
\newcommand{\G}{\mat{G}}

\newcommand{\vtheta}{\vec{\theta}}
\newcommand{\vdelta}{\vec{\delta}}

\renewcommand{\rshift}{\texttt{>>}\text{ }}
\renewcommand{\lshift}{\texttt{<<}\text{ }}

\renewcommand{\H}{\mathcal{H}}
\newcommand{\R}{\mathbb{R}}
% \newcommand{\V}{\mathcal{V}}
\newcommand{\Xs}{\mathcal{X}}
% \newcommand{\Z}{\mathcal{Z}}
% \renewcommand{\S}{\mathcal{S}}
% \newcommand{\Nb}{\mathcal{N}_b}

\newcommand{\pie}[1]{\frac{\pi}{#1}}
\newcommand{\piOver}[1]{\frac{\pi}{#1}}
\newcommand{\onehalf}{\frac{1}{2}}

\renewcommand{\sp}{\text{ }}

\newcommand{\Xcal}{\mathcal{X}}
\newcommand{\Ycal}{\mathcal{Y}}

% \newcommand{\pitwo}{\frac{\pi}{2}}
% \newcommand{\pifour}{\frac{\pi}{4}}

% \newcommand{\norm}[1]{\left\lVert #1 \right\rVert}
\DeclarePairedDelimiter\abs{\lvert}{\rvert}%
\DeclarePairedDelimiter\norm{\lVert}{\rVert}%

\DeclareMathOperator{\Beta}{Beta}
\DeclareMathOperator{\Normal}{\mathcal{N}}
\DeclareMathOperator{\erf}{erf}
\DeclareMathOperator{\Var}{Var}
\DeclareMathOperator{\sign}{sign}
% \DeclareMathOperator{\log2}{\text{log}_2}

\DeclareMathOperator{\MSE}{MSE}  % bolt math

\newcommand{\Lcal}{\mathcal{L}}
\newcommand{\Rcal}{\mathcal{R}}
\newcommand{\Scal}{\mathcal{S}}
\newcommand{\Dcal}{\mathcal{D}}
\newcommand{\Fcal}{\mathcal{F}}

% martingale math
\newcommand{\F}{\mathcal{F}}
\newcommand{\p}{P}

% \newcommand{\iid}{\stackrel{i.i.d.}{\sim}}
\newcommand{\iid}{\stackrel{iid}{\sim}}
% \renewcommand{\log}{\textit{log}}
% \renewcommand{\log}{\textit{log}}
% \RenewMathOperator{\log}{\text{log}}

% make *all* text 10pt
% \renewcommand{\footnotesize}{\normalsize}
% \renewcommand{\footnotesize}{\small}
% \renewcommand{\small}{\normalsize}


% ------------------------------------------------ stuff that might break things


% \usepackage{mathtools}
\usepackage{amsthm}           % theorems
\newtheorem{theorem}{Theorem}[section]
\newtheorem{lemma}{Lemma}[section]
\newtheorem{definition}{Definition}[section]
\newtheorem{corollary}{Corrolary}[section]
\newtheorem*{theorem*}{Theorem}  % no numbering
