
% \documentclass[conference]{IEEEtran}
% \documentclass{sig-alternate} % pre 2017
\documentclass[sigconf]{acmart}  % starting in 2017
%-------------------------------------------------------------- Includes

\usepackage{bbm}  % who knows?

\usepackage{amsmath}          % basic math
\usepackage{amssymb} 			    % math symbols
% \usepackage{amsthm}           % theorems
\usepackage{textcomp}

% \usepackage{float}            % make figures work
% \usepackage{cite}             % citations would be nice
\usepackage{url}              % better urls; magically keeping doc from breaking due to urls in refs

% \usepackage{tabu}
\usepackage{array}            % multiline table cells; somehow
\usepackage{tabularx}         % better tables
% \usepackage[table]{xcolor}    % colored table cells  % incompatible with sigconf
\usepackage{colortbl}
\newcolumntype{Y}{>{\centering\arraybackslash}X}	% centered column type for tabularx

% stuff for piecewise functions
\usepackage{mathtools}          %loads amsmath as well
\DeclarePairedDelimiter\Floor\lfloor\rfloor
\DeclarePairedDelimiter\Ceil\lceil\rceil

% \DeclareMathOperator*{\argmin}{arg\,min} % argmin
% \DeclareMathOperator*{\argmax}{arg\,max} % argmax
\DeclareMathOperator*{\argmin}{argmin} % argmin
\DeclareMathOperator*{\argmax}{argmax} % argmax

\usepackage{pbox}   % for trick to force linebreaks in table cells

\usepackage{setspace}

\usepackage{bm}  % bold math

\usepackage[dvipsnames]{xcolor}

% \usepackage{hyperref}

% \color{ForestGreen}

% \usepackage{setspace}

% \usepackage{booktabs} % acm recommended, and for pandas latex tables

%-------------------------------------------------------------- Algorithm setup

% commented out for sysml/icml

% \usepackage{algorithm}
% \usepackage{algorithmicx}
% \usepackage[noend]{algpseudocode} % I think this removes trailing "end {if,for,while}"

% % % \algnewcommand{\LineComment}[1]{\State \(\triangleright\) #1} % left-aligned comments
% % % \algnewcommand{\SideComment}[1]{\(//\) #1}
% \algnewcommand{\COMMENT}[2][.5\linewidth]{\leavevmode\hfill\makebox[#1][l]{//~#2}}
% \algnewcommand{\LineComment}[1]{\State \(//\) #1}	% left-aligned comments
% \algnewcommand\RETURN{\State \textbf{return} }


% sane comments when forced to use plain algorithmic instead of algorithmicx
% \renewcommand{\algorithmiccomment}[1]{\bgroup\hfill//~#1\egroup}
    % {\color{Gray}
\renewcommand{\algorithmiccomment}[1]{
    {\color{OliveGreen}
        \bgroup\hfill//~#1\egroup
    }
}
\newcommand{\LINECOMMENT}[1]{
    {\color{OliveGreen}
        //~#1
    }
}


%-------------------------------------------------------------- Figures setup

% \usepackage[pdftex]{graphicx}
\usepackage{graphicx}
\usepackage[space]{grffile}   % allow spaces in file names
% declare the path(s) where your graphic files are
\graphicspath{{../figs/}}
% and their extensions so you won't have to specify these
\DeclareGraphicsExtensions{.pdf,.jpeg,.jpg,.png}


%\textfloatsep: space between last top float or first bottom float and the text (default = 20.0pt plus 2.0pt minus 4.0pt).
%\intextsep : space left on top and bottom of an in-text float (default = 12.0pt plus 2.0pt minus 2.0pt).
\setlength{\textfloatsep}{4pt}
\setlength{\intextsep}{4pt}

% \usepackage{caption}
% \usepackage[font={small,it}]{caption}
\usepackage[font={bf}]{caption}
\setlength{\abovecaptionskip}{1pt} % less space between captions and figures
% \setlength{\abovecaptionskip}{-2pt} % less space between captions and figures
% \setlength{\abovecaptionskip}{-5pt}	% less space between captions and figures
% \setlength{\belowcaptionskip}{-13pt}  % less space below captions
% \setlength{\belowcaptionskip}{-10pt}  % less space below captions
\setlength{\belowcaptionskip}{-4pt}	% less space below captions

\usepackage{paralist}
\setdefaultleftmargin{10pt}{10pt}{}{}{}{}

% \usepackage{changepage}
% \usepackage{tabulary}

\usepackage{outlines}
\usepackage{enumitem}
\newcommand{\ItemSpacing}{0mm}
\newcommand{\ParSpacing}{0mm}
\setenumerate[1]{itemsep={\ItemSpacing},parsep={\ParSpacing},label=\arabic*.}
% \setenumerate[2]{itemsep={\ItemSpacing},parsep={\ParSpacing},label=\arabic*.}
\setenumerate[2]{itemsep={\ItemSpacing},parsep={\ParSpacing}}

% \usepackage[linewidth=1pt]{mdframed}
% \mdfsetup{frametitlealignment=\center, skipabove=0, innertopmargin=1mm,
% innerleftmargin=2mm, leftmargin=0mm, rightmargin=0mm}

%-------------------------------------------------------------- Miscellaneous setup

% \usepackage{enumitem}
% \setlist{nolistsep}
% \setlist{first=itemsep1em}
% \setlist[1]{labelindent=\parindent}

% \newlength\myindent
% \setlength\myindent{2em}
% \newcommand\bindent{%
%   \begingroup
%   \setlength{\itemindent}{\myindent}
%   \addtolength{\algorithmicindent}{\myindent}
% }
% \newcommand\eindent{\endgroup}

% remove unwanted space between paragraphs;
% it's set by the IEEE conference format, but no papers from this conference have it
% \parskip 0ex plus 0.2ex minus 0.1ex

% make vectors be bold instead of with arrows
% \renewcommand{\vec}[1]{\mathbf{#1}}
\renewcommand{\vec}[1]{\bm{#1}}
% add 'mat' command to make matrices bold
% \newcommand{\mat}[1]{\mathbf{#1}}
\newcommand{\mat}[1]{\bm{#1}}


\newcommand{\cs}[0]{\text{, }}

% \DeclarePairedDelimiter\ceil{\lceil}{\rceil}
% \DeclarePairedDelimiter\floor{\lfloor}{\rfloor}
\providecommand{\ceil}[1]{\left \lceil #1 \right \rceil }
\providecommand{\floor}[1]{\left \lfloor #1 \right \rfloor }

% \newtheorem{Definition}{Definition}[section]
% \newtheorem{Theorem}{Theorem}[section]
% \newtheorem{P}{Definition}[section]

% ------------------------ convenience commands
\newcommand\eps\varepsilon
\DeclareMathOperator{\infimum}{inf}
\renewcommand\inf\infty
\DeclareMathOperator{\erfc}{erfc}

\renewcommand{\c}{\vec{c}}
\newcommand{\q}{\vec{q}}
\renewcommand{\r}{\vec{r}}
\renewcommand{\v}{\vec{v}}
\newcommand{\vhat}{\hat{\vec{v}}}
\newcommand{\x}{\vec{x}}
\newcommand{\xhat}{\hat{\vec{x}}}
\newcommand{\y}{\vec{y}}
\newcommand{\yhat}{\hat{\vec{y}}}
\newcommand{\z}{\vec{z}}
\newcommand{\zhat}{\hat{\vec{z}}}

\newcommand{\lam}{\lambda}
\newcommand{\sig}{\sigma}
\newcommand{\Sig}{\Sigma}

\newcommand{\E}{\mathop{{}\mathbb{E}}}

\newcommand{\A}{\mat{A}}
\newcommand{\Ahat}{\mat{\hat{A}}}
\renewcommand{\a}{\vec{a}}
\newcommand{\ahat}{\vec{\hat{a}}}
\newcommand{\B}{\mat{B}}
\newcommand{\Bhat}{\mat{\hat{B}}}
\renewcommand{\b}{\vec{b}}
\newcommand{\bhat}{\vec{\hat{b}}}
\newcommand{\V}{\mat{V}}

\renewcommand{\H}{\mathcal{H}}
\newcommand{\R}{\mathbb{R}}
% \newcommand{\V}{\mathcal{V}}
\newcommand{\Xs}{\mathcal{X}}
% \newcommand{\Z}{\mathcal{Z}}
% \renewcommand{\S}{\mathcal{S}}
% \newcommand{\Nb}{\mathcal{N}_b}

\newcommand{\onehalf}{\frac{1}{2}}

\newcommand{\Xcal}{\mathcal{X}}
\newcommand{\Ycal}{\mathcal{Y}}

\newcommand{\pie}[1]{\frac{\pi}{#1}}
% \newcommand{\pitwo}{\frac{\pi}{2}}
% \newcommand{\pifour}{\frac{\pi}{4}}

% \newcommand{\norm}[1]{\left\lVert #1 \right\rVert}
\DeclarePairedDelimiter\abs{\lvert}{\rvert}%
\DeclarePairedDelimiter\norm{\lVert}{\rVert}%

\DeclareMathOperator{\Beta}{Beta}
\DeclareMathOperator{\Normal}{\mathcal{N}}
\DeclareMathOperator{\erf}{erf}
\DeclareMathOperator{\Var}{Var}

% \newcommand{\iid}{\stackrel{i.i.d.}{\sim}}
\newcommand{\iid}{\stackrel{iid}{\sim}}

% make *all* text 10pt
% \renewcommand{\footnotesize}{\normalsize}
% \renewcommand{\footnotesize}{\small}
% \renewcommand{\small}{\normalsize}

% ------------------------------------------------ stuff that might break things


% \usepackage{mathtools}
\usepackage{amsthm}           % theorems
\newtheorem{theorem}{Theorem}[section]
\newtheorem{lemma}{Lemma}[section]
\newtheorem{definition}{Definition}[section]
\newtheorem{corollary}{Corrolary}[section]


\begin{document}

\setcopyright{rightsretained}

%Conference
\acmConference[KDD 2017]{ACM SIGKDD}{August 2017}{Halifax, Nova Scotia Canada}
\acmYear{2017}
\copyrightyear{2017}
% \acmPrice{15.00}

% ================================================================
\title{Bolt: Accelerated Data Mining with Fast Vector Compression}
% ================================================================

\author{Davis W. Blalock}
% \orcid{1234-5678-9012}
\affiliation{%
  \institution{Computer Science and Artificial \\ Intelligence Laboratory}
  \institution{Massachusetts Institute of Technology}
  % \streetaddress{P.O. Box 1212}
  % \city{Dublin}
  % \state{Ohio}
  % \postcode{43017-6221}
}
\email{dblalock@mit.edu}

\author{John V. Guttag}
\affiliation{%
  \institution{Computer Science and Artificial \\ Intelligence Laboratory}
  \institution{Massachusetts Institute of Technology}
  % \streetaddress{P.O. Box 1212}
  % \city{Dublin}
  % \state{Ohio}
  % \postcode{43017-6221}
}
\email{guttag@mit.edu}

% ------------------------------------------------
\begin{abstract}
% ------------------------------------------------

% v0
% The need to compute dot products and distances between vectors of parameters is ubiquitous in machine learning. These computations are often responsible for most of an algorithm's running time, and storing the parameters is usually responsible for most of its space consumption.

% We describe an approximate method of computing dot products and distances that greatly reduces both time and space consumption with little or no loss in accuracy. Our approach is based on vector quantization, and entails replacing the original computation with lookups in a compact table of sufficient statistics about the vectors in question. Unlike similar techniques, our algorithm does not assume that vectors are static and is designed to be hardware-friendly.

% We show experimentally that our approach can be used to accelerate matrix multiplications, k-means clustering, nearest neighbor search, and maximum inner product search by an order of magnitude or more. Furthermore, our approach is faster than even hamming distance computation, which has explicit hardware support on the tested platforms.

% v3

Vectors of data and model weights are at the heart of data mining. Storing such vectors accounts for most of the space usage of many algorithms, and operating on them accounts for most of the compute time. Recently, vector quantization methods have shown great promise in reducing these costs when operating on fixed collections of vectors. However, the high cost of accurately encoding the vectors using these schemes makes them less practical for changing collections, such as production databases, streams of sensor data, or machine learning model weights during training.

We introduce a vector quantization technique that can encode vectors up to $10\times$ faster than existing schemes while also accelerating operations such as distance and dot product computations by over $5\times$. Moreover, because it can encode over two megabytes of vectors per millisecond (2 GB/s), it makes vector quantization cheap enough to employ as a subroutine to accelerate other algorithms.

Specifically, we show experimentally that our approach can be used to accelerate neural network training, k-means clustering, nearest neighbor search, and maximum inner product search by up to $20\times$ compared to floating point operations and $5\times$ compared to other vector quantization methods. Furthermore, our approximate Euclidean distance and dot product computations are faster not only than those of related algorithms with much slower encodings, but even faster than Hamming distance computations, which have direct hardware support on the tested platforms.


% Our approach is to quantize one of the vectors and replace the original computation with lookups in a compact table of sufficient statistics about the other.

\end{abstract}

% ------------------------------------------------
% CCS taxonomy stuff / keywords
% ------------------------------------------------

\begin{CCSXML}
<ccs2012>
<concept>
<concept_id>10002950.10003648</concept_id>
<concept_desc>Mathematics of computing~Probability and statistics</concept_desc>
<concept_significance>500</concept_significance>
</concept>
<concept>
<concept_id>10002950.10003648.10003671</concept_id>
<concept_desc>Mathematics of computing~Probabilistic algorithms</concept_desc>
<concept_significance>300</concept_significance>
</concept>
<concept>
<concept_id>10002950.10003648.10003688.10003696</concept_id>
<concept_desc>Mathematics of computing~Dimensionality reduction</concept_desc>
<concept_significance>300</concept_significance>
</concept>
<concept>
<concept_id>10002950.10003705</concept_id>
<concept_desc>Mathematics of computing~Mathematical software</concept_desc>
<concept_significance>100</concept_significance>
</concept>
</ccs2012>
\end{CCSXML}

\ccsdesc[500]{Mathematics of computing~Probability and statistics}
\ccsdesc[300]{Mathematics of computing~Probabilistic algorithms}
\ccsdesc[300]{Mathematics of computing~Dimensionality reduction}
\ccsdesc[100]{Mathematics of computing~Mathematical software}

\keywords{ACM proceedings, \LaTeX, text tagging}

\maketitle

% ================================================================
\section{Introduction} \label{sec:intro}
% ================================================================


% Get cost equation in here; ops are CRUD, and existing work focuses on extremely read-heavy workloads

% TODO how to frame this given that we actually have both 10x query speed, (almost) 10x encoding speed, and can do matmuls 5x faster than BLAS?

As datasets grow larger, so too do the costs of mining them. These costs include not only the space to store the data, but also the compute time to operate on it. These time costs can be decomposed into:
\begin{align}
    \texttt{Cost}_{\text{time}} = \texttt{Cost}_{\text{read}} + \texttt{Cost}_{\text{write}}
\end{align}
where $\texttt{Cost}_{\texttt{read}}$ is the time cost of reading data, and $\texttt{Cost}_{\texttt{write}}$ is the time cost of creating, updating, or deleting data.

Vector quantization methods presently enable significant reductions in $\texttt{Cost}_{\texttt{storage}}$ and $\texttt{Cost}_{\texttt{read}}$ on datasets of real-valued vectors. By replacing each vector with a learned approximation, these methods both decrease space usage and enable fast approximate scalar reductions, such as dot products and Euclidean distances. With as little as 8B per vector, these techniques can preserve distances and dot products with extremely high accuracy [][][].



 % but add overhead to Cost_{\texttt{write}}.


% When the dataset is fixed, vector quantization methods enable significant reductions in both of these costs. By replacing each vector with a learned approximation, these methods both reduce the space needed to store the vectors and enable fast approximate scalar reductions, such as dot products and Euclidean distances.

However, computing the approximation for a given vector can be time-consuming, adding greatly to $\texttt{Cost}_{\texttt{write}}$. The state-of-the-art method of [LSQ], for example, requires up to \textit{4ms} to encode a single $128$-dimensional vector. Other techniques are faster, but as we show experimentally, virtually all suffer from encoding times that yield significant overhead when data is being added or changed rapidly.

We describe a vector quantization method, Bolt, that greatly reduces both the time to encode vectors ($\texttt{Cost}_{\texttt{write}}$) and the time to compute scalar reductions over them ($\texttt{Cost}_{\texttt{read}}$). This makes quantization worthwhile even for fast-changing or fast-arriving data. Our key idea is to use much smaller quantization codebooks than similar techniques, which both facilitates finding the optimal encoding and allows scans over codes to be done in a vectorized manner.

Our contributions consist of:
\begin{enumerate}
\item A vector quantization algorithm that encodes vectors significantly faster than existing algorithms for a given level of representational fidelity.
\item A fast means of computing approximate similarities and distances using quantized vectors. Possible similarities and distances include dot products, cosine similarities, and distances in $L_p$ spaces, such as the Euclidean distance.
\item Theoretical and empirical evidence that the sufficient statistics used my dozens of recent vector quantization algorithms can be approximated with little or no loss of accuracy.
\item Theoretical analysis of both our approach and popular related approaches.
\end{enumerate}


% Many machine learning algorithms are built around some combination of matrix multiplications, dot products, and distance computations in vector spaces. Such algorithms include deep neural networks, k-means and hierarchical clustering, and linear regression, among many others.

% Lots of people have tried speeding up deep neural nets, mostly by pruning them post-hoc or adopting elementwise quantization / sparsity


% for ease of exposition, we will refer to computing ``similarities'' as shorthand for computing either dot products or euclidean distances throughout the remainder of this work, because typing ``dot products or distances'' is really verbose


% Contributions:

% A fast means of preprocessing vectors so as to minimize quantization loss. This method is compatible with most techniques in the rapidly-evolving area of vector quantization.

% A hardware-friendly means of computing approximate distances using quantized vectors.

% Theoretical analysis of our technique's effectiveness.

% Empirical demonstration that approximate distances suffice for many real-world applications.


% somewhere sneak in the fact that Bolt is an acronym for Based On Lookup Tables


\subsection{Problem Statement}

Formally, the problem our algorithm addresses is the following.

Let $\vec{q} \in \mathbb{R}^J$ be a \textit{query} vector and let $\mathcal{X} = \{\vec{x}_1,\ldots,\vec{x}_N\}, \vec{x}_i \in \mathbb{R}^J$ be a collection of \textit{database} vectors. Further let $d: \mathbb{R}^J \times \mathbb{R}^J \rightarrow \mathbb{R}$ be a distance or similarity function that can be written as:
\begin{align} \label{eq:distFuncForm}
        d(\vec{q}, \vec{x}) = f(\sum_{j=1}^J \delta(q_j, x_j))
\end{align}

% TODO this technically doesn't include how additive methods do ED;
%   wait, nvm, we're fine cuz they use the d_hat equation, not this one

where $f: \mathbb{R} \rightarrow \mathbb{R}$, $\delta: \mathbb{R}^J \times \mathbb{R}^J \rightarrow \mathbb{R}$. This includes both distances in $L_p$ spaces and dot products as special cases. In the former case, $\delta(q_j, x_j) = |q_j - x_j|^p$ and $f(r) = r^{(1/p)}$; in the latter case, $\delta(q_j, x_j) = q_j x_j$ and $f(r) = r$. For brevity, we will henceforth refer to $d$ as a distance function and its output as a distance, though our remarks apply to all functions of the above form unless noted otherwise.

Our task is to construct three functions $g: \mathbb{R}^J \rightarrow \mathcal{G}$, $h: \mathbb{R}^J \rightarrow \mathcal{H}$, and $\hat{d}: \mathcal{G} \times \mathcal{H} \rightarrow \mathbb{R}$ such that for a given approximation loss $\mathcal{L}$,
\begin{align}
    \mathcal{L} = E_{\vec{q},\vec{x}}[(d(\vec{q}, \vec{x}) - \hat{d}(g(\vec{q}), h(\vec{x})))^2]
    % d(\vec{q}, \vec{x}) \approx \hat{d}(g(\vec{q}), h(\vec{x}))
\end{align}
the computation time $T$,
\begin{align}
    T = T_g + T_h + T_d
\end{align}
is minimized, where $T_g$ is the time to encode received queries $\vec{q}$ using $g$\footnote{We cast the creation of query-specific lookup tables as encoding $\vec{q}$ rather than creating a new $\hat{d}$ (the typical interpretation in recent literature).}, $T_h$ the time to encode $\mathcal{X}$ using $h$, and $T_d$ the time to compute the approximate distances between the encoded queries and encoded database vectors. The relative contributions of each of these terms depends on how frequently $\mathcal{X}$ changes, so many of our experiments characterize each separately. % Because related work typically optimizes $T_g$ and $T_d$ at the expense of $T_h$, our differentiating assumption is that $T_h$ cannot be neglected. % This assumption is invalid when $\mathcal{X}$

% \footnote{We describe the creation of lookup tables for each query as encoding of the query, not creation of a query-specific $\hat{d}$ (the typical interpretation in recent literature).}

% . This is not to be confused with the \textit{symmetric} encoding used in some binary embedding methods.

There is a tradeoff between the value of the loss $\mathcal{L}$ and the time $T$, so multiple operating points are possible. In the extreme cases, $\mathcal{L}$ can be fixed at 0 by setting $g$ and $h$ to identity functions and setting $\hat{d} = d$. Similarly, $T$ can be set to $0$ by ignoring the data and estimating $d$ as a constant. The primary contribution of this work is therefore the introduction of $g$, $h$ and $\hat{d}$ functions that are significantly faster than those of existing work for a wide range of operating points.

% the loss:
% \begin{align}
%     \mathcal{L} = E_{\vec{q},\vec{x}}[(d(\vec{q}, \vec{x}) - \hat{d}(g(\vec{q}), h(\vec{x})))^2]
%     % \mathcal{L}[g, h, \hat{d}] = E_{q,x}[(d(q, x) - \hat{d}(g(q), h(x)))^2]
% \end{align}
% is minimized. Moreover, each of these functions should be as fast to compute as possible.

% Intuitively, the function $g$ encodes the query, $h$ encodes the database vectors, and $\hat{d}$ computes an approximate similarity or distance based on the encodings. In general, it is possible to drive the loss arbitrarily low by increasing the time and space usage of these functions. Indeed, the loss can be fixed at 0 by setting $g$ and $h$ to identity functions and setting $\hat{d} = d$. Consequently, our metric of interest is \textbf{computation time for a given loss}. The primary contribution of this work is the introduction of $g$, $h$ and $\hat{d}$ functions that are significantly faster than those of existing work for a given loss $\mathcal{L} > 0$.

% \begin{center}
% % \begin{tabu} to \linewidth [0]{ X[l] | X[c] | X[c] | X[c] | X[c]}
% \begin{tabu} to \textwidth [0]{ X[l] | X[c] | X[c] | X[c] | X[c]}
% % \caption{Performance of Vector Quantization Algorithms}
% \hline
%     & Data Encoding Speed & Query Encoding Speed & Search Speed & Compression \\
% \hline
%     Bolt (proposed) & Very High & Very High & Very High & Medium \\
%     % Binary Embedding & High & High & High & Very Low \\
%     PQ & High & High & High & Low \\
%     OPQ & High & High & High & Medium \\
%     RVQ & High & Medium & High & Medium \\
%     GRVQ, LSQ, CQ, AQ, OTQ & Low, Very Low & Medium & High & High \\
%     Raw Floats & N/A & N/A & Low & None \\
% \hline
% \end{tabu}
% \end{center}




\subsection{Assumptions}

Like other work [][][][], we assume that there is an initial offline phase during which the functions $g$ and $h$ may be learned. This phase contains a training dataset for $\mathcal{X}$ but not necessarily $\vec{q}$. Following this offline phase, there is an online phase wherein we are given database vectors $\vec{x}$ that must be encoded and query vectors $\vec{q}$ for which we must compute the distances to all of the database vectors received so far. Once a query is received, these distances must be computed with as little latency as possible. The vectors of $\mat{X}$ may be given all at once, or one at a time; they may also be modified or deleted, necessitating re-encoding or removal. This is in contrast to most existing work, which assumes that $\vec{x}$ vectors are all added at once before any queries are recieved [][][][].

In practice, one might require the distances between $\vec{q}$ and only some of the database vectors $\mathcal{X}$ (in particular, the $k$ closest vectors). This can be achieved using an indexing structure, such as an Inverted Multi-Index [][] or Locality-Sensitive Hashing hash tables [][], that allow inspection of only a fraction of $\mathcal{X}$. Such indexing is complementary to our work in that our approach could be used to accelerate the computation of distances to the subset of $\mathcal{X}$ that is inspected. Consequently, we assume that the task is to compute the distances to all vectors, noting that, in a production setting, ``all vectors'' for a given query might be a subset of a full database.

%     -we consider only dot products and euclidean distances throughout this work since they are by far the most common, but note that our approach could in principle be generalized to any similarity or distance of the above form.\footnote{Practically speaking, functions with a large range of values for \delta(q_j, x_j), such as distances in Lp space with p > 2, are unlikely to be approximated well.}




% % ================================================================
% \section{Definitions and Problem} \label{sec:problem}
% % ================================================================

% \input{problem.tex}

% ================================================================
\section{Related Work} \label{sec:relatedWork}
% ================================================================


Accelerating vector operations through compression has been the subject of a great deal of research in the computer vision, information retrieval, and machine learning communities, among others, so our review will necessarily be incomplete. We refer the reader to [learningToHash, thatOtherSurvey] for detailed surveys. %  and focus on methods not discussed in Section~\ref{sec:method}

Many existing approaches in the computer vision and information retrieval literature fall into one of two categories [thatOtherSurvey]: binary embedding and vector quantization. Binary embedding techniques seek to map vectors in $\mathbb{R}^J$ to $B$-dimensional Hamming space, typically with $B < J$. The appeal of binary embedding is that a $B$-element vector in hamming space can be stored in $B$ bits, affording excellent compression. Moreover, the \texttt{popcount} instruction present on virtually all desktop, smart phone, and server processors can be used to compute hamming distances between 8 byte vectors in as little as three cycles. This fast distance computation comes at the price of reduced representational accuracy for a given code length [][]. In fact, He et al. [] demonstrated that the popular technique of [ITQ] is a more constrained version of their vector quantization algorithm, and that the objective function of another state-of-the art binary embedding [isohash], can be understood as maximizing only one of two sufficient conditions for optimal encoding of gaussian data.

Vector quantization approaches yield lower errors for a given code length, but entail slower encoding and distance computations. The simplest and most popular vector quantization method is K-means [], which can be seen as encoding a vector as the centroid to which it is closest. A generalization of K-means, Product Quantization (PQ) [], splits the vector into $M$ disjoint subvectors and runs k-means on each. The resulting code is the concatenation of the codes for each subspace. Numerous generalizations of PQ have been published, including Cartesian K-means [] / Optimized Product Quantization (OPQ) [], Additive Quantization (AQ) [], Residual Vector Quantization (RVQ) [], Generalized Residual Vector Quantization (GRVQ), Composite Quantization (CQ) [], Optimized Tree Quantization (OTQ) [], and Local Search Quantization (LSQ) []. The idea behind most of these generalizations is to either rotate the vectors or relax the constraint that the subvectors be disjoint. Collectively, these techniques that rely on using the concatenation of multiple codes to describe a vector are known as Multi-Codebook Quantization (MCQ) methods.

An interesting hybrid between binary embedding and vector quantization is the recent Polysemous Coding of Douze et al. []. This encoding uses Product Quantization codebooks optimized to also function as binary codes, allowing the use of Hamming distances as a fast approximation that can be refined for promising nearest neighbor candidates. Like our own algorithm, it seeks the best of both binary embedding speed and vector quantization accuracy. However, our technique obtains this by speeding up the vector quantization distance computations directly, obviating the need for multiple passes and the possibility of false dismissals from unrepresentative Hamming distances.

In the machine learning community, accelerating vector operations has been done primarily through embedding, structured matrices, and model compression. Embedding yields acceleration by reducing the dimensionality of data while preserving the relevant structure of a dataset overall. There are strong theoretical gaurantees regarding the level of reduction attainable for a given level of distortion in pairwise distances [JL][JLIsTight], and even stronger empirical results [SuperBitLSH, thatWeirdCompressoinOne]. However, because embedding \textit{per se} only entails reducing the number of floating-point numbers stored, without reducing the size of each, we find that it is not at all competitive with vector quantization methods on its own. Moreover, even if it were, it is possible to embed data before applying vector quantization, so the two techniques are complementary.

An alternative to embedding that reduces the cost of storing and multiplying by matrices is the use of structured matrices. This consists of repeatedly applying a linear transform, such as permutation [adaptiveFastfood][hashNets], the Fast Fourier Transform [][], the Discrete Cosine Transform [], or the Fast Hadamard Transform [crossPolytope][][], possibly with learned elementwise weights, instead of performing a matrix multiply. These methods have strong theoretical grounding [structuredSpinners] and sometimes outperform non-structured matrices [adaptiveFastfood]. They are orthogonal to our work in that they bypass the need for a matrix entirely, while our approach can accelerate operations in which a matrix is needed.

Another vector-quantization-like technique common in machine learning is model compression. This typically consists of some combination of 1) restricting the representation of variables, such as nerual network weights, to fewer bits [][]; 2) reusing weights [hashNets][]; 3) pruning weights in a model after training [diversityNets][learningWeightsAndConnections]; and 4) training a small model to approximate the outputs of a larger model [][]. This has been a subject of intense research for neural networks in recent years [][][], so we do not believe that our approach could yield smaller neural networks than the current state of the art. Instead, our focus is on accelerating operations on uncompressed weights and data matching the form of Eq~\ref{eq:distFuncForm}.

In addition to all of the above approaches, there is the possibility of running algorithms on GPUs [], clusters [], ASICs [], or FPGAs []. Since such alternate hardware could accelerate both our own algorithm and the most similar comparisons, we do not address this possibility further.

% % In addition to these methods, there has been a great deal of work on compressing data


% % Our algorithm is similar to above multi-codebook quantization methods,

% In addition to these methods, there has been a great deal

% Vector quantization is an extremely well-studied problem, so our review will necessarily be incomplete. We refer the reader to [learningToHash, thatOtherSurvey] for detailed surveys. The classic K-means algorithm [] is by far the most widely-used vector quantization technique.

% Great deal of work on vector quantization
%     -mostly on generating better encoding algorithms
%         -typically through complex procedures that require tons of encoding time
%             -eg, state of the art (LSQ) encoding takes over a ms *per vector*; we're like a ten thousand times faster than this
%         -sometimes through direct integration with an index, such as the IMI (LOPQ does this)
%     -our scanning algorithm can use any of these encoding schemes that can spit out 4bit codes, and our preprocessing can help with basically any of them
%     -most similar to ours is that VLDB paper that used the same machine instructions; however, it required huge sets of static vectors that could be sorted ahead of time


%     -also lots of work on compressing neural nets
%         -replacing raw mats with structured mats, such as adaptive fastfood transform or the DCT ones
%             -heuristic (?) weight-sharing like hashnets
%             -zelda diversity nets
%         -quantization, sparsity
%             -pruning post-hoc
%             -learning with only quantized reprs
%             -ASIC acceleration
%         -other stuff I don't know that well

%     -sketching / embedding
%         -pretty sure there's some sketch for euclidean distance
%         -sketching and embedding equivalent for norms

%         -sketching for stats such as number of uniques




% ================================================================
% \vspace{-2mm}
\section{Method} \label{sec:method}
% ================================================================


Recall that our task is to construct functions $g(\cdot)$, $h(\cdot)$, and $f(\cdot)$ such that
\begin{align}
    \norm{f(g(\A), h(\B)) - \A\B}_F < \eps(\tau) \norm{\A\B}_F
\end{align}
$\eps(\tau)$ is minimized given a time constraint $\tau$.

% ------------------------------------------------
\subsection{Quantization Function - $g(\A)$}
% ------------------------------------------------

We begin by discussing

Basically just intuition for why it just splits on a bunch of dims. Then define a split formally. Then pseudocode (or probably just formula for applying them)

% ------------------------------------------------
\subsection{Learning Splits}
% ------------------------------------------------

Introduction to fact that we have two variations, one that requires training data and one that doesn't. Former is greedy and mention that we prove it's close to optimal.

Pseudocode for probably both, but maybe just the greedy one.

% ------------------------------------------------
\subsection{Table Construction - $h(\B)$}
% ------------------------------------------------

% ------------------------------------------------
\subsection{Lookup Scan - $f(\cdot,\cdot)$}
% ------------------------------------------------

If we didn't have pseudocode in background section, put it here. Break it down by encoding, LUT creation, and distance computation. Seems like we should probably already have explained these, so probably just text saying we kind of tie it all together and replace whatever lines in earlier pseudocode with hash-based encoding.




% ================================================================
\section{Results} \label{sec:results}
% ================================================================


To assess Bolt's effectiveness, we implemented both it and the algorithms to which we compare it in C++, with wrappers in Python. All of our code and raw results are publicly available at [website]. This website also s contains many additional experiments and thorough documentation of both our code and experimental setups that we hope facilitates reproduction and extension of our work. All experiments use a single thread on a 2013 Macbook Pro with a 2.6GHz Intel Core i7-4960HQ processor. % This processor supports 32 simultaneous table lookups (c.f. Section~\ref{sec:boltVectorize}), which is fewer than the 64 available on more recent processors; this suggests that Bolt would be even faster on a more recent machine.

\subsection{Datasets}

Because there are no conditional branches in either Bolt or the comparison algorithms (when implemented efficiently), all running times are independent of the data and query distributions. Consequently, we report timing results primarily on random data. All timings are the best of 5 runs, averaged over 10 trials (i.e., the code is executed 50 times). We use the best in each trial, rather than average, since this is standard practice in performance benchmarking.

For assessing accuracy, we use several datasets widely used to benchmark Multi-Codebook Quantization (MCQ) algorithms:
\begin{itemize}
\item \textbf{Sift1M} [] --- 1 million 128-dimensional SIFT descriptors of images. Sift1M vectors tend to have high correlations among many dimensions, and so be highly compressible for algorithms that allow global rotations or do not employ space partitions. This dataset has a predefined query/train database/test database split, consisting of 10,000 query vectors, 100,000 training vectors, and 1 million testing vectors.
\item \textbf{Convnet1M} [] --- 1 million 128-dimensional Convnet descriptors of images. These vectors have some amount of correlation, but less so than Sift1M. It has a query/train/test split matching that of Sift1M.
\item \textbf{LabelMe22k} [] --- 22,000 960-dimensional GIST descriptors of images. Like Sift, it has a great deal of correlation between many dimensions. It only has a train/test split, so we follow [][] and use the 2,000-vector test set as the queries and the 20,000 vector training set as both the training and test databse.
\item \textbf{MNIST} [] --- 60,000 28x28-pixel greyscale images, flattened to 768-dimensional vectors. This dataset is sparse and has high correlations between various dimensions. Again following [] and [], we split it the same way as the LabelMe dataset.
\end{itemize}

Experiments on additional datasets are availble on the Bolt website [website].

\subsection{Comparison Algorithms}

Our comparison algorithms include MCQ methods that have high encoding speeds ($\ll 1$ms / vector on a CPU). If encoding speed is not a design consideration or is dominated by a need for maximal compression, we recommend using a method such as GRVQ [] or LSQ [] instead of Bolt\footnote{Although Bolt \textit{might} still be desirable for its high query speed even if encoding speed is not a consideration.}.

Since most research in this area has focused on improving accuracy for a given code length at the \textit{expense} of encoding time, only Product Quantization (PQ) [], Optimized Product Quantization (OPQ) [], and variations thereon such as [polysemous], [googleMips], [pairwiseQ] [LOPQ], [NOIMI] meet our inclusion criterion. Since [LOPQ], [NOIMI] and similar methods are coupled to an indexing structure, which is compatible with but orthogonal to our work, we do not compare to them. Moreover, the optimization of [googleMips] is essentially subsumed by that of [pairwise], which can itself by combined with OPQ by multiplying the rotation matrix by another matrix learned from the query distribution. As a result, the effective encoding and query speeds of each these methods are equal to those of OPQ. In short, by comparing to PQ and OPQ, we can effectively assess the performance of almost all fast-encoding MCQ algorithms, modulo the accuracy improvements afforded by different optimization approaches for OPQ.

We do not to compare to binary embedding methods as they are known to yield much lower accuracy for a given code length than MCQ methods [][][] and, as we show, are also slower in computing distances than Bolt. % This implies that there is no reason to prefer them to Bolt, unless one both required an even more extreme encoding speed than Bolt's ($\gg$ 2GB/s) and devised a binary embedding algorithm that could achieve this.

We have done our best to optimize the implementations of the comparison algorithms to the greatest extent possible, and find that our timings are far superior to those described in previous works. For example, [] reports encoding roughly 190,000 128-dimensional vectors per second with PQ, while our implementation encodes over 24 million per second. Indeed, we believe that the sheer extent to which PQ and related methods can be sped up is an interesting result in itself, though we do not explore it further.

TODO integrate pairQ mat + OPQ rotation mat for each Bolt subspace; then, just compare to OPQ with pairQ premultiply, since it's strictly better. Or maybe PQ, OPQ, PairQ using OPQ so we have 3 things instead of 2.

% \subsection{Comparison }

\subsection{Greater Speed than Binary Embedding}

As mentioned in Section~\ref{sec:relatedWork}, the current fastest method of obtaining approximate distances over compressed vectors is to embed them into Hamming space and use the \texttt{popcount} instruction to quickly compute the Hamming distances between them. Indeed, the speed of this approach is much of the motivation for the class of binary embedding methods (e.g., [][][][]), as well as more recent attempts to integrate hamming distances into product quantization [polysemous].

In shown in Figure~\ref{fig:bolt_vs_popcount}, Bolt can compute approximate distances (of any kind expressible via ~\ref{eq:distFuncForm}) faster than popcount can compute Hamming distances. Moreover, because Bolt can compute Hamming distances exactly (since the possible distances between pairs of 4 bits can be stored exactly in a 16B lookup table), it is both faster and more flexible than using popcount. It also has the benefit that it can use encoding lengths that are not multiples of 8B without loss of efficiency.

\begin{figure}[h]
\begin{center}
\label{fig:bolt_vs_popcount}
% \includegraphics[width=\linewidth, trim={0 2cm 0 0},clip]{moose0}
% \includegraphics[width=\linewidth, trim={0 1cm 0 0},clip]{moose0}
\includegraphics[width=\linewidth]{moose0}
\vspace*{-1mm}
\caption{Bolt can compute various distances, including Hamming distances, faster than binary embedding methods can compute Hamming distances (using the \texttt{popcount} instruction).}
\end{center}
\end{figure}

Above figure should use 8, 16, 32B. And ideally version of Bolt that immediately upcasts to uint16s, which is the most robust to overflows but might not be faster than popcount.

\subsection{Encoding Speed}

Bolt can encode both data and queries faster than any other Multi-Codebook Quantization (MCQ) method of which we are aware. As shown in Figure~\ref{fig:encoding_speeds}a, Bolt can encode data vectors at up to 2GB/s, while the fastest comparison, PQ, reaches at most 200MB/s. For perspective, Bolt's encoding rate is sufficient to encode the entire Sift1M dataset of 1 million vectors in 250ms, and the Sift1B dataset of 1 billion vectors in 250s. This rate is also much higher than that of high-speed (but general-purpose) compression algorithms such as Snappy [], which reports an encoding speed of 250MB/s.

Similarly, Bolt can compute the distance matrix constituting a query's encoding at up to 7 million queries/s, while PQ obtains only 500,000 queries/s Figure~\ref{fig:encoding_speeds}b. Both of these numbers are sufficiently high that encoding the query is unlikely to ever be a bottleneck in computing distances to it.

\begin{figure}[h]
\begin{center}
\label{fig:encoding_speeds}
% \includegraphics[width=\linewidth, trim={0 2cm 0 0},clip]{moose0}
% \includegraphics[width=\linewidth, trim={0 1cm 0 0},clip]{moose0}
\includegraphics[width=\linewidth]{moose1}
\vspace*{-1mm}
\caption{Bolt encodes both data vectors and query vectors significantly faster than existing algorithms.}
\end{center}
\end{figure}

Above needs to show times for 8B and 16B codes.
TODO vary query batch size? Because OPQ (and bolt if we add in rotations) will go faster with bigger batches


\subsection{Query Speed}

Much of the appeal of MCQ methods is that they allow fast computation of approximate distances and similarites directly on compressed data. We compared Bolt's speed in computing Euclidean distances from a batch of queries to each vector in a compressed dataset. We omit profiling of other distances and similarities since they only alter the computation of queries' distance matrices and therefore have nearly identical speeds. In all experiments, the number of compressed data vectors $N$ is fixed at $100,000$.

\begin{figure}[h]
\begin{center}
\label{fig:query_speeds}
\includegraphics[width=\linewidth]{moose2}
\vspace*{-1mm}
\caption{Bolt can compute the distances/similarities between a query $\vec{q}$ and the vectors of a compressed database $H$ up to $5\times$ faster than other MCQ algorithms. It is also faster than batched distance computations on floats using matrix multiplies up to a batch size of TODO.}
\end{center}
\end{figure}

% In addition to reducing the space consumption of a database of vectors $\mathcal{X}$, MCQ methods also allow direct computation of distances and similarities matching the form of~\ref{eq:distFuncForm}.


Use query batch sizes of {1, 32, 64, 128,...,512}; NBytes={8, 16, 32}; fix N at 100k. Compare Bolt, matmul, other MCQ methods.


\subsection{Nearest Neighbor Accuracy}

By far the most common assessment of MCQ and binary embedding algorithms' accuracy is their Recall@R on various benchmark datasets. The Recall@R is the fraction of the queries $\vec{q}$ for which the true nearest neighbor in Euclidean space is among the top $R$ points with smallest approximate distances to $\vec{q}$. As shown in Figure~\ref{fig:nn_acc}, Bolt yields lower accuracy for a given encoding length than other methods designed for maximal compression. And hopefully some comment about how it's at least better than PQ.

\begin{figure}[h]
\begin{center}
\label{fig:nn_acc}
\includegraphics[width=\linewidth]{moose3}
\vspace*{-1mm}
\caption{Bolt is less accurate in retrieving the nearest neighbor for a given encoding length than MCQ algorithms optimized for small encodings, but $<$ hopefully optimistic comment $>$.}
\end{center}
\end{figure}

Datasets: Sift1M, LabelMe, Convnet1M, MNIST; displayed in 2x2 grid of subplots
Within each plot, have 3 lines for each algo: dotted for 32B, solid for 16B, dashed for 8B.
Revert to table if time crunched.


% \begin{table}[h]
%   \caption{Recall@R for Bolt and other MCQ algorithms}
%   \label{tbl:nn_accuracy}
%   % \def\arraystretch{1.1}%  1 is the default, change whatever you need
%   \begin{tabularx}{\linewidth}{X|YYYY}
% \toprule
%             &  Recall@1 & Recall@10 & Recall@100 & Recall@1000
% \hline
%     Bolt 8B    & 0 & 0 & 0 & 0
% \bottomrule
% \end{tabularx}
% \end{table}

\subsection{Maximum Inner Product Search Accuracy}

Because dot products are perhaps even more common vector operations than Euclidian distance computations, we repeated the previous experiment for maximum inner product search (MIPS) instead of 1nn search (Figure~\ref{fig:mips_acc}).

\begin{figure}[h]
\begin{center}
\label{fig:mips_acc}
\includegraphics[width=\linewidth]{moose4}
\vspace*{-1mm}
\caption{Bolt is less accurate in retrieving the neighbor having maximum dot product with the query than MCQ algorithms optimized for small encodings. Again, $<$ hopefully optimistic comment $>$.}
\end{center}
\end{figure}

Might not get to this because we need to modify Bolt to allow int8s in LUT instead of just uint8s.

\subsection{Accuracy in Preserving Distances and Dot Products}

The above experiments characterize how well algorithms preserve distances and similarities to exceptionally similar points, but not whether distances and similarities in general tend to be preserved. To characterize this, we computed the correlations between the true Euclidean distances and approximate Euclidean distances from Bolt and other MCQ algorithms on the LabelMe22k and MNIST datasets (see [website] for results on other datasets).

\begin{figure}[h]
\begin{center}
\label{fig:corr_acc}
\includegraphics[width=\linewidth]{moose5}
\vspace*{-1mm}
\caption{Bolt Euclidean distances are highly correlated with true Euclidean distances, though slightly less so than the distances from other MCQ algorithms.}
\end{center}
\end{figure}

Maybe omit this because accuracy results make us look worse and it's not reported in any other MCQ paper.


\subsection{Case Study: K-Means Clustering}

In addition to accelerating standalone similarity searches, Bolt can be embedded within other algorithms to improve their speed. As an example, we show that using Bolt to compute the distances between centroids and data vectors can greatly accelerate k-means clustering of the full MNIST dataset with little or no loss in accuracy (Figure~\ref{fig:kmeans}). Accuracy is measured by mean squared Euclidean distance between vectors and their associated centroids. This distance is computed exactly, not with Bolt. Reported times are the for Bolt include initial encoding of the data.

\begin{figure}[h]
\begin{center}
\label{fig:kmeans}
\includegraphics[width=\linewidth]{moose6}
\vspace*{-1mm}
\caption{Using Bolt to compute distances within K-means decreases its runtime greatly without increasing its mean squared error (MSE). Standard errors across 10 runs are shown shaded.}
\end{center}
\end{figure}

Use K = 16, K = 256.

\subsection{Case Study: Power Iteration}

Backup if we don't have time for word embedding. Should compare time taken to find top k eigenvects + eigenvals, and just report cosine sims between former and raw values of latter in a table. Use gram schmidt to get eigenvectors after the first (think we have to decompress to do this--so this might not be faster at all).

\subsection{Case Study: Word Embedding}

Hope I have time to do this. Need to ask whether there's an obvious successor to word2vec other than Glove (which isn't amenable to our approach).



% show that it makes matrix-vector muls way faster for various matrix sizes
%     -also show approximation error
%         -which suggests getting lots of meaningful matrices, ideally fc layer weights
%     -prolly have to show both speed and err as a function of code length
% same for matrix-matrix; speed and acc
%     -is there any way we'll be better here? maybe for skinny mats...

% show that we can speed up kmeans a lot, and point out that this is orthogonal to other ppl's speedups based on pruning which ones you consider moving and/or using minibatches
%     -prolly pick a couple datasets and show it as a function of k

% a couple experiments on sift1m scan and mnist scan, prolly; maybe sift10m / deep10m also

% somewhere introduce the "cold scan" problem, wherein we also have to add in time to compress everything, but then we get a batch of queries
%     -if just 1 query, you're always worse off doing the encoding
%     -metric of interest is how many queries you have to do @k database vectors before it's faster than a matmul
%         -and you should also report the times for different query counts and db counts

%     -and prolly also the "warm scan" problem; you only have to encode some subset of the db (cuz it changed)

%     -or maybe the "hot scan" problem cuz data is hot; in contrast to fixed data, which is cold/frozen





% ================================================================
\section{Conclusion} \label{sec:conclusion}
% ================================================================

% ================================================================
% References
% ================================================================

% \IEEEtriggeratref{27}	% trigger column break to make cols even
\bibliographystyle{ACM-Reference-Format}
\bibliography{doc}

\end{document}
