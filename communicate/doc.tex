
% \documentclass[conference]{IEEEtran}
% \documentclass{sig-alternate} % pre 2017
\documentclass[sigconf]{acmart}  % starting in 2017
%-------------------------------------------------------------- Includes

\usepackage{amsmath}          % basic math
\usepackage{amssymb} 			    % math symbols
% \usepackage{amsthm}           % theorems
\usepackage{textcomp}

\usepackage{float}            % make figures work
\usepackage{cite}             % citations would be nice
\usepackage{url}              % better urls; magically keeping doc from breaking due to urls in refs

% \usepackage{tabu}
\usepackage{array}            % multiline table cells; somehow
\usepackage{tabularx}         % better tables
% \usepackage[table]{xcolor}    % colored table cells  % incompatible with sigconf
\usepackage{colortbl}
\newcolumntype{Y}{>{\centering\arraybackslash}X}	% centered column type for tabularx

% stuff for piecewise functions
\usepackage{mathtools}          %loads amsmath as well
\DeclarePairedDelimiter\Floor\lfloor\rfloor
\DeclarePairedDelimiter\Ceil\lceil\rceil

\DeclareMathOperator*{\argmin}{arg\,min} % argmin
\DeclareMathOperator*{\argmax}{arg\,max} % argmax

\usepackage{pbox}   % for trick to force linebreaks in table cells

\usepackage{setspace}
% \usepackage{setspace}

\usepackage{booktabs} % acm recommended, and for pandas latex tables

%-------------------------------------------------------------- Algorithm setup

\usepackage{algorithm}
\usepackage[noend]{algpseudocode} % I think this removes trailing "end {if,for,while}"

% \algnewcommand{\LineComment}[1]{\State \(\triangleright\) #1} % left-aligned comments
% \algnewcommand{\SideComment}[1]{\(//\) #1}
\algnewcommand{\COMMENT}[2][.5\linewidth]{\leavevmode\hfill\makebox[#1][l]{//~#2}}
\algnewcommand{\LineComment}[1]{\State \(//\) #1}	% left-aligned comments
\algnewcommand\RETURN{\State \textbf{return} }


%-------------------------------------------------------------- Figures setup

% \usepackage[pdftex]{graphicx}
\usepackage{graphicx}
\usepackage[space]{grffile}   % allow spaces in file names
% declare the path(s) where your graphic files are
\graphicspath{{../figs/}}
% and their extensions so you won't have to specify these
\DeclareGraphicsExtensions{.pdf,.jpeg,.jpg,.png}


%\textfloatsep: space between last top float or first bottom float and the text (default = 20.0pt plus 2.0pt minus 4.0pt).
%\intextsep : space left on top and bottom of an in-text float (default = 12.0pt plus 2.0pt minus 2.0pt).
\setlength{\textfloatsep}{4pt}
\setlength{\intextsep}{4pt}

% \usepackage{caption}
% \usepackage[font={small,it}]{caption}
\usepackage[font={bf}]{caption}
% \setlength{\abovecaptionskip}{-2pt} % less space between captions and figures
\setlength{\abovecaptionskip}{4pt}	% less space between captions and figures
% \setlength{\belowcaptionskip}{-13pt}  % less space below captions
\setlength{\belowcaptionskip}{-10pt}	% less space below captions

\usepackage{paralist}
\setdefaultleftmargin{10pt}{10pt}{}{}{}{}

% \usepackage{changepage}
% \usepackage{tabulary}

\usepackage{outlines}
\usepackage{enumitem}
\newcommand{\ItemSpacing}{0mm}
\newcommand{\ParSpacing}{0mm}
\setenumerate[1]{itemsep={\ItemSpacing},parsep={\ParSpacing},label=\arabic*.}
% \setenumerate[2]{itemsep={\ItemSpacing},parsep={\ParSpacing},label=\arabic*.}
\setenumerate[2]{itemsep={\ItemSpacing},parsep={\ParSpacing}}

\usepackage[linewidth=1pt]{mdframed}
\mdfsetup{frametitlealignment=\center, skipabove=0, innertopmargin=1mm,
innerleftmargin=2mm, leftmargin=0mm, rightmargin=0mm}

%-------------------------------------------------------------- Miscellaneous setup

\newlength\myindent
\setlength\myindent{2em}
\newcommand\bindent{%
  \begingroup
  \setlength{\itemindent}{\myindent}
  \addtolength{\algorithmicindent}{\myindent}
}
\newcommand\eindent{\endgroup}

% remove unwanted space between paragraphs;
% it's set by the IEEE conference format, but no papers from this conference have it
% \parskip 0ex plus 0.2ex minus 0.1ex

% make vectors be bold instead of with arrows
\renewcommand{\vec}[1]{\mathbf{#1}}
% add 'mat' command to make matrices bold
\newcommand{\mat}[1]{\mathbf{#1}}

\DeclarePairedDelimiter\ceil{\lceil}{\rceil}
\DeclarePairedDelimiter\floor{\lfloor}{\rfloor}

\newtheorem{Definition}{Definition}[section]

% make *all* text 10pt
% \renewcommand{\footnotesize}{\normalsize}
% \renewcommand{\footnotesize}{\small}
% \renewcommand{\small}{\normalsize}


\begin{document}

\setcopyright{rightsretained}

%Conference
\acmConference[KDD 2017]{ACM SIGKDD}{August 2017}{Halifax, Nova Scotia Canada}
\acmYear{2017}
\copyrightyear{2017}
% \acmPrice{15.00}

% ================================================================
\title{Bolt: Accelerated Data Mining with Fast Vector Compression}
% ================================================================

\author{Davis W. Blalock}
% \orcid{1234-5678-9012}
\affiliation{%
  \institution{Computer Science and Artificial \\ Intelligence Laboratory}
  \institution{Massachusetts Institute of Technology}
  % \streetaddress{P.O. Box 1212}
  % \city{Dublin}
  % \state{Ohio}
  % \postcode{43017-6221}
}
\email{dblalock@mit.edu}

\author{John V. Guttag}
\affiliation{%
  \institution{Computer Science and Artificial \\ Intelligence Laboratory}
  \institution{Massachusetts Institute of Technology}
  % \streetaddress{P.O. Box 1212}
  % \city{Dublin}
  % \state{Ohio}
  % \postcode{43017-6221}
}
\email{guttag@mit.edu}

% ------------------------------------------------
\begin{abstract}
% ------------------------------------------------

% v0
% The need to compute dot products and distances between vectors of parameters is ubiquitous in machine learning. These computations are often responsible for most of an algorithm's running time, and storing the parameters is usually responsible for most of its space consumption.

% We describe an approximate method of computing dot products and distances that greatly reduces both time and space consumption with little or no loss in accuracy. Our approach is based on vector quantization, and entails replacing the original computation with lookups in a compact table of sufficient statistics about the vectors in question. Unlike similar techniques, our algorithm does not assume that vectors are static and is designed to be hardware-friendly.

% We show experimentally that our approach can be used to accelerate matrix multiplications, k-means clustering, nearest neighbor search, and maximum inner product search by an order of magnitude or more. Furthermore, our approach is faster than even hamming distance computation, which has explicit hardware support on the tested platforms.

% v3

% Storing such vectors accounts for most of the space usage of many algorithms, and operating on them accounts for most of the compute time. % Recently, vector quantization methods have shown great promise in reducing these costs. These methods compress raw floating-point vectors into a compact representation that can be used to quickly approximate scalar reductions such as Euclidean distances and dot products. However, the overhead associated with accurately compressing the vectors using these schemes makes them less practical for changing collections. % Moreover, as we show experimentally, the speedup they afford in carrying out reductions is often not significant.

Vectors of data are at the heart of machine learning and data mining. Recently, vector quantization methods have shown great promise in reducing both the time and space costs of operating on vectors. We introduce a vector quantization algorithm that can compress vectors up to $12\times$ faster than existing techniques while also accelerating approximate vector operations such as distance and dot product computations by over $10\times$. Because it can encode over two megabytes of vectors per millisecond (2 GB/s), it makes vector quantization cheap enough to employ in many more circumstances. As an example, using our technique to compute approximate dot products in a nested loop can multiply matrices faster than a state-of-the-art BLAS implementation, even when our algorithm must first compress the matrices.

In addition to showing the above speedups, we show experimentally that our approach can be used to accelerate nearest neighbor search and maximum inner product search by up to $140\times$ compared to floating point operations and $10\times$ compared to other vector quantization methods. Our approximate Euclidean distance and dot product computations are not only faster than those of related algorithms with slower encodings, but also faster than Hamming distance computations, which have direct hardware support on the tested platforms. We also assess the errors of our algorithm's approximate distances and dot products, and find that it is competitive with existing, slower vector quantization algorithms. % Finally, we offer theoretical guarantees regarding the performance of our algorithm.

% Our approach is to quantize one of the vectors and replace the original computation with lookups in a compact table of sufficient statistics about the other.

\end{abstract}

% ------------------------------------------------
% CCS taxonomy stuff / keywords
% ------------------------------------------------

% TODO get the right xml in here

\begin{CCSXML}
<ccs2012>
<concept>
<concept_id>10002950.10003648</concept_id>
<concept_desc>Mathematics of computing~Probability and statistics</concept_desc>
<concept_significance>500</concept_significance>
</concept>
<concept>
<concept_id>10002950.10003648.10003671</concept_id>
<concept_desc>Mathematics of computing~Probabilistic algorithms</concept_desc>
<concept_significance>300</concept_significance>
</concept>
<concept>
<concept_id>10002950.10003648.10003688.10003696</concept_id>
<concept_desc>Mathematics of computing~Dimensionality reduction</concept_desc>
<concept_significance>300</concept_significance>
</concept>
<concept>
<concept_id>10002950.10003705</concept_id>
<concept_desc>Mathematics of computing~Mathematical software</concept_desc>
<concept_significance>100</concept_significance>
</concept>
</ccs2012>
\end{CCSXML}

% \ccsdesc[500]{Scalability}
% \ccsdesc[300]{Information Retrieval}
% \ccsdesc[100]{Engineering Challenges}

\ccsdesc[500]{Mathematics of computing~Probability and statistics}
\ccsdesc[300]{Mathematics of computing~Probabilistic algorithms}
\ccsdesc[300]{Mathematics of computing~Dimensionality reduction}
\ccsdesc[100]{Mathematics of computing~Mathematical software}

\keywords{Vector Quantization, Scalability, Nearest Neighbor Search}

\maketitle

% ================================================================
\section{Introduction} \label{sec:intro}
% ================================================================


Matrix multiplication is among the most fundamental subroutines used in machine learning and scientific computing. As a result, there has been a great deal of work on implementing high-speed matrix multiplication libraries \cite{pytorch,eigen,tensorflow}, designing custom hardware to accelerate multiplication of certain classes of matrices \cite{eie,eyeriss,scnn,tpu}, scaling matrix multiplication across many machines \cite{distributedCoded, shortDot, entangledPolynomial, matmulCommunicationBounds}, and designing efficient Approximate Matrix Multiplication (AMM) algorithms under various assumptions and problem settings \cite{drineas_fast_2006,manne_fast_2014,ye_frequent_2016,mroueh_co-occuring_2016,bolt}.

We focus on the AMM task under the assumptions that the matrices are tall, relatively dense, and resident in a single machine's memory. In this setting, the primary challenge is minimization of the amount of CPU time required to approximate linear operations with a given level of fidelity.
%not reduction of disk or network usage [], efficient coordination between distributed workers [], maximization of a space-distortion tradeoff [], or reduction of asymptotic complexity []. Instead, it is minimization of the amount of CPU time required to approximate linear operations with a given level of fidelity.

This setting arises naturally in machine learning and data mining when one has a data matrix $\mat{A}$ whose rows are samples and a linear operator $\mat{B}$ one wishes to apply to these samples. $\mat{B}$ could be a linear classifier, linear regressor, or an embedding matrix, among other possibilities.

As a concrete example, consider the task of approximating a softmax classifier trained to predict image labels given embeddings derived from a neural network. Here, the rows of $\A$ are the embeddings for each image, and the columns of $\B$ are the weight vectors for each class. Classification is performed by computing the product $\A\B$ and taking the argmax within each row of the result.
In Figure~\ref{fig:fig1}, we see the results of approximating $\A\B$ using our method and its best-performing rivals \cite{hashjl, sparsePCA} on the CIFAR-10 and CIFAR-100 datasets.
\vspace{1mm}
\begin{figure}[h]
\begin{center}
\includegraphics[width=\linewidth]{amm/fig1}
\caption{Our method achieves a dramatically better speed-accuracy tradeoff than existing methods when approximating two linear classifiers.}
\label{fig:fig1}
\end{center}
\end{figure}
\vspace{-1mm}
% Each method yields a curve whose points are specific speedups and associated levels of classification accuracy. More speedup and more accuracy (up and to the right) is better. While we defer detailed discussion to Section~\ref{sec:results}, it is clear that our proposed approach significantly outperforms alternatives.

% In addition to achieving strong empirical performance, our method also has an

% In addition to achieving strong empirical performance,
Our method represents a significant methodological departure from most traditional approaches to this problem. Traditional AMM methods construct matrices $\V_A, \V_B \in \R^{D \times d}, d \ll D$ such that
\begin{align}
    \A \B \approx (\A \V_A) (\V_B^\top \B).
\end{align}
Often, $\V_A$ and $\V_B$ are sparse, embody some sort of sampling scheme, or have other structure such that these projection operations are faster than a dense matrix multiply. In short, these methods use linear functions to preprocess $\A$ and $\B$ and reduce the problem to exact matrix multiplication in a lower-dimensional space.
% reduce the AMM problem to exact matrix multiplication in a lower-dimensional space using linear functions to preprocess $\A$ and $\B$.

Our proposed method, \ours\footnote{Multiply-ADDitioN-lESS}, instead employs a \textit{nonlinear} preprocessing function and reduces the problem to table lookups. Moreover, in the case that $\B$ is known ahead of time---which happens when applying a trained linear model to new data, among other situations---\oursp does not require any multiply-add operations.
% This includes binary \texttt{XNOR} and \texttt{popcount} operations \cite{xnornet,dorefanet}, as well as the summation of shifted and masked scalars to approximate multiplication \cite{hackyQuasiMultiplies}. I.e., we are not merely approximating scalar multiplies at the bit manipulation level, but approximating matrix products at the algorithm level.

% An extension of this method common in the information retrieval literature is to define a nonlinear function $g(\cdot)$ such that
% \begin{align}
%     \A \B \approx g(\A \V_A) (\V_B^\top \B).
% \end{align}
% The $g(\cdot)$ function typically binarizes and induces structured sparsity in the resulting matrix. This allows the product of this matrix and $(\V_B^\top \B)$ to be computed using only additions, instead of multiply-adds. Moreover, if the sparsity is structured appropriately, dot products can be reduced to table lookups, with the indices into the tables given by the indices of the nonzeros.




% Our proposed method, \ours, instead preprocesses $\A$ and $\B$ with \textit{nonlinear} functions and reduce the problem to table lookups.


% These methods may also attempt to reduce the cost of each multiply-add by reducing the number of bits used to store each scalar. This works b



% Existing approaches to this task attempt to reduce the CPU time by performing fewer multiply-add operations or by reducing the cost of each such operation. The former typically entails projecting $\A$ and $\B$ into a lower dimensional space before multiplying them, and the latter typically entails reducing the number of bits used to store each scalar element.
% % This typically entails constructing a matrix $\V \in \R^{D \times d}, d < D$ such that
% % \begin{align}
% %     \A \B \approx (\A \V) (\V^\top \B).
% % \end{align}
% % Often, $\V$ is sparse or has structure such that these projection operations are faster than a dense matrix multiply. These methods may also attempt to reduce the cost of each multiply-add by reducing the number of bits used to store each scalar.

% % Instead of performing dimensionality or bitwidth reduction,
% In contrast, we introduce \ours, a method that can eliminate the multiply-add operations entirely when given time to preprocess $\B$. It does this by precomputing certain statistics about $\B$ and replacing the multiply-adds with table lookups. Crucially, these table lookups are not merely a different implementation of multiplication---as we discuss in section~\ref{sec:method}, they implement nonlinear functions. % that cannot be characterized as an algebraic ring.

Our method is most closely related to vector quantization methods used for similarity search (e.g., \cite{bolt,quickAdc,quickerAdc,pq,opq}). However, instead of using an expensive quantization function that requires many multiply-adds, we introduce a family of quantization functions that require no multiply-adds. %, including binary \texttt{XNOR} and \texttt{popcount} operations (though excluding implementation-level multiplies to compute memory addresses).
% TODO something about multiplication taking a lot of transistors vs table lookups.

Our contributions can be summarized as follows:
\vspace{-3mm}
\begin{itemize}\itemsep-1mm
    \item An efficient family of vector quantization functions that can encode over 100GB of data per second in a single CPU thread.
    % \item A provably good procedure for learning one of these functions from training data, in addition to other theoretical analysis.
    \item A high-speed summation algorithm for low-bitwidth integers that avoids upcasting, saturation, and overflow.
    \item An algorithm based on these functions for approximate matrix multiplication. Experiments across hundreds of diverse matrices demonstrate that this algorithm significantly outperforms existing alternatives.
\end{itemize}
\vspace{-3mm}

% ------------------------------------------------
\subsection{Problem Formulation} \label{sec:problemStatement}
% ------------------------------------------------

Let $\A \in \R^{N \times D}$ and $\B \in \R^{D \times M}$ be two matrices, with $N \gg D, M$, and $M$ not significantly larger than $D$. Given a computation time budget $\tau$, our task is to
construct three functions $g(\cdot)$, $h(\cdot)$, and $f(\cdot)$, along with constants $\alpha$ and $\beta$, such that
\begin{align} \label{eq:objective}
    \norm{\alpha f(g(\A), h(\B)) + \mat{\beta} - \A\B}_F < \eps(\tau) \norm{\A\B}_F
\end{align}
for the smallest error $\eps(\tau)$ possible. The constants $\alpha$ and $\beta$ are separated from $f(\cdot,\cdot)$ so that $f(\cdot,\cdot)$ can produce low-bitwidth outputs (e.g., in the range $[0, 255]$) even when the entries of $\A\B$ do not fall in this range.

% it produce low-bitwidth outputs

% The scalar $\alpha$ and bias matrix $\mat{\beta}$ must be known constants. These constants are separated from $f(\cdot,\cdot)$ so that it produce low-bitwidth outputs.

% , and are separated from $f(\cdot)$ so that it can produce low-bitwidth outputs that may not capture the range of values in $\A\B$. %Furthermore, $\alpha$ must be a power of $2$ so that multiplication can be replaced with bit shifting.\footnote{}
% to allow $f(\cdot)$ to

% . Moreover, the computation time $\tau(\eps)$ to run these three functions should be much smaller than the comuputation time $\tau_0$ to compute $\A\B$ directly.

We assume the existence of a training set $\tilde{\A}$, whose rows are drawn from the same distribution as the rows of $\A$. This is a natural assumption in the case that rows of $\A$ represent examples in training data, or structured subsets thereof (such as patches of images). This assumption is common in the information retrieval literature \cite{bolt,pairq,quip}, but is a significant departure from most theoretical work.

% We also assume that the cost of computing $h(\B)$ negligible provided that $h(\B) \in O(MD)$. One sufficient condition for this is $\B$ being provided ahead of time (as is the case when $\B$ represents fixed model weights). A second is that $\B$ has sufficiently few columns compared to the number of rows in $\A$ (i.e., $M \ll N$).


% ================================================================

% These methods transform the two matrices such that one becomes a collection of lookup tables and the other indices into those tables. Once transformed, the matrix multiplication \textit{per se} requires only table lookups. We adopt this same pattern, but replace the multiply-intensive transform with an efficient alternative. This replacement is enabled through our introduction of a fast and trainable family of quantization functions.

%  which already compute matrix products \textit{per se} with no multiplies. However, they require many multiplies in

% our key insight is that the quantization codebooks used by these methods


% The basic insight behind our method is that vector quantization can be carried

% In machine learning specifically, it is often the bottleneck in deep neural networks, which have emerged as an indispensible class of models for many tasks.



% multiplications [] or reduces the cost of each multiplication [], with the most extreme case of the latter being multiplication of binary values [].

% Most existing approaches to this task project $\mat{A}$ and $\mat{B}$ into a lower-dimensional space and then perform an exact matrix multiplication in this reduced space []. Concretely

% Most existing approaches to this task project $\mat{A}$ and $\mat{B}$

 % They may also reduce the number of bits used to store elements of $\A$ and $\B$ or


% We formalize these notions in the following section, but this corresponds to the common scenario in machine learning wherein one has a matrix whose

% Tall and (relatively) dense arise naturally in machine learning,
% real-world assumptions, such as matrices
% We focus on approximate multiplication of tall, dense matrices that already reside in memory.
% More specifically, consider two matrices $\mat{A} \in \R^{N \times D}$ and $\mat{B} \in \R^{D \times M}$, and their product $\mat{C} \triangleq \mat{A}\mat{B}$. We consider the setting in which $N \gg D, M$, and $M$ is not significantly larger than $D$. Furthermore, we focus on the typical real-world case in which $N$, $D$, and $M$ are all small enough that asymptotic complexity is not an adequate characterization of speed.
% Such matrices arise naturally in machine learning, when $A$ is a matrix of data and $B$ is a matrix of model weights or some other sort of linear projection. % It is also common in this setting to have a matrix of training data $\mat{\tilde{A}}$ whose rows share the same distribution as those of $\mat{A}$, and we assume access to such a matrix.

% By tall, we are referring to the typical real-world case in which asymptotic complexity is a poor definition of speed.

% In contrast to most existing AMM work, we also consider various levels of prior knowledge about the distributions from which the matrices are drawn. In the least informed case, nothing is known about either matrix until the instant they are provided. In the most informed case, one matrix is known ahead of time and the other has rows drawn from a distribution from which we have numerous i.i.d. samples.

% Provided that one matrix is known ahead of time, our method requires zero multiplication operations, aside from possible pointer arithmetic at the implementation level.
%This is in contrast to existing work, which either reduces the number of multiplications [] or reduces the cost of each multiplication [], with the most extreme case of the latter being multiplication of binary values [].


% % ================================================================
% \section{Definitions and Problem} \label{sec:problem}
% % ================================================================

% \input{problem.tex}

% ================================================================
\section{Related Work} \label{sec:relatedWork}
% ================================================================


Accelerating vector operations through compression has been the subject of a great deal of research in the computer vision, information retrieval, and machine learning communities, among others. Our review will necessarily be incomplete, so we refer the reader to \cite{learningToHashSurvey, hashingSimilaritySurvey} for detailed surveys. %  and focus on methods not discussed in Section~\ref{sec:method}

Many existing approaches in the computer vision and information retrieval literature fall into one of two categories \cite{hashingSimilaritySurvey}: binary embedding and vector quantization. Binary embedding techniques seek to map vectors in $\mathbb{R}^J$ to $B$-dimensional Hamming space, typically with $B < J$. The appeal of binary embedding is that a $B$-element vector in Hamming space can be stored in $B$ bits, affording excellent compression. Moreover, the \texttt{popcount} instruction present on virtually all desktop, smart phone, and server processors can be used to compute Hamming distances between 8 byte vectors in as little as three cycles. This fast distance computation comes at the price of reduced representational accuracy for a given code length \cite{opq,hashingSimilaritySurvey} . He et al. \cite{opq} demonstrated that the popular binary embedding technique of \cite{ITQ} is a more constrained version of their vector quantization algorithm, and that the objective function of another state-of-the art binary embedding \cite{isohash}, can be understood as maximizing only one of two sufficient conditions for optimal encoding of Gaussian data.

Vector quantization approaches yield lower errors than binary embedding for a given code length, but entail slower encoding and distance computations. The simplest and most popular vector quantization method is K-means, which can be seen as encoding a vector as the centroid to which it is closest. A generalization of K-means, Product Quantization (PQ) \cite{pq}, splits the vector into $M$ disjoint subvectors and runs K-means on each. The resulting code is the concatenation of the codes for each subspace. Numerous generalizations of PQ have been published, including Cartesian K-means \cite{cartesianKmeans}, Optimized Product Quantization \cite{opq},  Generalized Residual Vector Quantization \cite{grvq}, Additive Quantization \cite{aq}, Composite Quantization \cite{cq}, Optimized Tree Quantization \cite{otq}, Stacked Quantizers \cite{stackedQuantizers}, and Local Search Quantization \cite{lsq}. The idea behind most of these generalizations is to either rotate the vectors or relax the constraint that the subvectors be disjoint. Collectively, these techniques that rely on using the concatenation of multiple codes to describe a vector are known as Multi-Codebook Quantization (MCQ) methods. % , Residual Vector Quantization [],

An interesting hybrid between binary embedding and vector quantization is the recent Polysemous Coding of Douze et al. \cite{polysemous}. This encoding uses product quantization codebooks optimized to also function as binary codes, allowing the use of Hamming distances as a fast approximation that can be refined for promising nearest neighbor candidates. % Like our own algorithm, it seeks the best of both binary embedding speed and vector quantization accuracy. Our technique obtains this by speeding up the vector quantization distance computations directly, obviating the need for multiple passes and the possibility of false dismissals from unrepresentative Hamming distances.

The most similar vector quantization-related algorithm to our own is that of \cite{simdpq}, which also vectorizes PQ distance computations. However, their method requires hundreds of thousands or millions of encodings to be sorted lexicographically and stored contiguously ahead of time, as well as scanned through serially. This is untenable when the data is rapidly changing or when using an indexing structure, which would split the data into far smaller partitions. Their approach also requires a second refinement pass of non-vectorized PQ distance computations, making their reported speedups significantly lower than our own.

In the machine learning community, accelerating vector operations has been done primarily through (non-binary) embedding, structured matrices, and model compression. Embedding yields acceleration by reducing the dimensionality of data while preserving the relevant structure of a dataset overall. There are strong theoretical gaurantees regarding the level of reduction attainable for a given level of distortion in pairwise distances \cite{jl, fastJL, jlIsTight}, as well as strong empirical results \cite{superBitLSH, compressiveMining}. However, because embedding \textit{per se} only entails reducing the number of floating-point numbers stored, without reducing the size of each, it is not usually competitive with vector quantization methods. It is possible to embed data before applying vector quantization, so the two techniques are complementary.

An alternative to embedding that reduces the cost of storing and multiplying by matrices is the use of structured matrices. This consists of repeatedly applying a linear transform, such as permutation \cite{adaptiveFastfood}, the Fast Fourier Transform \cite{orthogonalRandomFeatures}, the Discrete Cosine Transform \cite{acdc}, or the Fast Hadamard Transform \cite{crossPolytope, structuredSpinners}, possibly with learned elementwise weights, instead of performing a matrix multiply. These methods have strong theoretical grounding \cite{structuredSpinners} and sometimes outperform non-structured matrices [adaptiveFastfood]. They are orthogonal to our work in that they bypass the need for a matrix entirely, while our approach can accelerate operations in which a matrix is needed.

Another vector-quantization-like technique common in machine learning is model compression. This typically consists of some combination of 1) restricting the representation of variables, such as neural network weights, to fewer bits \cite{lossAwareDnnBinarization}; 2) reusing weights \cite{hashnet}; 3) pruning weights in a model after training \cite{diversityNets, learningWeightsConnections}; and 4) training a small model to approximate the outputs of a larger model \cite{rnnDarkKnowledge}. This has been a subject of intense research for neural networks in recent years, so we do not believe that our approach could yield smaller neural networks than the current state of the art. Instead, our focus is on accelerating operations on weights and data that would otherwise not have been compressed.% matching the form of Eq~\ref{eq:distFuncForm}.

% In addition to all of the above approaches, there is the possibility of running algorithms on GPUs [], clusters [], ASICs [], or FPGAs []. Since such alternate hardware could accelerate both our own algorithm and the most similar comparisons, we do not address this possibility further.

% % In addition to these methods, there has been a great deal of work on compressing data


% % Our algorithm is similar to above multi-codebook quantization methods,

% In addition to these methods, there has been a great deal

% Vector quantization is an extremely well-studied problem, so our review will necessarily be incomplete. We refer the reader to [learningToHash, thatOtherSurvey] for detailed surveys. The classic K-means algorithm [] is by far the most widely-used vector quantization technique.

% Great deal of work on vector quantization
%     -mostly on generating better encoding algorithms
%         -typically through complex procedures that require tons of encoding time
%             -eg, state of the art (LSQ) encoding takes over a ms *per vector*; we're like a ten thousand times faster than this
%         -sometimes through direct integration with an index, such as the IMI (LOPQ does this)
%     -our scanning algorithm can use any of these encoding schemes that can spit out 4bit codes, and our preprocessing can help with basically any of them
%     -most similar to ours is that VLDB paper that used the same machine instructions; however, it required huge sets of static vectors that could be sorted ahead of time


%     -also lots of work on compressing neural nets
%         -replacing raw mats with structured mats, such as adaptive fastfood transform or the DCT ones
%             -heuristic (?) weight-sharing like hashnets
%             -zelda diversity nets
%         -quantization, sparsity
%             -pruning post-hoc
%             -learning with only quantized reprs
%             -ASIC acceleration
%         -other stuff I don't know that well

%     -sketching / embedding
%         -pretty sure there's some sketch for euclidean distance
%         -sketching and embedding equivalent for norms

%         -sketching for stats such as number of uniques




% ================================================================
% \vspace{-2mm}
\section{Method} \label{sec:method}
% ================================================================


As mentioned in the problem statement, our goal is to construct a distance function $\hat{d}$ and two encoding functions $g$ and $h$ such that $\hat{d}(g(\vec{q}), h(\vec{x})) \approx d(\vec{q}, \vec{x})$ for some ``true'' distance function $d$. To explain how we do this, we first begin with a review of Product Quantization [], and then describe how our method differs.

\subsection{Background: Product Quantization}

Recall that, by assumption, the function $d$ can be written as:
\begin{align*}
        d(\vec{q}, \vec{x}) = f(\sum_{j=1}^J d_j(q_j, x_j))
\end{align*}
where $f: \mathbb{R} \rightarrow \mathbb{R}$, $d_j: \mathbb{R}^J \times \mathbb{R}^J \rightarrow \mathbb{R}$. Now, suppose one has a partition $\mathcal{P} = \{p_1,\ldots,p_M \}$ of the indices $j$, such that $i \ne j \implies p_i \cap p_j = \emptyset$ and $\bigcup_m p_m = \{1,\ldots,D\}$. The argument to $f$ can then be written as:
\begin{align}
        \sum_{m=1}^M \sum_{j \in p_m} d_j(q_j, x_j)
            = \sum_{m=1}^M d_m(\vec{q}_m, \vec{x}_m)
\end{align}
where $\vec{q}_m$ and $\vec{x}_m$ are the vectors formed by gathering the indices of $\vec{q}$ and $\vec{x}$ at the indices $j \in p_m$, and $d_m$ sums the relevant $d_j$ functions. The idea of Product Quantization (PQ) is to replace each $x_m$ with one vector $\vec{c}_{m}$ from a \textit{codebook} set $\mathcal{C}_m$ of possibilities. That is: % (\vec{q_m}, \vec{x}) = \sum_{j = 1}^|p_m d_j(q_j, x_j)$.
\begin{align} \label{eq:pqDistNoLut}
        \sum_{m=1}^M d_m(\vec{q}_m, \vec{x}_m) \approx \sum_{m=1}^M d_m(\vec{q}_m, \vec{c}_{m})
\end{align}
This allows one to store only the identity of the codebook vector chosen, instead of the elements of the original vector $\vec{x}_m$. More formally, the PQ encoding function $h(\vec{x})$ for a given set of codebooks $\mathcal{C} = \{\mathcal{C}_1,\ldots,\mathcal{C}_C\}, \mathcal{C}_m = \{\vec{c}_{1m},\ldots,\vec{c}_{Cm}\}$ is:
\begin{align}
    h(\vec{x}) = \langle i_1,\ldots,i_M \rangle,  i_m = \argmin_i d(\vec{c}_{im}, \vec{x_m})
\end{align}

Using a fixed set of codebooks also enables construction of a fast query encoding $g$ and distance approximation $\hat{d}$. Specifically, let the query encoding space $\mathcal{G}$ be $R^{M \times C}$ and define $\mat{D} = g(\vec{q})$ as: % $g: \mathbb{R}^J \rightarrow R^{M \times C}$ as:
\begin{align}
    \mat{D}_{im} \triangleq d_m(\vec{q}_m, \vec{c}_{im})
\end{align}
Then we can rewrite the approximate distance on the right hand side of ~\ref{eq:pqDistNoLut} as:
\begin{align} \label{eq:pqDist}
        \sum_{m=1}^M \mat{D}_{im}, i = h(x)_m
\end{align}
In other words, the distance can be reduced to a sum of precomputed distances between $\vec{q_m}$ and the codebook vectors $\vec{c}_{im}$ used to approximate $\vec{x}$. Finally, by reintroducing $f$, one can now define:
\begin{align} \label{eq:pq_dhat}
    \hat{d}(g(\vec{q}), h(\vec{x})) \triangleq f(\sum_{m=1}^M \mat{D}_{im}, i = h(x)_m)
\end{align}
If $M \ll D$ and $C \ll |\mathcal{X}|$, then computation of $\hat{d}$ is much faster than computation of $d$ given the $g(\vec{q})$ matrix $\mat{D}$ and data encodings $\mathcal{H} = \{h(\vec{x}), \vec{x} \in \mathcal{X} \}$.

The total computational cost of product quantization is $\Theta(CJ)$ to encode each $\vec{x}$, $\Theta(CJ)$ to encode each query $\vec{q}$, and $\Theta(M)$ to compute the approximate distance between a given $\vec{q}$ and $\vec{x}$. Because queries must be encoded before distance computations can be performed, this means that the cost of computing the distances to the $N$ database vectors $\mathcal{X}$ when a query is received is $\Theta(CJ) + \Theta(NM)$. Lastly, since codebooks are learned using k-means clustering, the time to learn the codebook vectors is $O(CNJT)$, where $T$ is the number of k-means iterations. In all works of which we are aware, $C$ is set to $256$ so that each element of $h(\vec{x})$ can be encoded as one byte. Further, the subspaces $p_m$ are the contiguous blocks of $J/M$ dimensions, possibly after a random permutation.

In certain cases, Product Quantization is nearly an optimal encoding scheme. Specifically, under the assumptions that 1) $\vec{x}_m \sim MVN(\vec{\mu}_m, \Sigma_m)$, 2) $Pr[\vec{x}] = \prod_m Pr[\vec{x}_m]$; and 3) $\forall_m |\Sigma_m| = \textit{const}$, PQ achieves the information-theoretic lower bound on code length for a given quantization error [OPQ]. In words, this means that PQ encoding is optimal if $x$ is drawn from a multivariate gaussian and the subspaces $p_m$ are independent and have the same variance.

In practice, however, most datasets are not Gaussian and their subspaces are neither independent nor homoscedastic. Consequently, many works have generalized PQ to capture relationships across subspaces or decrease the dependenies between them [][][][]. One work of particular note is Optimized Product Quantization (OPQ). OPQ exploits the fact that Euclidean distances and dot products are invariant to rotatations by learning a rotation that reduces the PQ quantization error.

Talk about OPQ more here if we have time to plug in our modified version of it.

\subsection{Bolt}

Bolt is similar to Optimized Product Quantization but differs in three key ways:
\begin{enumerate}
\item It constrains the rotation matrix such that vectors can be rotated more quickly.
\item It uses far fewer centroids.
\item It uses an approximate query encoding distance matrix $\mat{D}$.
\end{enumerate}

Change (1) directly increases the speeds of $g$ and $h$, as well as training, while (2) and (3) work together to increase the speeds of $g$, $h$, and $\hat{d}$.

\subsubsection{Constrained Rotation Matrix}

Write this if we end up using this approach. Otherwise remove change 1. We're just gonna make the rotation matrix block diagonal so we can run OPQ and in disjoint subspaces and apply several small rotations instead of one big, slow one.

\subsubsection{Fewer Centroids and Approximate Distances}

Change (2), reducing the size of codebooks $C$, directly reduces the constant factor associated with both query and data encoding. Although this is novel insofar as virtually all other papers have taken for granted the value $C = 256$ to the best of our knowledge, this is a simple change of parameter setting and we do not discuss it further.

More importantly, changes (2) and (3) together allow accelerated computation of $\hat{d}$. Specifically, if we fix $C = 16$, each of the $M$ codes in the data encoding $h(\vec{x})$ can be expressed using 4 bits. At the same time, if we quantize each entry of $\mat{D}$ to a single byte, instead of a 32-bit or 64-bit float, the rows of $\mat{D}$ can be fit in 16 bytes. This means that each $i \in h(\vec{x})$ is a 4-bit index into a 16-byte array. On virtually all personal computers, servers, CUDA-enabled GPUs, tablets, and smart phones, lookups into 16B arrays can be vectorized; i.e., some number $V$ of them (typically 16, 32, or 64) can be performed by the processor simultaneously\footnote{The relevant instructions are \texttt{vpshufb} on x86 [], \texttt{vtbl} on ARM [], \texttt{vperm} on PowerPC [], and \texttt{\_\_shfl} on CUDA [], among others on less popular architectures.}. Moreover, on CPUs, the entire column of $\mat{D}$ in question can be stored in a SIMD register, obviating the need to access L1 cache. In theory then, for $V$ simultaneous lookups, this affords over a $V$-fold speedup in the computation of $\hat{d}$. The speedup is lower in practice---see Section~\ref{sec:results}).

It is worth noting that [] used a similar vectorization strategy to accelerate PQ distance computations. However, their method requires hundreds of thousands or millions of encodings to be sorted lexicographically and stored contiguously ahead of time, as well as scanned through serially. This is untenable when the data is rapidly changing or when using an indexing structure, which would split the data into far smaller partitions. Their approach also requires a second refinement pass of non-vectorized PQ distance computations for encodings that might correspond to nearest neighbors or are otherwise of interest, since it computes only a lower bound on the PQ approximate distance. Probably move this to related work. % In practice, this theoretical speedup is not attained because of the loads, stores, and other instructions necessary to scan through the data, but a factor of $5$ or more is common (c.f. Section~\ref{sec:results}).

Unfortunately, it is not obvious how to quantize the elements of the distance matrix $\mat{D}$ in a manner that captures the full range of distances in the matrix. Or maybe it is; I don't have a formal experiment showing this, particularly when you upcast to uint16s as soon as you do the lookup. Either way, write a paragraph about how we end up doing it.

Maybe pseudocode for the query encoding and/or dist computations here? I think the latter would just add confusion because we have to use a weird partially-column-major storage layout that I'd rather not walk people through.

Worth switching from D as dimensionality to J or something? The goal is to get the distance matrix $\mat{D}$ to instead be $\mat{D}$. Also to decouple $d$ as a distance function from $D$ which has little to do with it.

% modern x86 processors [vpshufb], ARM processors [vtbl], PowerPC processors [vperm], and CUDA GPUs [__shfl], which together comprise the overwhelming majority of personal computers, servers, and smart phones, lookups into 16B arrays can be vectorized; i.e., 16 or, more commonly, 32 of them can be carried out simultaneously.

% Product quantization can yield high accuracy, but depends upon the characteristics of the data. In particular, because it quantizes subspaces independently, it cannot exploit mutual information between variables in different subspaces. When there is little or none of this information, however,



 % When there is a great deal of this information, it requires longer encodings (larger $M$) than

% with each vector in the codebook corresponding to one centroid. To facilitate exposition, we will henceforth refer to these vectors as \textit{centroids}, and the indices $i$ identifying them as \textit{codes}.

% Computing If the set $\mathcal{X}$ is fixed, then the data encodings can be precomputed.

% TODO: index in above shouldn't be i; it should be $\hat{x}_m$






% TODO put this back in when we transition to OPQ
% Moreover, under the assumptions that 1) $\vec{x}_m \sim MVN(\vec{\mu}_m, \Sigma_m)$, 2) $Pr[\vec{x}] = \prod_m Pr[\vec{x}_m]$; and 3) $\forall_m |\Sigma_m| = const$, this formulation achieves the information-theoretic lower bound on code length for a given squared error [OPQ]. In words, this means that PQ encoding is optimal if $x$ is a multivariate gaussian and the subspaces $p_m$ are independent and have the same variance.



%  1) $x \sim MVN(\mu, \Sigma)$, and therefore $x_m \sim MVN(\mu_m, \Sigma_m)$; 2) $\Sigma_{ij} = 0$ for all $i, j$ in different subspaces $p_m$; and 3) $\forall_m |\Sigma_m| = const$, where $\Sigma_m$ is the covariance within subspace $p_m$, this formulation achieves the information-theoretic lower bound on code length for a given squared error [OPQ].

% overview
%     -as mentioned in the problem statement, our goal is to construct a distance (or similarity) function $\hat{d}(q, x)$ that approximates some true distance (or similarity) function d(q, x).

%     -Further recall that $\hat{d}$ need not operate on the original spaces Q and X, but can instead use transformed spaces $G = g(Q)$ and $H = h(X)$, so that its signature is instead $\hat{d}: G x H \rightarrow R$.

%     -our method entails offline learning of the functions $G$ and $H$ in a manner similar to existing Multi-Codebook Quantization (MCQ) techniques, and online computation of the distance

%     -we first describe the function $h(.)$, which quantizes the vectors X.

%     -we then describe the functions $g(.)$, which computes sufficient statistics about the vector q

%     -we finally describe the function $\hat{d}(.,.)$, which returns a distance based on $g(q)$ and $h(x)$.

% Quantization scheme
%     -product quantization background
%     -our novel permutation method

% Distance computation
%     -asymetric distance computation background





% ================================================================
\section{Experimental Results} \label{sec:results}
% ================================================================


To assess \ours's effectiveness in practice, we implemented both it and existing algorithms in C++ and Python. All of our code and raw numerical results are publicly available, along with additional experiments omitted from this section due to space constraints.\footnote{https://smarturl.it/\ours}. In order to avoid implementation bias, we built upon the source code provided by \citep{bolt}\footnote{https://github.com/dblalock/bolt}, which has the most in common with our method and includes highly tuned implementations of many algorithms to which we compare. All experiments use a single thread on a 2013 Macbook Pro with a 2.6GHz Intel Core i7-4960HQ processor. Unless stated otherwise, all timing results use five trials, with each trial reporting the lowest time it took the code to execute in 20 runs. We use the best, rather than average, since this is standard practice in performance benchmarking and is robust to the purely additive noise introduced by competing CPU tasks.

Because nearly all existing work on approximate matrix multiplication either focuses on special cases that do not satisfy our problem definition \cite{quickerAdc, pq, opq} or synthetic matrices, there is not a clear set of benchmark matrix multiply tasks to use. We therefore propose a collection of tasks that we believe are both reproducible and representative of many real-world matrices. To the best of our knowledge, our experiments use over an order of magnitude more matrices than any previous study.% These tasks include multiplication of hundreds of real-world matrices, which is, to the best of our knowledge, orders of magnitude more than have been used in any previous study.

% ------------------------------------------------
\subsection{Methods Tested}
% ------------------------------------------------
Recall that most baselines take the form of selecting a matrix $\V \in \R^{D \times d}, d < D$ such that $\A \B \approx (\A \V) (\V^\top \B)$. Here $d$ is a free parameter that adjusts the quality vs speed tradeoff. We therefore characterize these methods by how they set $\V$.

\begin{itemize}\itemsep0em
    \item PCA. Set $\V$ equal to the top principal components of the training matrix $\tilde{\A}$.
    \item SparsePCA \cite{sparsePCA}. Set $\V = \argmin_{\V} \min_{\mat{U}} \frac{1}{N_{train}} \frac{1}{2} \norm{ \tilde{\A} - \mat{U}\mat{V} }^2_F + \lambda \norm{\V}_1$. This is not the only dictionary learning formulation referred to as SparsePCA \cite{spcaSurvey1,spcaSurvey2}, but it is reasonably representative and is the only one with support in any major Python machine learning library.%\footnote{https://scikit-learn.org/stable/modules/generated/sklearn.decomposition.SparsePCA.html}.

    Because SparsePCA requires the tuning of both $d$ and $\lambda$ for each matrix, we allowed it tune these parameters \textit{on the test set} to ensure that insufficient hyperparameter tuning did not hamper its performance. See Appendix~\ref{sec:experimentDetails} for more information.

    % First, for any given $d$ value and level of sparsity, we report
    % For each matrix product, we tried $\lambda \in 2^i, i \in \{-5, -4, -3, -2, -1, 0, 1, 2, 3\}$ for each matrix product. To ensure that our results err on the side of optimism (and also speed up our experiments), we report the best
    % selected the lambda \textit{post-hoc}---i.e., we cherrypicked the best results on the test set rather than cross-validating,
    % \item FastJL \cite{fastjl}. Note that we don't time the fast hadamard transform step.
    \item HashJL \cite{hashjl}. $\V$ is zero except for a $\pm 1$ in each row, with both sign and position chosen uniformly at random.
    % \item OSNAP \cite{osnap}. The OSNAP sketch is essentially a collection of $s$ HashJL sketches, each of dimensionality $d / s$. Following \cite{iSVD}, we set $s = 4$. We also tried other values of $s$ and found that other values did not perform significantly better (except $s=1$, which reduces to HashJL).
    \item RandGauss \cite{lshOrig,E2LSH}. The entries of $\V$ are drawn i.i.d. from $\mathcal{N}(0, D^{-1})$.
    % \item OrthoGauss \cite{superbitLSH}. The entries of $\V$ initialized as in RandGauss, and then $\V$ is set to the $\mat{Q}$ matrix in a QR decomposition of a RandGauss matrix.
    % \item Rademacher \cite{rademacherJL}. The entries of $\V$ are i.i.d. Rademacher random variables scaled by $\frac{1}{\sqrt{D}}$. TODO is that the right scale?
    \item Bolt \cite{bolt}. As discussed in Section~\ref{sec:relatedWork} Bolt is the most similar to our own method. It is not optimized for small $M$, however, effectively performing a preliminary matrix product with $M=16$ as a preprocessing step.
    \item Exact Multiplication. We simply compute the matrix product $\A\B$ using a modern BLAS implementation. We also implemented our own matrix product function specialized for tall, skinny matrices. In all cases, we report the timings based on the better of these two functions for a given matrix product.
\end{itemize}

In addition to these baselines, we test two variations of our method:
\begin{itemize}
    \item \ours. This is the algorithm described in Section~\ref{sec:method}.
    \item \ours-PQ. This is a handicapped version of \oursp without the prototype optimization step. The disparity in performance between \oursp and \ours-PQ is the gain from optimizing the prototypes.
\end{itemize}

We also compared to many additional methods (see Appendix~\ref{sec:experimentDetails}), but omit their results since they were not competitive with the listed baselines.

% ------------------------------------------------
\subsection{How fast is \ours?}
% ------------------------------------------------

We begin by profiling the raw speed of our method and the most similar baselines. In Figure~\ref{fig:encodeSpeed}, we time the $g(\A)$ functions for $\A$ matrices with $2^{15}$ rows and varying number of columns $D$. Following \cite{bolt}, we also vary the size of the row encodings, profiling 8, 16, and 32 bytes per row. \oursp is not only faster than existing methods, but faster than the machine's memory bandwidth. This is possible because it only reads $O(C)$ columns, and $C$ can be lower than $D$.

\begin{figure}[h]
\begin{center}
\includegraphics[width=\linewidth]{amm/encode_speed}
\caption{\oursp encodes the matrix $\A$ orders of magnitude more quickly than existing vector quantization methods.}
\label{fig:encodeSpeed}
\end{center}
\end{figure}

We also profile the speed of our aggregation function $f(\cdot, \cdot)$ using the same baselines as \citet{bolt}. Though less dramatic a speedup than for the encoding function, we see that our average-based aggregation is significantly faster than the upcasting-based method of Bolt, its nearest rival.

\begin{figure}[h]
\begin{center}
\includegraphics[width=\linewidth]{amm/scan_speed}
\caption{Given the encoded matrices, \oursp computes the approximate output twice as fast as the fastest existing method.}
\label{fig:scanSpeed}
\end{center}
\end{figure}

% ------------------------------------------------
\subsection{Softmax Classifier}
% ------------------------------------------------

As described in section~\ref{sec:intro}, we approximated linear classifiers on the widely used CIFAR-10 and CIFAR-100 datasets \cite{cifarDsets}. The classifiers use as input features the 512-dimensional activations of an open-source, VGG-like neural networks trained on each dataset \cite{cifarVgg}. The matrices $\A$ are the $10000 \times 512$-dimensional floating point activations for the full test sets, and the matrices $\B$ are each original network's final dense layer. The $50000 \times 512$-dimensional activations from the training set served as the training matrices $\tilde{\A}$.

As shown in Figure~\ref{fig:cifar}, \oursp significantly outperforms all existing methods, achieving near-perfect accuracy more than an order of magnitude faster than brute force.

\begin{figure}[h]
\begin{center}
\includegraphics[width=\linewidth]{amm/cifar_Speedup_Accuracy}
\caption{\oursp achieves a far better speed-accuracy tradeoff than any existing method when approximating two softmax classifiers.}
\label{fig:cifar}
\end{center}
\end{figure}

% ------------------------------------------------
\subsection{Kernel-based classification}
% ------------------------------------------------

To assess the efficacy of our method on a larger and more diverse set of datasets than simply CIFAR-10 and CIFAR-100, we trained kernel classifiers on the datasets from the UCR Time Series Archive \cite{UCRArchive2018}. In addition to being a large, public corpus of over a hundred datasets from a huge variety of different domains, it also has the advantage that it can be used to produce matrix multiplication tasks of a fixed size. This is necessary for meaningful comparison of performance across datasets. We constructed training and test matrices $\tilde{\A}$ and $\A$ by resampling each time series in each dataset's train and test set to a length of $320$ (the closest multiple of 32 to the median length of 310). We obtained the matrix $\B$ for each dataset by running Stochastic Neighbor Compression \cite{snc} on the training set with an RBF kernel of bandwidth one. We set the number of returned neighbors to 128 (results with 64 and 256 were similar), yielding a $\B$ matrix of size $320 \times 128$. See Appendix~\ref{sec:experimentDetails} for additional details.

This is not the state-of-the-art means of classifying time series, but it does yield fixed-sized matrices and is representative of several modern techniques for constructing highly efficient classifiers \cite{snc,dsnc,bnc,protonn}. This setup also complements the CIFAR results by being an extremely difficult task, since Stochastic Neighbor Compression has already optimized the classifer to avoid redundancy. % since the distances between time series are often far smaller than their euclidean lengths; this property means that even small amounts of approximation error are sufficient to change predictions. It is also difficult because the $\B$ matrix has been optimized to avoid redundancy, so any acceleration

As shown in Figure~\ref{fig:ucr}, \oursp is significantly faster at a given level of accuracy. A counter-intuitive result, however is that optimization of the prototypes is occasionally worse than not optimizing them (see the red line dipping below the blue one in the lower subplot). Since the optimization strictly increases the expressive power, we believe that this is a product of overfitting and could be corrected were we to tune the ridge regression's parameter (instead of always using $\lambda = 1$). This subplot also reveals that the most stringent degredation requirements can sometimes only be satisfied by PCA at a low speedup, since this is the only curve close to $1$.
% , at high accuracy preservation thresholds,  % However as indicated by \oursp never approaching a fraction of 1 on the bottom subplot, \oursp does not consistently preserve the full accuracy. This suggests that, for difficult classification problems in which almost no accuracy can be sacrificed, \outsp is often not the best choice. The only methods that reliably preserve nearly the full original accuracy are PCA with a small speedup and, with certain parameter settings, Bolt.

\begin{figure}[h]
\begin{center}
\includegraphics[width=\linewidth]{amm/ucr2_Speedup_Relative Accuracy_rbf}
\caption{Fraction of UCR datasets for which each method preserves a given fraction of the original accuracy, versus the method's degree of speedup. \oursp enables much greater speedups for a given level of accuracy degredation}
\label{fig:ucr}
\end{center}
\end{figure}

% ------------------------------------------------
\subsection{Image Processing}
% ------------------------------------------------

To test the extreme limits of \ours, we benchmarked the various techniques' ability to apply small filters to images. As representative examples we chose $3 \times 3$ Sobel filters and $5 \times 5$ gaussian filters. The former are common high-pass filters and the latter are common low-pass filters. We took the first 10 images from the first 50 classes of the Caltech101 dataset as the training set, and the first 10 images from the remaining 51 classes as the test set. We constructed the $\A$ matrices by extracting each patch of each image in

To allow meaningful speed comparisons across images, we resized and center cropped each image to $224 \times 224$ as commonly done in image classification pipelines \cite{resNet,resnet2,densenet}. We then extracted sliding windows of the appropriate size and used each (flattened) window as one row of $\tilde{\A}$ or $\A$. We similarly flattened the filters, with each set of coefficients forming one column of $\B$. In both cases, $\B$ has two columns---this is because using a single filter would mean timing a matrix-vector product instead of a matrix-matrix product. Moreover, because Sobel filters are often used in horizontal and vertical pairings, and Gaussian filters are often used together to perform difference-of-Gaussians transforms. See Appendix~\ref{sec:experimentDetails} for additional details.

% \frac{1}{NM}
We report the normalized mean-squared error (NMSE), which is defined as $\norm{\mat{\hat{C}}_{i,j} - \A\B}^2_F / \norm{\A\B}^2_F$, where $\hat{\mat{C}}$ is the method's approximation of $\A\B$. An NMSE of 0 is perfect and an NMSE of 1 corresponds to always predicting 0.

In Figure~\ref{fig:caltech}, we see that it is only \oursp that offers any advantage over exact matrix products. This is likely due to the fact that two columns afford almost no time to amortize preprocessing costs for $\A$; indeed, rival vector quantization methods cannot logically do less worth than brute force in this setting, and dense linear methods can only save work by embedding rows of $\A$ in one-dimensional space.

\oursp performs much worse on the high-pass filters (top) than the low-pass filters (bottom). This is likely because the former produce outputs with variance that is orders of magnitude lower than that of the original image; this means that the NMSE denominator, $\norm{\A\B}^2_F$ is extremely small.% relative to $\norm{\A}_F$ and $\norm{\B}_F$.

% a method must preserve nearly all the information about the image to get $\norm{\mat{\hat{C}}_{i,j} - \A\B}^2_F$

% Notes
%   -CHW order
%   -resized and center cropped to 224x224 as is common in deep learning
%   -stride of (2, 2) on training set because

\begin{figure}[h]
\begin{center}
\includegraphics[width=\linewidth]{amm/caltech_Speedup_1 - NMSE}
\caption{Despite having only $M=2$ columns in the matrix $\B$, \oursp still achieves a significant speedup with reasonable accuracy. It is, however, much more effective for approximating a 5x5 low-pass filter (bottom) than a 3x3 high-pass filter (top). Methods that did not simultanesouly perform better than chance and provide any speedup over exact matrix multiplication are not shown.}
\label{fig:caltech}
\end{center}
\end{figure}

% % ------------------------------------------------
% \subsection{Summary}
% % ------------------------------------------------

% Our method without \textit{any} hyperparameter tuning outperforms its nearest rival with all its hyperameters tuned \textit{on the test set}.


% ================================================================
\section{Conclusion} \label{sec:conclusion}
% ================================================================

% We have described Bolt, a vector quantization algorithm that rapidly compresses large collections of vectors and enables computation of approximate Euclidean distances and dot products directly on the lossily-compressed representations. Bolt both compresses data and computes these distances/products significantly faster than similar existing algorithms. Its approximate computations are up to $20\times$ faster than the exact computations on the original floating-point numbers but up to $99\%$+ correlated with the true values. Moreover, Bolt can achieve up to $15\times$ compression at this level of accuracy. Furthermore, because Bolt compression is fast enough to be nearly free, it can be used on rapidly-changing data with little or no overhead. This also means Bolt can easily be embedded in other algorithms, such as k-means, to dramatically speed them up without deteriorating the quality of the results.

% It is our hope that Bolt will be used in many production systems to greatly reduce storage and computation costs for large, real-valued datasets.

We described Bolt, a vector quantization algorithm that rapidly compresses large collections of vectors and enables fast computation of approximate Euclidean distances and dot products directly on the compressed representations. Bolt both compresses data and computes distances and dot products more than $10\times$ faster than existing algorithms, making it advantageous both in read-heavy and write-heavy scenarios. Its approximate computations can be over $140\times$ faster than the exact computations on the original floating-point numbers, while maintaining correlations with the true values of over $.95$. Moreover, at this level of correlation, Bolt can achieve $10$-$200\times$ compression or more. These attributes make Bolt ideal as a subroutine in algorithms that are amenable to approximate computations, such as nearest neighbor search or maximum inner product search. % Because Bolt compression is extremely fast, it is ideal for rapidly-changing data and can cheaply be embedded as a subroutine in other algorithms that afford approximate computations, such as nearest neighbor or maximum inner product search.

It is our hope that Bolt will be used in many production systems to greatly reduce storage and computation costs for large, real-valued datasets.


% just say can achieve X compression while maintaining corr of .85

% We described Bolt, a vector quantization algorithm that rapidly compresses large collections of vectors and enables fast computation of approximate Euclidean distances and dot products directly on the compressed representations. Bolt both compresses data and computes distances and products significantly faster than existing algorithms. Its approximate computations are up to 20× faster than the exact computations on the original floating-point numbers, and in our experiments highly correlated with the true values. Moreover, Bolt can achieve up to 15× compression while maintaining this level of accuracy. Because Bolt compression is extremely fast it is ideal for  <<nearly free seems a stretch>> rapidly-changing data.

% Bolt can easily be embedded in other algorithms, such as k-means clustering, to dramatically improve their performance.  It is our hope that Bolt will be used in many production systems to greatly reduce storage and computation costs for large, real-valued datasets.

% Bolt can easily be embedded in other algorithms, such as k-means clustering, to dramatically improve their performance.  It is our hope that Bolt will be used in many production systems to greatly reduce storage and computation costs for large, real-valued datasets.

% ================================================================
% References
% ================================================================

% \nocite{*}

% \IEEEtriggeratref{27}	% trigger column break to make cols even
\bibliographystyle{ACM-Reference-Format}
% \bibliographystyle{abbrev}
\bibliography{doc}

\end{document}
