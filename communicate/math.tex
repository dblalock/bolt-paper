
% \documentclass[sigconf]{acmart}  % starting in 2017
% \documentclass[acmlarge]{acmart}  % starting in 2017
% \documentclass[manuscript]{acmart}  % starting in 2017
% \documentclass[conference]{IEEEtran}
\documentclass[]{article}
%-------------------------------------------------------------- Includes

\usepackage{amsmath}          % basic math
\usepackage{amssymb} 			    % math symbols
% \usepackage{amsthm}           % theorems
\usepackage{textcomp}

\usepackage{float}            % make figures work
\usepackage{cite}             % citations would be nice
\usepackage{url}              % better urls; magically keeping doc from breaking due to urls in refs

% \usepackage{tabu}
\usepackage{array}            % multiline table cells; somehow
\usepackage{tabularx}         % better tables
% \usepackage[table]{xcolor}    % colored table cells  % incompatible with sigconf
\usepackage{colortbl}
\newcolumntype{Y}{>{\centering\arraybackslash}X}	% centered column type for tabularx

% stuff for piecewise functions
\usepackage{mathtools}          %loads amsmath as well
\DeclarePairedDelimiter\Floor\lfloor\rfloor
\DeclarePairedDelimiter\Ceil\lceil\rceil

\DeclareMathOperator*{\argmin}{arg\,min} % argmin
\DeclareMathOperator*{\argmax}{arg\,max} % argmax

\usepackage{pbox}   % for trick to force linebreaks in table cells

\usepackage{setspace}
% \usepackage{setspace}

\usepackage{booktabs} % acm recommended, and for pandas latex tables

%-------------------------------------------------------------- Algorithm setup

\usepackage{algorithm}
\usepackage[noend]{algpseudocode} % I think this removes trailing "end {if,for,while}"

% \algnewcommand{\LineComment}[1]{\State \(\triangleright\) #1} % left-aligned comments
% \algnewcommand{\SideComment}[1]{\(//\) #1}
\algnewcommand{\COMMENT}[2][.5\linewidth]{\leavevmode\hfill\makebox[#1][l]{//~#2}}
\algnewcommand{\LineComment}[1]{\State \(//\) #1}	% left-aligned comments
\algnewcommand\RETURN{\State \textbf{return} }


%-------------------------------------------------------------- Figures setup

% \usepackage[pdftex]{graphicx}
\usepackage{graphicx}
\usepackage[space]{grffile}   % allow spaces in file names
% declare the path(s) where your graphic files are
\graphicspath{{../figs/}}
% and their extensions so you won't have to specify these
\DeclareGraphicsExtensions{.pdf,.jpeg,.jpg,.png}


%\textfloatsep: space between last top float or first bottom float and the text (default = 20.0pt plus 2.0pt minus 4.0pt).
%\intextsep : space left on top and bottom of an in-text float (default = 12.0pt plus 2.0pt minus 2.0pt).
\setlength{\textfloatsep}{4pt}
\setlength{\intextsep}{4pt}

% \usepackage{caption}
% \usepackage[font={small,it}]{caption}
\usepackage[font={bf}]{caption}
% \setlength{\abovecaptionskip}{-2pt} % less space between captions and figures
\setlength{\abovecaptionskip}{4pt}	% less space between captions and figures
% \setlength{\belowcaptionskip}{-13pt}  % less space below captions
\setlength{\belowcaptionskip}{-10pt}	% less space below captions

\usepackage{paralist}
\setdefaultleftmargin{10pt}{10pt}{}{}{}{}

% \usepackage{changepage}
% \usepackage{tabulary}

\usepackage{outlines}
\usepackage{enumitem}
\newcommand{\ItemSpacing}{0mm}
\newcommand{\ParSpacing}{0mm}
\setenumerate[1]{itemsep={\ItemSpacing},parsep={\ParSpacing},label=\arabic*.}
% \setenumerate[2]{itemsep={\ItemSpacing},parsep={\ParSpacing},label=\arabic*.}
\setenumerate[2]{itemsep={\ItemSpacing},parsep={\ParSpacing}}

\usepackage[linewidth=1pt]{mdframed}
\mdfsetup{frametitlealignment=\center, skipabove=0, innertopmargin=1mm,
innerleftmargin=2mm, leftmargin=0mm, rightmargin=0mm}

%-------------------------------------------------------------- Miscellaneous setup

\newlength\myindent
\setlength\myindent{2em}
\newcommand\bindent{%
  \begingroup
  \setlength{\itemindent}{\myindent}
  \addtolength{\algorithmicindent}{\myindent}
}
\newcommand\eindent{\endgroup}

% remove unwanted space between paragraphs;
% it's set by the IEEE conference format, but no papers from this conference have it
% \parskip 0ex plus 0.2ex minus 0.1ex

% make vectors be bold instead of with arrows
\renewcommand{\vec}[1]{\mathbf{#1}}
% add 'mat' command to make matrices bold
\newcommand{\mat}[1]{\mathbf{#1}}

\DeclarePairedDelimiter\ceil{\lceil}{\rceil}
\DeclarePairedDelimiter\floor{\lfloor}{\rfloor}

\newtheorem{Definition}{Definition}[section]

% make *all* text 10pt
% \renewcommand{\footnotesize}{\normalsize}
% \renewcommand{\footnotesize}{\small}
% \renewcommand{\small}{\normalsize}


% \usepackage{cite}

\usepackage{mathtools}
\usepackage{amsthm}           % theorems
\newtheorem{theorem}{Theorem}[section]
\newtheorem{lemma}{Lemma}[section]
\newtheorem{definition}{Definition}[section]
\newtheorem{corollary}{Corrolary}[section]

\DeclareMathOperator{\MSE}{MSE}

% \newcommand\q\vec{q}
% \newcommand\q\vec{q}
% \newcommand\r\vec{r}
% \newcommand\x\vec{x}
% \newcommand\xhat\hat{\vec{x}}
% \newcommand\y\vec{y}
% \newcommand\yhat\hat{\vec{y}}
% \newcommand\z\vec{z}
% \newcommand\zhat\hat{\vec{z}}

% \setcopyright{rightsretained}

% %Conference
% \acmConference[KDD 2017]{ACM SIGKDD}{August 2017}{Halifax, Nova Scotia Canada}
% \acmYear{2017}
% \copyrightyear{2017}

\begin{document}

\title{Bolt: Theoretical Analysis}

\author{Davis W. Blalock, John V. Guttag}

% \author{Davis W. Blalock}
% \affiliation{%
%   \institution{Computer Science and Artificial \\ Intelligence Laboratory}
%   \institution{Massachusetts Institute of Technology}
% }
% \email{dblalock@mit.edu}

% \author{John V. Guttag}
% \affiliation{%
%   \institution{Computer Science and Artificial \\ Intelligence Laboratory}
%   \institution{Massachusetts Institute of Technology}
% }
% \email{guttag@mit.edu}

% \date{} % don't show date
\date{\Large February 18 2017}
\maketitle

Note: this document assumes you have already read the Bolt paper.

% ================================================================
\section{Quantization Error}
% ================================================================

\subsection{Definitions}

Let $Q$ be the distribution of query subvectors $\vec{q}_m$ for lookup table $m$, $X$ be the distribution of database subvectors $\vec{x}_m$ for this table, and $Y$ be the scalar-valued distribution of distances within that table. I.e.:
\begin{align}
    p(Y = y) \triangleq \int_{Q, X} p(\vec{q}_m, \vec{x}_m)I\{d_m(\vec{q}_m, \vec{x}_m) = y\}
\end{align}

Recall that we seek to learn a quantization function $\beta_m: \mathbb{R} \rightarrow \{0,\ldots,255\}$ of the form:
\begin{align}
    \beta_m(y) = \max(0, \min(255, \floor*{ay - b}))
\end{align}
\noindent that minimizes the loss:
\begin{align}
    E_Y[(\hat{y} - y)^2]
\end{align}
where $\hat{y} \triangleq (\beta_m(y) + b)/a$ is the \textit{reconstruction} of $y$.

In the paper, we propose setting $b = F^{-1}(\alpha)$ and $a = 255 / (F^{-1}(1 - \alpha) - b)$ for some suitable $\alpha$. $F^{-1}$ is the inverse CDF of $Y$, estimated empirically on a training set. The value of $\alpha$ is optimized using a simple grid search.

To analyze the performance of $\beta_m$ from a theoretical perspective, let us define the following:
\begin{itemize}
\itemindent3mm
\itemsep1.5mm
\item{Let $|\hat{y} - y|$ be the \textit{quantization error}. }
\item{Let $B = 256$ be the number of quantization bins.}
% \item{Let $y_{min} \triangleq \min_y(y)$ be the minimum value of $Y$ in the training dataset.}
% \item{Let $y_{max} \triangleq \max_y(y)$ be the maximum value of $Y$ in the training dataset.}
% \item{Let $b_{min} \triangleq \sup_y \{\beta_m(y) = 0\} = F^{-1}(\alpha)$ be the smallest value that can be quantized without clipping.}
% \item{Let $b_{max} \triangleq \inf_y \{\beta_m(y) = B\} = F^{-1}(1 - \alpha)$ be the largest value that can be quantized without clipping.}
\item{Let $b_{min} \triangleq F^{-1}(\alpha)$ be the smallest value that can be quantized without clipping.}
\item{Let $b_{max} \triangleq F^{-1}(1 - \alpha)$ be the largest value that can be quantized without clipping.}
\item{Let $\Delta \triangleq \frac{b_{max} - b_{min} }{ B }$ be the width of each quantization bin.}
\end{itemize}

Using these quantities, the quantization error for a given $y$ value can be decomposed into:
\begin{align} \label{eq:decomposition}
    |y - \hat{y}| =
    \begin{cases}
        b_{min} - y         & \text{if } y \le b_{min} \\
        \Delta c(y)         & \text{if } b_{min} < y \le b_{max} \\
        y - b_{max}         & \text{if } y > b_{max}
        % (y - b_{min})^2           & \text{if } y \le b_{min} \\
        % (y - \floor*{ay - b})^2   & \text{if } b_{min} < y \le b_{max} \\
        % (y - b_{max})^2           & \text{if } y > b_{max}
    \end{cases}
\end{align}
where $c(y) = (y - \hat{y}) / \Delta$ returns a value in $[0, 1)$ indicating where $\hat{y}$ lies within its quantization bin. These three cases represent $y$ being clipped at $b_{min}$, being rounded down to the nearest bin boundary, or being clipped at $b_{max}$, respectively.

It will also be helpful to define the following properties.

\begin{definition}
% A random variable $X$ is $(\lambda, l, h)$-exponential, if and only if:
A random variable $X$ is $(l, h)$-exponential if and only if:
\begin{align}
    % & \lambda > 0 \\
    l < \hspace{1mm} & E[X] < h \\
    % p(X < \gamma) < \hspace{1mm} & \lambda e^{-\lambda (E[X] - \gamma)}, \hspace{1mm} \gamma \le l \\
    % p(X > \gamma) < \hspace{1mm} & \lambda e^{-\lambda (\gamma - E[X])}, \hspace{1mm} \gamma \ge h
    p(X < \gamma) < \hspace{1mm} & \frac{1}{\sigma_X} e^{-(E[X] - \gamma) / \sigma_X}, \hspace{1mm} \gamma \le l \\
    p(X > \gamma) < \hspace{1mm} & \frac{1}{\sigma_X} e^{-(\gamma - E[X]) / \sigma_X}, \hspace{1mm} \gamma \ge h
\end{align}
where $\sigma_X$ is the standard deviation of $X$.
% and
% \begin{align}
%     p(X > h) < \lambda e^{-\lambda (h - E[X])}
% \end{align}
\end{definition}
In words, $X$ is $(l, h)$-exponential if its tails are bounded by exponential distributions. For appropriate $l$ and $h$, this definition includes distributions including the Laplace, Exponential, Gaussian, and all subgaussian distributions.

% We also defined a one-tailed correlary, useful for nonnegative distances.

% \begin{definition}
% % A random variable $X$ is $(\lambda, l, h)$-exponential, if and only if:
% A random variable $X$ is $(h)$-exponential if and only if:
% \begin{align}
%     p(X < 0) &= 0 \\
%     p(X > \gamma) < \hspace{1mm} & \frac{1}{\sigma_X} e^{-\gamma / \sigma_Y}, \hspace{1mm} \gamma \ge h
% \end{align}
% where $\sigma_Y$ is the standard deviation of $Y$.
% \end{definition}


% ------------------------------------------------
\subsection{Guarantees}
% ------------------------------------------------

% \noindent Using these definitions, we obtain the following guarantees.

% ------------------------ alpha = 0 proof

\begin{lemma} \label{thm:max_bin_err}
% Let $b = 255$ be the number of quantization levels and let $y_{min} \triangleq \min_y(y)$ and $y_{max} \triangleq \max_y(y)$. Then $(\beta_m(y) - y)^2$ is at most $(\frac{ y_{max} - y_{min} }{ b })^2$.
% Let $p(Y < y_{min}) = 0$ and $p(Y > y_{max}) = 0$. Then $E_Y[|\hat{y} - y|]$ is at most $\frac{ y_{max} - y_{min} }{ b }$.
Let $p(Y < b_{min}) = 0$ and $p(Y > b_{max}) = 0$. Then $|\hat{y} - y| < \Delta$.
\end{lemma}

% \begin{proof} Suppose that $\alpha$ is fixed at $0$. In this case, $\Delta = (y_{max} - y_{min}) / B$. In the worst case, all of the probability density of $Y$ is concentrated at the maximum possible value that maps to a given bin; formally:
\begin{proof}
The error $|\hat{y} - y| > \eps$ can be decomposed according to (\ref{eq:decomposition}). By assumption, the first and last terms in this decomposition, wherein $Y$ clips, have probability 0. This leaves only:
\begin{align}
    |y - \hat{y}| = \Delta c(y)
\end{align}
where $0 \le c(y) < 1$. For any value of $c(y)$, $|y - \hat{y}| < \Delta$. Intuitively, this means that if the distribution isn't clipped, the worst quantization error is the width of a quantization bin.

% In the worst case, all of the probability density of $Y$ is concentrated at the maximum possible value that maps to a given bin. I.e.:
% \begin{align}
%     p(y) > 0 \iff \hspace{1mm} (b_{min} + i \Delta - \epsilon) \le y < (b_{min} + i \Delta), \hspace{1mm} i \in \{0,\ldots,B-1 \}
% \end{align}
% % where $\epsilon \approx 0$ is a positive constant. Then the quantization error for values within a given bin is $\Delta$. Since $Y$ falls entirely within $[b_{min}, b_{max}]$, the expected loss across all values of $Y$ is also $\Delta^2$. Since the loss for the learned value of $\alpha$ can be no worse than that when $\alpha = 0$ (since selecting $\alpha = 0$ is an option), the loss for the learned $\alpha$ is at most $\Delta^2 = (\frac{ y_{max} - y_{min} }{ b })^2$.
% where $\epsilon \approx 0$ is a positive constant. Then the quantization error for values within a given bin is $\Delta$. Since $Y$ falls entirely within $[b_{min}, b_{max}]$, the maximum quantization error for any value of $y$ is also $\Delta$.
\end{proof}

% ------------------------ exponential tails PAC bound


% So the idea here is that we define "(l, h)-exponential", which means that below l and above h, distro is described by exponential decaying from mean u, and then we show that for distros with this property, we get a sum of two squared exponentials
%   -

\begin{theorem}[Two-tailed generalization bound] \label{thm:pac_quant}
Let $Y$ be $(b_{min}, b_{max})$-exponential. Then:
\begin{align} \label{eq:bound_two_tailed}
    p(|y - \hat{y}| > \eps) <
                            % \frac{1}{\sigma_Y} \left(
                            %     e^{-(b_{max} + \eps - E[Y]) / \sigma_Y}
                            %     + e^{-(E[Y] - b_{min} - \eps) / \sigma_Y}
                            % \right)
                            \frac{1}{\sigma_Y} \left(
                                e^{-(b_{max}- E[Y]) / \sigma_Y}
                                + e^{-(E[Y] - b_{min}) / \sigma_Y}
                            \right)e^{-\eps / \sigma_Y}
\end{align}
for all $\eps > \Delta$.
% The loss $E_Y[(\hat{y} - y)^2]$ is at most $\Delta^2$ when evaluated on the training set.
\end{theorem}

\begin{proof}
% If $Y$ lies entirely within $[b_{min}, b_{max}]$, then $p(|y - \hat{y}| < \Delta) = 1$ by Lemma~\ref{thm:max_bin_err}. Consequently, the component of $Y$ that lies in this interval can contribute quantization error of at most $\Delta$. The remaining error comes from the components of $Y$ lying outside this interval. Using $\Delta$ as an upper bound on , and
Using the decomposition in (\ref{eq:decomposition}), we have:
\begin{align} \label{eq:decomp_proba}
    % p(|y - \hat{y}| > \eps) &= p(|y - b_{min}| > \eps)p(y \le b_{min}) \\
    %                         &+ p(|y - b_{max}| > \eps)p(y > b_{max}) \\
    %                         &+ p(|y - \floor*{ay - b}| > \eps)p(b_{min} < y \le b_{max})
    p(|y - \hat{y}| > \eps)
                            &= p(c(y)\Delta > \eps)p(b_{min} < y \le b_{max}) \nonumber \\
                            &+ p(b_{min} - y > \eps) \\
                            &+ p(y - b_{max} > \eps) \nonumber
\end{align}
The first term corresponds to $y$ being truncated within a bin, and the latter two correspond to $y$ clipping. Since $0 \le c(y) < 1$ and $p(b_{min} < y \le b_{max}) \le 1$, the first term can be bounded as:
\begin{align}
    p(c(y)\Delta > \eps)p(b_{min} < y \le b_{max}) < I\{\eps < \Delta\} \label{eq:bound_bin}
\end{align}
where $I\{\cdot\}$ is the indicator function. The latter two terms can be bounded using the fact that $Y$ is $(b_{min}, b_{max})$-exponential:
\begin{align}
    p(b_{min} - y > \eps) &= p(y < b_{min} - \eps) < \frac{1}{\sigma_Y} e^{-(E[Y] - b_{min} - \eps) / \sigma_Y} \\
    p(y - b_{max} > \eps) &= p(y > b_{max} + \eps) < \frac{1}{\sigma_Y} e^{-(b_{max} + \eps - E[Y]) / \sigma_Y} \label{eq:bound_bmax}
% \end{align}
% Similarly, for the second term:
% \begin{align}
\end{align}
Combining (\ref{eq:bound_bin})-(\ref{eq:bound_bmax}), we have:
\begin{align}
    p(|y - \hat{y}| > \eps) < I\{\eps < \Delta\} +
                            \frac{1}{\sigma_Y} \left(
                                e^{-(b_{max} + \eps - E[Y]) / \sigma_Y}
                                + e^{-(E[Y] - b_{min} - \eps) / \sigma_Y}
                            \right)
\end{align}
When $\eps \ge \Delta$, the first term is zero and we obtain (\ref{eq:bound_two_tailed}).
% \begin{align}
%     p(|y - \hat{y}| > \eps) <
%                             \frac{1}{\sigma_Y} \left(
%                                 e^{(b_{max} + \eps - E[Y]) / \sigma_Y}
%                                 + e^{(E[Y] - b_{min} - \eps) / \sigma_Y}
%                             \right)
% \end{align}
\end{proof}

% letting $\Delta \triangleq \frac{y_{max} - y_{min} }{ b }$ be the width of each bin,

For ease of understanding, it is helpful to consider the case wherein $b_{min}$ and $b_{max}$ are symmetric about the mean. When this holds, the bound of Theorem~\ref{thm:pac_quant} simplifies to the more concise expression of Lemma~\ref{thm:pac_quant_z}. This shows that the error probability decays exponentially with the number of standard deviations $b_{min}$ and $b_{max}$ are from the mean, as well as the size of $\eps$ (normalized by the standard deviation).

\begin{lemma}[Symmetric generalization bound] \label{thm:pac_quant_z}
Let $z$ be any scalar such that $Y$ is $(E[y] - z \sigma_Y, E[y] + z \sigma_Y)$-exponential. Then:
\begin{align}
    % p(|y - \hat{y}| > \epsilon) < TODO
    p(|y - \hat{y}| > \eps) < \frac{1}{\sigma_Y} 2e^{-(z + \eps / \sigma_Y)}
\end{align}
where $\eps > \Delta = 2z\sigma_Y / B$.
% The loss $E_Y[(\hat{y} - y)^2]$ is at most $\Delta^2$ when evaluated on the training set.
\end{lemma}

\begin{proof}
This follows immediately from Theorem~\ref{thm:pac_quant} using $b_{min} = E[y] - z \sigma_Y$, $b_{max} = E[y] + z \sigma_Y$. % Note that the constraint on $\eps$ corresponds to the bin width using these values of $b_{min}$ and $b_{max}$. % These values correspond to $\alpha = 0$. Since the learned value of $\alpha$ can be no worse than this (since selecting $\alpha = 0$ is an option), the loss for the learned $\alpha$ is at most TODO.
\end{proof}

The bound of \ref{thm:pac_quant} is effective when $Y$ is roughly symmetric (as is the case when quantizing dot products, and sometimes the case when quantizing $L_p$ distances), but less so when $Y$ is heavily skewed (as is often the case when quantizing $L_p$ distances). In the presence of severe skewness, $E[Y]$ is close to either $b_{min}$ or $b_{max}$, and so one of the two exponentials in parentheses approaches $1$. Theorem~\ref{thm:pac_quant_one_tail} describes a tighter bound for the case of right skew and a hard lower limit of $0$, since this is often the distribution observed for $L_p$ distances. The corresponding bound for left skew and a hard upper limit is trivial so we omit it. Note that this theorem is useful only if $b_{min} \approx \Delta$, but this is commonly the case when the $L_p$ distances are highly skewed.

\begin{lemma}[One-tailed generalization bound] \label{thm:pac_quant_one_tail}
Let $Y$ be $(b_{min}, b_{max})$-exponential, with $p(Y < 0) = 0$. Then:
\begin{align} \label{eq:bound_one_tailed}
    p(|y - \hat{y}| > \eps) <
        \frac{1}{\sigma_Y} e^{-(b_{max} + \eps - E[Y]) / \sigma_Y}
\end{align}
for all $\eps > \max(\Delta, b_{min})$.
% The loss $E_Y[(\hat{y} - y)^2]$ is at most $\Delta^2$ when evaluated on the training set.
\end{lemma}

\begin{proof}
Using (\ref{eq:decomp_proba}) with the fact that $\eps > b_{min}$, we have:
\begin{align} \label{eq:decomp_proba}
    p(|y - \hat{y}| > \eps)
        % = p(c(y)\Delta > \eps)p(b_{min} < y \le b_{max}) + p(y - b_{max} > \eps)
                            &= p(c(y)\Delta > \eps)p(b_{min} < y \le b_{max}) \\
                            &+ p(y - b_{max} > \eps) \nonumber
\end{align}
Again applying the bounds from (\ref{eq:bound_bin}) and (\ref{eq:bound_bmax}), we obtain (\ref{eq:bound_one_tailed}).
% \begin{align} \label{eq:decomp_proba}
%     p(|y - \hat{y}| > \eps)
%                             &= p(c(y)\Delta > \eps)p(b_{min} < y \le b_{max}) \nonumber \\
%                             &+ p(y - b_{max} > \eps) \nonumber
% \end{align}
\end{proof}

% \begin{lemma} \label{thm:pac_quant_z}
% Let $\alpha$ be the scalar used to set $b_{min}$, $b_{max}$ and let $Y$ be $(b_{min}, b_{max})$-exponential. Then:
% \begin{align}
%     p(|y - \hat{y}| > \eps) <
%                             \frac{1}{\sigma_Y} \left(
%                                 e^{-(b_{max} + \eps - E[Y]) / \sigma_Y}
%                                 + e^{-(E[Y] - b_{min} - \eps) / \sigma_Y}
%                             \right)
% \end{align}
% for all $\eps > \Delta$.
% \end{lemma}

% ================================================================
\section{Dot Product Error}
% ================================================================
In this section, we bound the error in Bolt's approximate dot products. We also introduce a useful closed-form approximation that helps to explain the high performance of product quantization-based algorithms in general.

\subsection{Definitions and preliminaries}

\begin{definition}[Codebook]
A $(B, J)$-\textit{codebook} $C$ is an ordered collection of $2^B$ vectors $\vec{c} \in \mathbb{R}^J$. Each vector is referred to as a ``centroid'' or ``codeword.'' The notation $\vec{c}_i$ denotes the $i$th centroid in the codebook.
\end{definition}

\begin{definition}[Codelist]
A $(K, B, J)$-codelist $\mathcal{C}$ is an ordered collection of $k$ $(B, J / K)$-codebooks. Because zero-padding is trivial and does not affect any relevant measure of accuracy, we assume that $J$ is a multiple of $K$. The notation $\vec{c}_{ij}$ denotes the $i$th centroid in the $j$th codebook. A codelist can be thought of (and stored) as a rank-3 tensor whose columns are codebooks, treated as row-major 2D arrrays.
\end{definition}

\begin{definition}[Subvectors of a vector]
Let $\vec{x} \in \R^J$ be a vector, let $K > 0$ be an integer, and let $L = J / K$. Then $\x^{(1)},\ldots,\x^{(K)}$ are the subvectors of $\vec{x}$, where $\x^{(k)} \in \mathbb{R}^L \triangleq x_{(k-1)L + 1},\ldots,x_{L}$. As with codelists, $J$ is assumed to be a multiple of $K$.
\end{definition}

\begin{definition}[Encoding of a vector]
Let $\vec{x} \in \R^J$ be a vector with subvectors $\x^{(1)},\ldots,\x^{(K)}$ and let $\mathcal{C}$ be a $(K, B, J)$-codelist. Then the encoding of $\vec{x}$ is the sequence of integers $a_1,\ldots,a_K$, $0 < a_k \le 2^B$ where $a_k \triangleq \argmin_i ||\x^{(k)} - \vec{c}_{ik}||^2$.
\end{definition}

\begin{definition}[Reconstruction]
Let $a_1,\ldots,a_K$, $0 < a_k \le 2^B$ be the encoding of some vector $\vec{x}$, and let $\mathcal{C}$ be a $(K, B, J)$-codelist. Then the concatenation of the vectors $c_{a_{1}1},c_{a_{2}2},\ldots,c_{a_{K}K}$ is the reconstruction of $\vec{x}$, denoted $\hat{\vec{x}}$.
\end{definition}

\begin{definition}[Residuals]
Let $\hat{\vec{x}}$ be the reconstruction of $\vec{x}$. Then $\vec{r} \triangleq \vec{x} - \hat{\vec{x}}$ is the residual vector for $\x$.
\end{definition}

Apart from these definitions, it is also necessary to establish several geometric properties of random (encoded) vectors in high-dimensional spaces.

% \begin{lemma}[Area of a hypersphere]
% The area of a hypersphere in $\R^J$ with radius $r$, denoted $A_J(r)$ is given by:
% \begin{align}
%     A_J(r) = \frac{ 2\pi^{J/2} }{ \Gamma({\frac{J}{2}}) } r^{J-1}
% \end{align}
% \end{lemma}

% ------------------------ dot prod and ED biases

\begin{lemma}[Dot product bias (\cite{pairQ})]
Let $\hat{\vec{x}}$ be the reconstruction of $\vec{x}$ using codelist $\mathcal{C}$, and suppose that the centroids of all codebooks within $\mathcal{C}$ were learned using k-means. Then $E[\q^\top \x - \q^\top \xhat] = 0$.
\end{lemma}

\begin{lemma}[Euclidean distance bias (\cite{pq, pairQ})] \label{thm:l2_bias}
Let $a_1,\ldots,a_K$, $0 < a_k \le 2^B$ be the encoding of some vector $\vec{x}$ using codelist $\mathcal{C}$ and $\xhat$ be the reconstruction of $\x$. Further suppose that the centroids of all codebooks within $\mathcal{C}$ were learned using k-means. Then:
\begin{align}
    E[\norm{\q - \x}^2 - \norm{\q - \xhat}^2] = \sum_{k=1}^K \MSE(a_k, k)
\end{align}
where $\MSE(a_k, k)$ is the expected squared Euclidean distance between centroid $\vec{c}_{a{_k}k}$ and the subvectors assigned to it by k-means. I.e.,
\begin{align}
    % \MSE(a_k, k) \triangleq E_X[\norm{\vec{c}_{a{_k}k} - \x^{(k)} }^2 \hspace{1mm} | \hspace{1mm} \argmin_i \norm{\vec{c}_{ik} - \x^{(k)} }^2 = a_k ]
    \MSE(a_k, k) \triangleq E_X[\norm{\vec{c}_{a{_k}k} - \x^{(k)} }^2], \hspace{1mm} a_k = \argmin_i \norm{\vec{c}_{ik} - \x^{(k)} }^2
\end{align}
\end{lemma}

\begin{lemma}[Area of a hyperspherical cap (Li. 2011 \cite{hypersphericalCap})] \label{thm:cap_area}
Suppose that a hyperphere in $\R^J$ with radius $r$ is cut into two caps by a hyperplane, with the angle $\theta, 0 \le \theta \le \frac{\pi}{2}$ defining the radius of the smaller cap. Then the area of the smaller cap is given by
\begin{align}
    A_J(r) = \frac{1}{2} A^s_J(r) I_{\sin^2(\theta)} \left( \frac{J-1}{2}, \frac{1}{2} \right)
\end{align}
where $A^s_J(r)$ is the area of the hyperpsphere and $I_x(a, b)$ denotes the regularized incomplete beta function (i.e., the CDF of a $\Beta(a, b)$ distribution).
\end{lemma}

% ------------------------ angle between random vects, cdf

\begin{lemma}[Minimum angle between random vectors] \label{thm:rand_angle}
Let $\x, \y \in \R^J$ be vectors such that $\frac{\x^\top\y}{\norm{\x}\norm{\y}} \x$ is sampled uniformly from the surface of the unit hypersphere $S^{J-1}$, and let $\theta \triangleq \arccos(\frac{\x^\top\y}{\norm{\x} \cdot \norm{\y} })$ be the angle between $\x$ and $\y$. Then for $0 \le a \le \pie{2}$,
\begin{align} \label{eq:rand_angle}
    p(\abs{\theta} \ge a) = I_{\sin^2(a)} \left( \frac{J-1}{2}, \frac{1}{2} \right)
\end{align}
%\sim \Beta(\frac{J-1}{2}, \frac{1}{2})$, where $0 \le a \le 2\pi$ and $\cos(\theta) = \frac{\x^\top\y / \y}{\norm{\x} \cdot \norm{\y} }$.
\end{lemma}

\begin{proof} Since the angle between $\x$ and $\y$ is independent of their norms, assume without loss of generality that $\x$ and $\y$ have been scaled such that $\norm{x} = \norm{y} = 1$. %This implies that $\x^\top \y = \cos(\theta)$ is sampled uniformly from the unit hypersphere.
For a given $\x$, the set of $\y$ vectors such that $\cos(\theta) \ge a, \theta \le \pie{2}$ is exactly the set of vectors comprising a hyperspherical cap of $S^{J-1}$ with radius defined by $a$.
% The set of possible vectors $\y$ on the unit hypersphere such that $\cos(\theta) \ge a, \theta \le \pie{2}$ is exactly set of vectors comprising the hyperspherical cap with radius $a$.
Because the projection onto $\x$ of $\y$ has probability mass uniformly distributed across $S^{J-1}$, the probability that $\y$ lies within this cap is equal to the area of the cap divided by the area of the hypersphere. Using Lemma~\ref{thm:cap_area}, this ratio is given by:
\begin{align} \label{eq:beta_survival}
    \frac{1}{2} I_{\sin^2(a)} \left( \frac{J-1}{2}, \frac{1}{2} \right)
\end{align}
By symmetry, this is also $p(\theta < -a)$. Summing the probabilities of these two events yields (\ref{eq:rand_angle}).

% Noting that $I_x(a, b) = p(\beta < x), \beta \sim \Beta(a, b)$, we can differentiate to obtain
%
% NOTE: if we wanna throw in gaussian approximation, we need to use the pdf, not this cdf
\end{proof}

% ------------------------ angle between random vects, pdf

% \begin{lemma}[Angle between random vectors]
% Let $\x$, $\q$, and $\theta$ be defined as in Lemma~\ref{thm:rand_angle}. Then
% \begin{align}
%     \abs{\cos(\theta)} \sim \Beta(\frac{J-1}{2}, \frac{J-1}{2})
% \end{align}
% \end{lemma}

% \begin{proof}
% Using the identity $I_z(\alpha, \alpha) = I_{4z(1-z)}(\alpha, \frac{1}{2})$ \cite{DMLF_beta}, we can rewrite (\ref{eq:beta_survival}) as:
% \begin{align}
%     I_{\abs{\phi}}(\frac{J-1}{2}, \frac{J-1}{2})
% \end{align}
% where $\phi = \onehalf(1 - \cos(\theta))$. Observing that this defines the CDF of $\abs{ \frac{1}{2}(1 - \cos(\theta) }$ under a $\Beta(\frac{J-1}{2}, \frac{J-1}{2})$ distribution, we can differentiate to obtain:
% \begin{align}
%     \phi \sim \onehalf \Beta(\frac{J-1}{2}, \frac{J-1}{2})
%     % \sin^2(\theta) \sim \frac{1}{2} \Beta(\frac{J-1}{2}, \frac{1}{2})
%     % TODO use different beta bounds
% \end{align}
% and thus
% \begin{align}
%     \cos(\theta) \sim 1 - 2\Beta(\frac{J-1}{2}, \frac{J-1}{2})
% \end{align}

% \end{proof}

% ------------------------ gaussian approx

\begin{lemma}[Gaussian approximation to angle between random vectors]
Let $\x$, $\q$, and $\theta$ be defined as in Lemma~\ref{thm:rand_angle}. Then
\begin{align} \label{eq:gauss_approx}
    % p(cos(\theta)) \sim \Normal(0, 4\sqrt(J))
    p(cos(\theta) > a) \approx \onehalf + \onehalf \erf \left( -cos(\theta)\sqrt{\frac{J}{2}} \right)
\end{align}
\end{lemma}

\begin{proof}
Using the identity $I_z(\alpha, \alpha) = \onehalf I_{4z(1-z)}(\alpha, \frac{1}{2})$ \cite[Eq.~8.17.6]{DLMF}, we can rewrite (\ref{eq:beta_survival}) as:
\begin{align}
    I_{\phi}(\frac{J-1}{2}, \frac{J-1}{2})
\end{align}
where $\phi = \onehalf(1 - \cos(\theta))$. Recall that a $\Beta(\alpha, \beta)$ distribution can be approximated by a normal distribution with:
\begin{align} \label{eq:normal_approx}
    \mu &= \frac{\alpha}{\alpha + \beta} \\
    \sigma^2 &= \frac{\alpha \beta}{(\alpha + \beta)^2(1 + \alpha + \beta)}
    % I_{\abs{\phi}}(\frac{J-1}{2}, \frac{J-1}{2}) \approx \erf \left( -cos(\theta)\sqrt{\frac{J}{2}} \right)
\end{align}
Using $\alpha = \beta = \frac{J-1}{2}$, this yeilds
\begin{align}
    \mu &= \onehalf \label{eq:mu} \\
    \sigma^2 &= \frac{ \left( \frac{J-1}{2} \right)^2 }{
        4 \left( \frac{J-1}{2} \right)^2 \left( 1 + 2\frac{J-1}{2} \right)
    }
    = \frac{1}{4J}  \label{eq:sigma2}
    % I_{\abs{\phi}}(\frac{J-1}{2}, \frac{J-1}{2}) \approx \erf(\frac{a - \mu}{\sigma\sqrt(2)})
\end{align}
Further recall that the CDF of a normal distribution with a given mean $\mu$ and variance $\sigma^2$ is given by
\begin{align} \label{eq:erf_cdf}
    \Phi(a) = \onehalf + \onehalf \erf \left( \frac{a - \mu}{\sigma\sqrt{2}} \right)
\end{align}
Substituting (\ref{eq:mu}) and (\ref{eq:sigma2}) into (\ref{eq:erf_cdf}), we obtain
\begin{align}
    I_{\phi}(\frac{J-1}{2}, \frac{J-1}{2}) \approx \erf((\phi - \onehalf)\sqrt{2J})
\end{align}
Finally, substituting $\onehalf(1 - \cos(\theta))$ for $\phi$ yields (\ref{eq:gauss_approx}).
    % I_{\cos(\theta)}(\frac{J-1}{2}, \frac{J-1}{2}) \approx
\end{proof}

% \begin{corollary}[Gaussian PDF approximation] \label{thm:gauss_pdf}
\begin{lemma}[Gaussian PDF approximation] \label{thm:gauss_pdf}
Let $\x$, $\q$, and $\theta$ be defined as in Lemma~\ref{thm:rand_angle}. Then
\begin{align} \label{eq:normal_approx_pdf}
    cos(\theta) \sim \Normal(0, J^{-1})
\end{align}
\end{lemma}

\begin{proof} Writing (\ref{eq:gauss_approx}) in the form of (\ref{eq:erf_cdf}) gives
\begin{align}
    \mu &= 0 \\
    \sigma^2 &= \frac{1}{J}
\end{align}
Because (\ref{eq:gauss_approx}) is the CDF of a Gaussian random variable with this $\mu$ and $\sigma^2$, the PDF is given by $\Normal(0, J^{-1})$.
\end{proof}

% Observing that this defines the CDF of $\abs{ \frac{1}{2}(1 - \cos(\theta) }$ under a $\Beta(\frac{J-1}{2}, \frac{J-1}{2})$ distribution, we can differentiate to obtain:
% \begin{align}
%     \phi \sim \onehalf \Beta(\frac{J-1}{2}, \frac{J-1}{2})
% \end{align}
% and thus
% \begin{align}
%     \cos(\theta) \sim 1 - 2\Beta(\frac{J-1}{2}, \frac{J-1}{2})
% \end{align}

% ------------------------------------------------
\subsection{Guarantees}% and Analytic Approximation}
% ------------------------------------------------

We now prove several bounds on the errors caused by product quantization using an arbitrary number of subvectors. We begin with no distributional assumptions, and then prove increasingly tight bounds as more assumptions are added.

We begin with Lemma~\ref{thm:worst_dotprod}, which is not probabalistic.

% ------------------------ dot worst case

\begin{lemma}[Worst-case dot product error] \label{thm:worst_dotprod}
Let $\hat{\vec{x}}$ be the reconstruction of $\x$ and let $\q \in \R^J$ be a vector. Then $\abs{\q^\top \x - \q^\top \xhat} < \norm{\q} \cdot \norm{\r}$.
\end{lemma}

\begin{proof} \label{thm:dot_unif}
This follows immediately from application of the Cauchy-Schwarz inequality.
\begin{align}
    % \q^\top \x = \q^\top (\xhat + \r) = \q^\top \xhat + \q^\top \r \\
    \abs{\q^\top \x - \q^\top \xhat} = \abs{\q^\top \x - \q^\top (\x - \r)} = \abs{\q^\top \r} \le \norm{q} \cdot \norm{r}
\end{align}
\end{proof}

% ------------------------ dot unif

If we are willing to make the extremely pessimistic assumption that the cosine of the angle between $\q$ and $\r$ is uniformly distributed, a tighter bound (and indeed, an exact expression for the error probability) is possible (Theorem~\ref{thm:dot_unif}). This assumption is pessimistic because angles close to $0$, which yield smaller errors, are much more probable in high dimensions. % In fact, this assumption is even more pessimistic than assuming all $\theta$ are equiprobable, as it concentrates more mass on $\theta$ closer to $\pm 1$.%; the former corresponds to a $\theta \sim Beta(1, 1)$, while the latter corrsponds to $\theta \sim Beta(\frac{1}{2}, \frac{1}{2})$. Actually, no, first is cos(theta) ~ Beta(.5, .5), second is cos(theta) ~ Beta(1, 1)

\begin{theorem}[Pessimistic dot product error]
Let $\theta$ denote the angle between $\r$ and some vector $\q$, and suppose that $\cos(\theta) \sim Unif(-1, 1)$. Then
\begin{align}
    p(\abs{\q^\top \x - \q^\top \xhat} > \eps) = \max(0, 1 - \frac{\eps}{\norm{\q} \cdot \norm{\r}})
\end{align}
\end{theorem}

\begin{proof}
Simple algebra shows that
\begin{align} \label{eq:qTr}
    \q^\top \x - \q^\top \xhat = \q^\top(\xhat + \r) - \q^\top \xhat = \q^\top \r = \norm{\q} \cdot \norm{\r} \cos(\theta)
\end{align}
Since $\cos(\theta) \sim Unif(-1, 1)$, we have that $\abs{\cos(\theta)} \sim Unif(0, 1)$, and therefore
\begin{align}
    p(\abs{\q^\top \x - \q^\top \xhat} > \eps) &= p(\abs{\norm{\q} \cdot \norm{\r} \cos(\theta)} > \eps ) \\
    &= p \left( \abs{\cos(\theta)} > \frac{\eps}{\norm{\q} \cdot \norm{\r} } \right) \\
    &= \max \left( 0, 1 - \frac{\eps}{\norm{\q} \cdot \norm{\r}} \right)
\end{align}
\end{proof}

% ------------------------ dot indep subvects

The assumption that the cosine similarity of vectors is uniform can be replaced with the slightly more optimistic assumption that the errors in quantizing each subvector are independent, yielding Theorem~\ref{thm:dot_indep}.

\begin{theorem}[Dot product error with independent subspaces] \label{thm:dot_indep}
% Let $\x^{(1)},\ldots,\x^{(K)}$ be the subvectors of $\x$, let $\xhat^{(1)},\ldots,\xhat^{(K)}$ be the subvectors of $\xhat$, and let $\r^{(1)},\ldots,\r^{(K)}$ be the $K$ residual vectors, where $\r^{(k)} \triangleq \x^{(k)} - \xhat^{(k)}$. Then $p(\abs{\q^\top \x - \q^\top \xhat} \ge \eps) \le 2\exp(\frac{-2\eps^2}{\sum_{k=1}^K \norm{ \r^{(k)} }^2 })$.
Let $\x^{(1)},\ldots,\x^{(K)}$ be the subvectors of $\x$, let $\xhat^{(1)},\ldots,\xhat^{(K)}$ be the subvectors of $\xhat$, and let $\q^{(1)},\ldots,\q^{(K)}$ be the subvectors of an arbitrary vector $\q \in R^J$. Further let $\r^{(k)} \triangleq \x^{(k)} - \xhat^{(k)}$, and assume that the values of $\norm{ \r^{(k)} }$ are independent for all $k$. Then: % $\norm{\q^{(k)}} \cdot \norm{ \r^{(k)} }$ are independent for all $k$. Then:
\begin{align} \label{eq:dot_indep}
    p(\abs{\q^\top \x - \q^\top \xhat} \ge \eps) \le 2\exp \left( \frac{-\eps^2}{
        4 \sum_{k=1}^K (\norm{\q^{(k)}} \cdot \norm{ \r^{(k)} }))^2
    }\right)
        % \sum_{k=1}^K (\norm{\q^{(k)}} \cdot \norm{\x^{(k)} - \xhat^{(k)} }))^2
\end{align}
\end{theorem}

\begin{proof}
The quantity $\q^\top \x - \q^\top \xhat$ can be expressed as the sum
\begin{align}
    % \q^\top \x - \q^\top \xhat = \sum_{k=1}^K \q^{(k)\top} (\x^{(k)} - \xhat^{(k)})
    \sum_{k=1}^K \q^{(k)\top} (\x^{(k)} - \xhat^{(k)}) = \sum_{k=1}^K \q^{(k)\top} \r^{(k)}
\end{align}
Each element of this sum can be viewed as an independent random variable $v_k$. By Lemma~\ref{thm:worst_dotprod}, $-\norm{\q^{(k)}} \cdot \norm{\r^{(k)}} < v_k < \norm{\q^{(k)}} \cdot \norm{\r^{(k)}}$. The inequality (\ref{eq:dot_indep}) then follows from Hoeffding's inequality.
\end{proof}

This bound assumes the worst-case distribution of errors for each subvector. If we instead assume that the errors are random as defined in Lemma~\ref{thm:rand_angle}, it is possible to obtain not only a bound, but a closed-form expression for the probability of a given error.

% ------------------------ dot prod approx

\begin{theorem}[Dot product error approximation] \label{thm:dot_approx}
Let $\x^{(1)},\ldots,\x^{(K)}$, \\ $\xhat^{(1)},\ldots,\xhat^{(K)}$, $\q^{(1)},\ldots,\q^{(K)}$, $\r^{(1)},\ldots,\r^{(K)}$ be defined as in Theorem~\ref{thm:dot_indep}. Suppose that each $(\q^{(k)}, \r^{(k)})$ satisfy the conditions of Lemma~\ref{thm:rand_angle} and the values of $\q^{(k)\top} \r^{(k)}$ are independent across all $k$. % quantities $\q^{(k)\top} \r^{(k)}$ and $\norm{ \q^{(k)} } \cdot \norm{ \r^{(k)} }$ are independent across all $k$. Then
\begin{align} \label{eq:dot_approx}
    p(\q^\top \x - \q^\top \xhat) \approx \Normal(0, \sigma^2)
\end{align}
% where $\sigma^2 \triangleq \frac{1}{J} \sum_{k=1}^K \norm{\q^{(k)} } \cdot \norm{\r^{(k)} }$.
where $\sigma^2 \triangleq J^{-1} \sum_{k=1}^K \norm{\q^{(k)} } \cdot \norm{\r^{(k)} }$.
\end{theorem}

\begin{proof}
Applying Lemma~\ref{thm:gauss_pdf} to a given $(\q^{(k)}, \r^{(k)})$, we have the approximation:
\begin{align}
    cos(\theta_k) \sim \Normal(0, L^{-1})
\end{align}
where $\theta_k \triangleq \frac{ \q^{(k)^\top}\r^{(k)} }{ \norm{\q^{(k)}} \cdot \norm{\r^{(k)}} }$. Recalling from (\ref{eq:qTr}) that $\q^\top \x - \q^\top \xhat = \norm{\q} \cdot \norm{\r} \cos(\theta)$, this implies that
\begin{align}
    % \q^{(k)^\top} \x^{(k)} - \q^\top \xhat \sim \Normal (foo, bar)
    \q^{(k)^\top} \x^{(k)} - \q^\top \xhat \sim \Normal(0, \sigma_k^2)
\end{align}
where $\sigma_k^2 = \frac{ \norm{\q^{(k)}} \cdot \norm{\r^{(k)}} }{ L }$.
Because the errors from each subspace are independent, one can sum their variances and divide by $K$ to obtain (\ref{eq:dot_approx}). Observe that we have simplified $(KL)^{-1}$ to $J^{-1}$.
\end{proof}

This approximation is optimistic if the codebooks are trained from k-means using the Euclidean distance, since the residuals' directions are unlikely to be uniformly distributed on the unit hypersphere. However, if the centroids are trained under the Mahalanobis distance as in \cite{googleMips, pairQ}, then this approximation may be pessimistic. This is because the latter approach tends to concentrate $cos(\theta)$ around $0$ (by construction), which yields even smaller variances in each subspace.

It is also interesting to note that this error expression is independent of $K$. One might expect that having more independent estimators would be preferable, but the tendency of larger subvectors to concentrate $\cos(\theta)$ near $0$ exactly cancels the resulting variance reduction, making the error depend only on the vector dimensionality $J$ and residuals within each subspace. % Because the square of the error itself is likely to be proportional to $J$,

% ================================================================
\section{Euclidean Distance Error}
% ================================================================

The guarantees in this section closely parallel those of the previous section, so we state them without comment.

% ------------------------ l2 worst case

\begin{theorem}[Worst-case $L_2$ error] \label{thm:worst_l2}
Let $\xhat$ be the reconstruction of $\x$ and let $\q \in \R^J$ be a vector. Then $\abs{\norm{\q - \x} - \norm{\q - \xhat} } < \norm{\r}$.
\end{theorem}

\begin{proof}
This follows immediately from application of the triangle inequality. % Simple algebra shows that
\begin{align}
   \norm{\q - \x} - \norm{\r} \le \norm{\q - \xhat} = \norm{\q - \x + \r} \le \norm{\q - \x} + \norm{\r}
\end{align}
and therefore
\begin{align}
    \abs{\norm{\q - \x} - \norm{\q - \xhat} } \le \norm{\r}
\end{align}
\end{proof}

% ------------------------ l2 unif

\begin{theorem}[Pessimistic $L_2$ error]
Let $\theta$ denote the angle between $\r$ and some vector $\q$, and suppose that $\cos(\theta) \sim Unif(-1, 1)$. Then
\begin{align} \label{eq:l2_unif}
    % p(\abs{ \norm{\q - \x}^2 - \norm{\q - \xhat}^2 } > \eps) = \max(0, 1 - \frac{\eps}{ \norm{\r}^2 })
    p(\abs{ \norm{\q - \x}^2 - \norm{\q - \xhat}^2 } > \eps) = \max \left( 0, 1 - \frac{\norm{\r}^2 - \eps }{ 2\norm{\r}\norm{\q - \x} } \right)
\end{align}
\end{theorem}

\begin{proof}
Using the Law of Cosines, we have
\begin{align}
    \norm{\q - \x}^2 &= \norm{\q - \xhat}^2 + \norm{\x - \xhat}^2 - 2\norm{\q - \xhat}\norm{\x - \xhat}\cos(\theta) \nonumber \\
    & = \norm{\q - \xhat}^2 + \norm{\r}^2 - 2\norm{\q - \xhat}\norm{\r}\cos(\theta)
\end{align}
and therefore
\begin{align} \label{eq:l2_diff_cos}
    \norm{\q - \x}^2 - \norm{\q - \xhat}^2 = \norm{\r}^2 - 2\norm{\r}\norm{\q - \xhat}\cos(\theta)
\end{align}
This implies that
\begin{align}
    % p(\abs{ \norm{\q - \x}^2 - \norm{\q - \xhat}^2 } > \eps) &= p(\norm{\r}^2 > \eps + 2\norm{\r}\norm{\q - \x}\cos(\theta) ) \nolabel \\
    p(\norm{\q - \x}^2 - \norm{\q - \xhat}^2 > \eps) &= p(\norm{\r}^2 - 2\norm{\r}\norm{\q - \xhat}\cos(\theta) > \eps) \nonumber \\
    &= p(\frac{\norm{\r}^2 - \eps }{ 2\norm{\r}\norm{\q - \xhat} } > \cos(\theta)) \nonumber \\
    &= \onehalf \max \left( 0, 1 - \frac{\norm{\r}^2 - \eps }{ 2\norm{\r}\norm{\q - \xhat} } \right)
\end{align}
Equation (\ref{eq:l2_unif}) follows by symmetry.

\end{proof}

% ------------------------ l2 indep subvects

\begin{theorem}[$L_2$ error with independent subspaces] \label{thm:l2_indep}
Let $\x^{(1)},\ldots,\x^{(K)}$ be the subvectors of $\x$, let $\xhat^{(1)},\ldots,\xhat^{(K)}$ be the subvectors of $\xhat$, and let $\q^{(1)},\ldots,\q^{(K)}$ be the subvectors of an arbitrary vector $\q \in R^J$. Further let $\r^{(k)} \triangleq \x^{(k)} - \xhat^{(k)}$, and assume that the values of $\norm{\q^{(k)} - \x^{(k)} }^2 - \norm{\q^{(k)} - \xhat^{(k)} }^2$ are independent for all $k$. % $\norm{ \r^{(k)} }$ are independent for all $k$. Then:
\begin{align} \label{eq:l2_indep}
    p(\abs{ \norm{\q - \x}^2 - \norm{\q - \xhat}^2 } > \eps) \le
        2\exp \left( \frac{-\eps^2}{
            4 \sum_{k=1}^K \norm{\r^{(k)}}^4
        }\right)
\end{align}
\end{theorem}

\begin{proof}
The quantity $\norm{\q - \x}^2 - \norm{\q - \xhat}^2$ can be expressed as the sum
\begin{align}
    \sum_{k=1}^K \norm{\q^{(k)} - \x^{(k)} }^2 - \norm{\q^{(k)} - \xhat^{(k)} }^2
\end{align}
By assumption, each element of this sum can be viewed as an independent random variable $v_k$. By Lemma~\ref{thm:worst_l2}, $-\norm{\r^{(k)}}^2 \le v_k \le \norm{\r^{(k)}}^2$. Assuming that one adds in the bias correction described in Lemma~\ref{thm:l2_bias}, one can apply Hoeffding's inequality to obtain (\ref{eq:dot_indep}).
\end{proof}

% ------------------------ l2 approx

\begin{theorem}[$L_2$ error approximation] \label{thm:l2_approx}
Let $\x^{(1)},\ldots,\x^{(K)}$, $\xhat^{(1)},\ldots,\xhat^{(K)}$, \\ $\q^{(1)},\ldots,\q^{(K)}$, $\r^{(1)},\ldots,\r^{(K)}$ be defined as in Theorem~\ref{thm:l2_indep}. Suppose that each pair $(\q^{(k)} - \xhat^{(k)}, \r^{(k)})$ satisfy the conditions of Lemma~\ref{thm:rand_angle} and the values of $(\q^{(k)} - \xhat^{(k)})^\top \r^{(k)}$ are independent across all $k$.
\begin{align} \label{eq:l2_approx}
    p(\norm{\q - \x}^2 - \norm{\q - \xhat}^2) \approx \Normal(\norm{r}^2, \sigma^2)
\end{align}
% where $\sigma^2 \triangleq \frac{1}{J} \sum_{k=1}^K \norm{\q^{(k)} } \cdot \norm{\r^{(k)} }$.
where $\sigma^2 \triangleq 4 \norm{\r}^2\norm{\q - \x}^2 J^{-1}$.
\end{theorem}

\begin{proof}
Applying Lemma~\ref{thm:gauss_pdf} to a given $(\q^{(k)}, \r^{(k)})$, we have the approximation:
\begin{align} \label{eq:subspace_theta}
    cos(\theta_k) \sim \Normal(0, L^{-1})
\end{align}
where $\theta_k \triangleq \frac{ \q^{(k)^\top}\r^{(k)} }{ \norm{\q^{(k)}} \cdot \norm{\r^{(k)}} }$. Further recall from (\ref{eq:l2_diff_cos}) that
\begin{align} \label{eq:asdf}
    \norm{\q^{(k)} - \x^{(k)}}^2 - \norm{\q^{(k)} - \xhat^{(k)}}^2 = \norm{\r^{(k)}}^2 - 2\norm{\r^{(k)}}\norm{\q^{(k)} - \xhat^{(k)}}\cos(\theta_k)
\end{align}
Combining (\ref{eq:asdf}) and (\ref{eq:subspace_theta}) yields
\begin{align}
    \norm{\q^{(k)} - \x^{(k)}}^2 - \norm{\q^{(k)} - \xhat^{(k)}}^2 \sim \mathcal{N}(\norm{\r^{(k)}}^2, \sigma_k^2)
\end{align}
where $\sigma_k^2 \triangleq 4 \norm{\r^{(k)}}^2\norm{\q^{(k)} - \xhat^{(k)}}^2 L^{-1}$. Because the errors from each subspace are independent, one can sum their variances and divide by $K$ to obtain (\ref{eq:l2_approx}).

\end{proof}

% ================================================================ Bib

% \bibliographystyle{ACM-Reference-Format}
% \bibliographystyle{abbrv}
\bibliographystyle{acm}
\bibliography{math}

\end{document}
