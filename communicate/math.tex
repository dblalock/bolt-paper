
% \documentclass[sigconf]{acmart}  % starting in 2017
% \documentclass[acmlarge]{acmart}  % starting in 2017
% \documentclass[manuscript]{acmart}  % starting in 2017
% \documentclass[conference]{IEEEtran}
\documentclass[]{article}
%-------------------------------------------------------------- Includes

\usepackage{bbm}  % who knows?

\usepackage{amsmath}          % basic math
\usepackage{amssymb} 			    % math symbols
% \usepackage{amsthm}           % theorems
\usepackage{textcomp}

% \usepackage{float}            % make figures work
% \usepackage{cite}             % citations would be nice
\usepackage{url}              % better urls; magically keeping doc from breaking due to urls in refs

% \usepackage{tabu}
\usepackage{array}            % multiline table cells; somehow
\usepackage{tabularx}         % better tables
% \usepackage[table]{xcolor}    % colored table cells  % incompatible with sigconf
\usepackage{colortbl}
\newcolumntype{Y}{>{\centering\arraybackslash}X}	% centered column type for tabularx

% stuff for piecewise functions
\usepackage{mathtools}          %loads amsmath as well
\DeclarePairedDelimiter\Floor\lfloor\rfloor
\DeclarePairedDelimiter\Ceil\lceil\rceil

% \DeclareMathOperator*{\argmin}{arg\,min} % argmin
% \DeclareMathOperator*{\argmax}{arg\,max} % argmax
\DeclareMathOperator*{\argmin}{argmin} % argmin
\DeclareMathOperator*{\argmax}{argmax} % argmax

\usepackage{pbox}   % for trick to force linebreaks in table cells

\usepackage{setspace}

\usepackage{bm}  % bold math

\usepackage[dvipsnames]{xcolor}

% \usepackage{hyperref}

% \color{ForestGreen}

% \usepackage{setspace}

% \usepackage{booktabs} % acm recommended, and for pandas latex tables

%-------------------------------------------------------------- Algorithm setup

% commented out for sysml/icml

% \usepackage{algorithm}
% \usepackage{algorithmicx}
% \usepackage[noend]{algpseudocode} % I think this removes trailing "end {if,for,while}"

% % % \algnewcommand{\LineComment}[1]{\State \(\triangleright\) #1} % left-aligned comments
% % % \algnewcommand{\SideComment}[1]{\(//\) #1}
% \algnewcommand{\COMMENT}[2][.5\linewidth]{\leavevmode\hfill\makebox[#1][l]{//~#2}}
% \algnewcommand{\LineComment}[1]{\State \(//\) #1}	% left-aligned comments
% \algnewcommand\RETURN{\State \textbf{return} }


% sane comments when forced to use plain algorithmic instead of algorithmicx
% \renewcommand{\algorithmiccomment}[1]{\bgroup\hfill//~#1\egroup}
    % {\color{Gray}
\renewcommand{\algorithmiccomment}[1]{
    {\color{OliveGreen}
        \bgroup\hfill//~#1\egroup
    }
}
\newcommand{\LINECOMMENT}[1]{
    {\color{OliveGreen}
        //~#1
    }
}


%-------------------------------------------------------------- Figures setup

% \usepackage[pdftex]{graphicx}
\usepackage{graphicx}
\usepackage[space]{grffile}   % allow spaces in file names
% declare the path(s) where your graphic files are
\graphicspath{{../figs/}}
% and their extensions so you won't have to specify these
\DeclareGraphicsExtensions{.pdf,.jpeg,.jpg,.png}


%\textfloatsep: space between last top float or first bottom float and the text (default = 20.0pt plus 2.0pt minus 4.0pt).
%\intextsep : space left on top and bottom of an in-text float (default = 12.0pt plus 2.0pt minus 2.0pt).
\setlength{\textfloatsep}{4pt}
\setlength{\intextsep}{4pt}

% \usepackage{caption}
% \usepackage[font={small,it}]{caption}
\usepackage[font={bf}]{caption}
\setlength{\abovecaptionskip}{1pt} % less space between captions and figures
% \setlength{\abovecaptionskip}{-2pt} % less space between captions and figures
% \setlength{\abovecaptionskip}{-5pt}	% less space between captions and figures
% \setlength{\belowcaptionskip}{-13pt}  % less space below captions
% \setlength{\belowcaptionskip}{-10pt}  % less space below captions
\setlength{\belowcaptionskip}{-4pt}	% less space below captions

\usepackage{paralist}
\setdefaultleftmargin{10pt}{10pt}{}{}{}{}

% \usepackage{changepage}
% \usepackage{tabulary}

\usepackage{outlines}
\usepackage{enumitem}
\newcommand{\ItemSpacing}{0mm}
\newcommand{\ParSpacing}{0mm}
\setenumerate[1]{itemsep={\ItemSpacing},parsep={\ParSpacing},label=\arabic*.}
% \setenumerate[2]{itemsep={\ItemSpacing},parsep={\ParSpacing},label=\arabic*.}
\setenumerate[2]{itemsep={\ItemSpacing},parsep={\ParSpacing}}

% \usepackage[linewidth=1pt]{mdframed}
% \mdfsetup{frametitlealignment=\center, skipabove=0, innertopmargin=1mm,
% innerleftmargin=2mm, leftmargin=0mm, rightmargin=0mm}

%-------------------------------------------------------------- Miscellaneous setup

% \usepackage{enumitem}
% \setlist{nolistsep}
% \setlist{first=itemsep1em}
% \setlist[1]{labelindent=\parindent}

% \newlength\myindent
% \setlength\myindent{2em}
% \newcommand\bindent{%
%   \begingroup
%   \setlength{\itemindent}{\myindent}
%   \addtolength{\algorithmicindent}{\myindent}
% }
% \newcommand\eindent{\endgroup}

% remove unwanted space between paragraphs;
% it's set by the IEEE conference format, but no papers from this conference have it
% \parskip 0ex plus 0.2ex minus 0.1ex

% make vectors be bold instead of with arrows
% \renewcommand{\vec}[1]{\mathbf{#1}}
\renewcommand{\vec}[1]{\bm{#1}}
% add 'mat' command to make matrices bold
% \newcommand{\mat}[1]{\mathbf{#1}}
\newcommand{\mat}[1]{\bm{#1}}


\newcommand{\cs}[0]{\text{, }}

% \DeclarePairedDelimiter\ceil{\lceil}{\rceil}
% \DeclarePairedDelimiter\floor{\lfloor}{\rfloor}
\providecommand{\ceil}[1]{\left \lceil #1 \right \rceil }
\providecommand{\floor}[1]{\left \lfloor #1 \right \rfloor }

% \newtheorem{Definition}{Definition}[section]
% \newtheorem{Theorem}{Theorem}[section]
% \newtheorem{P}{Definition}[section]

% ------------------------ convenience commands
\newcommand\eps\varepsilon
\DeclareMathOperator{\infimum}{inf}
\renewcommand\inf\infty
\DeclareMathOperator{\erfc}{erfc}

\renewcommand{\c}{\vec{c}}
\newcommand{\q}{\vec{q}}
\renewcommand{\r}{\vec{r}}
\renewcommand{\v}{\vec{v}}
\newcommand{\vhat}{\hat{\vec{v}}}
\newcommand{\x}{\vec{x}}
\newcommand{\xhat}{\hat{\vec{x}}}
\newcommand{\y}{\vec{y}}
\newcommand{\yhat}{\hat{\vec{y}}}
\newcommand{\z}{\vec{z}}
\newcommand{\zhat}{\hat{\vec{z}}}

\newcommand{\lam}{\lambda}
\newcommand{\sig}{\sigma}
\newcommand{\Sig}{\Sigma}

\newcommand{\E}{\mathop{{}\mathbb{E}}}

\newcommand{\A}{\mat{A}}
\newcommand{\Ahat}{\mat{\hat{A}}}
\renewcommand{\a}{\vec{a}}
\newcommand{\ahat}{\vec{\hat{a}}}
\newcommand{\B}{\mat{B}}
\newcommand{\Bhat}{\mat{\hat{B}}}
\renewcommand{\b}{\vec{b}}
\newcommand{\bhat}{\vec{\hat{b}}}
\newcommand{\V}{\mat{V}}

\renewcommand{\H}{\mathcal{H}}
\newcommand{\R}{\mathbb{R}}
% \newcommand{\V}{\mathcal{V}}
\newcommand{\Xs}{\mathcal{X}}
% \newcommand{\Z}{\mathcal{Z}}
% \renewcommand{\S}{\mathcal{S}}
% \newcommand{\Nb}{\mathcal{N}_b}

\newcommand{\onehalf}{\frac{1}{2}}

\newcommand{\Xcal}{\mathcal{X}}
\newcommand{\Ycal}{\mathcal{Y}}

\newcommand{\pie}[1]{\frac{\pi}{#1}}
% \newcommand{\pitwo}{\frac{\pi}{2}}
% \newcommand{\pifour}{\frac{\pi}{4}}

% \newcommand{\norm}[1]{\left\lVert #1 \right\rVert}
\DeclarePairedDelimiter\abs{\lvert}{\rvert}%
\DeclarePairedDelimiter\norm{\lVert}{\rVert}%

\DeclareMathOperator{\Beta}{Beta}
\DeclareMathOperator{\Normal}{\mathcal{N}}
\DeclareMathOperator{\erf}{erf}
\DeclareMathOperator{\Var}{Var}

% \newcommand{\iid}{\stackrel{i.i.d.}{\sim}}
\newcommand{\iid}{\stackrel{iid}{\sim}}

% make *all* text 10pt
% \renewcommand{\footnotesize}{\normalsize}
% \renewcommand{\footnotesize}{\small}
% \renewcommand{\small}{\normalsize}

% ------------------------------------------------ stuff that might break things


% \usepackage{mathtools}
\usepackage{amsthm}           % theorems
\newtheorem{theorem}{Theorem}[section]
\newtheorem{lemma}{Lemma}[section]
\newtheorem{definition}{Definition}[section]
\newtheorem{corollary}{Corrolary}[section]


\usepackage{mathtools}
\usepackage{amsthm}           % theorems
\newtheorem{theorem}{Theorem}[section]
\newtheorem{lemma}{Lemma}[section]
\newtheorem{definition}{Definition}[section]

% \setcopyright{rightsretained}

% %Conference
% \acmConference[KDD 2017]{ACM SIGKDD}{August 2017}{Halifax, Nova Scotia Canada}
% \acmYear{2017}
% \copyrightyear{2017}

\begin{document}

\title{Bolt: Theoretical Analysis}

\author{Davis W. Blalock, John V. Guttag}

% \author{Davis W. Blalock}
% \affiliation{%
%   \institution{Computer Science and Artificial \\ Intelligence Laboratory}
%   \institution{Massachusetts Institute of Technology}
% }
% \email{dblalock@mit.edu}

% \author{John V. Guttag}
% \affiliation{%
%   \institution{Computer Science and Artificial \\ Intelligence Laboratory}
%   \institution{Massachusetts Institute of Technology}
% }
% \email{guttag@mit.edu}

% \date{} % don't show date
\date{\Large February 18 2017}
\maketitle

% ================================================================
\section{Quantization Error}
% ================================================================

\subsection{Definitions}

Let $Q$ be the distribution of query subvectors $\vec{q}_m$ for lookup table $m$, $X$ be the distribution of database subvectors $\vec{x}_m$ for this table, and $Y$ be the scalar-valued distribution of distances within that table. I.e.:
\begin{align}
    p(Y = y) \triangleq \int_{Q, X} p(\vec{q}_m, \vec{x}_m)I\{d_m(\vec{q}_m, \vec{x}_m) = y\}
\end{align}

Recall that we seek to learn a quantization function $\beta_m: \mathbb{R} \rightarrow \{0,\ldots,255\}$ of the form:
\begin{align}
    \beta_m(y) = \max(0, \min(255, \floor*{ay - b}))
\end{align}
\noindent that minimizes the loss:
\begin{align}
    E_Y[(\hat{y} - y)^2]
\end{align}
where $\hat{y} \triangleq a(\beta_m(y) + b)$ is the \textit{reconstruction} of $y$.

In the paper, we propose setting $b = F^{-1}(\alpha)$ and $a = 255 / (F^{-1}(1 - \alpha) - b)$ for some suitable $\alpha$. $F^{-1}$ is the inverse CDF of $Y$, estimated empirically on a training set. The value of $\alpha$ is optimized using a simple grid search.

To analyze the performance of $\beta_m$ from a theoretical perspective, let us define the following:
\begin{itemize}
\itemindent3mm
\itemsep1.5mm
\item{Let $|\hat{y} - y|$ be the \textit{quantization error}. }
\item{Let $B = 255$ be the number of quantization bins.}
% \item{Let $y_{min} \triangleq \min_y(y)$ be the minimum value of $Y$ in the training dataset.}
% \item{Let $y_{max} \triangleq \max_y(y)$ be the maximum value of $Y$ in the training dataset.}
\item{Let $b_{min} \triangleq \sup_y \{\beta_m(y) = 0\} = F^{-1}(\alpha)$ be the smallest value that can be quantized without clipping.}
\item{Let $b_{max} \triangleq \inf_y \{\beta_m(y) = B\} = F^{-1}(1 - \alpha)$ be the largest value that can be quantized without clipping.}
\item{Let $\Delta \triangleq \frac{b_{max} - b_{min} }{ B }$ be the width of each quantization bin.}
\end{itemize}

Using these quantities, the quantization error for a given $y$ value can be decomposed into:
\begin{align} \label{eq:decomposition}
    |y - \hat{y}| =
    \begin{cases}
        b_{min} - y         & \text{if } y \le b_{min} \\
        \Delta c(y)         & \text{if } b_{min} < y \le b_{max} \\
        y - b_{max}         & \text{if } y > b_{max}
        % (y - b_{min})^2           & \text{if } y \le b_{min} \\
        % (y - \floor*{ay - b})^2   & \text{if } b_{min} < y \le b_{max} \\
        % (y - b_{max})^2           & \text{if } y > b_{max}
    \end{cases}
\end{align}
where $c(y) = (y - \hat{y}) / \Delta$ returns a value in $[0, 1)$ indicating where $\hat{y}$ lies within its quantization bin. These three cases represent $y$ being clipped at $b_{min}$, being rounded down to the nearest bin boundary, or being clipped at $b_{max}$, respectively.

It will also be helpful to define the following properties.

\begin{definition}
% A random variable $X$ is $(\lambda, l, h)$-exponential, if and only if:
A random variable $X$ is $(l, h)$-exponential if and only if:
\begin{align}
    % & \lambda > 0 \\
    l < \hspace{1mm} & E[X] < h \\
    % p(X < \gamma) < \hspace{1mm} & \lambda e^{-\lambda (E[X] - \gamma)}, \hspace{1mm} \gamma \le l \\
    % p(X > \gamma) < \hspace{1mm} & \lambda e^{-\lambda (\gamma - E[X])}, \hspace{1mm} \gamma \ge h
    p(X < \gamma) < \hspace{1mm} & \frac{1}{\sigma_X} e^{-(E[X] - \gamma) / \sigma_X}, \hspace{1mm} \gamma \le l \\
    p(X > \gamma) < \hspace{1mm} & \frac{1}{\sigma_X} e^{-(\gamma - E[X]) / \sigma_X}, \hspace{1mm} \gamma \ge h
\end{align}
where $\sigma_Y$ is the standard deviation of $Y$.
% and
% \begin{align}
%     p(X > h) < \lambda e^{-\lambda (h - E[X])}
% \end{align}
\end{definition}
In words, $X$ is $(\lambda, l, h)$-exponential if its tails are bounded by exponential distributions. For appropriate $l$ and $h$, this definition includes distributions including the Laplace, Exponential, Gaussian, and all subgaussian distributions.

We also defined a one-tailed correlary, useful for nonnegative distances.

\begin{definition}
% A random variable $X$ is $(\lambda, l, h)$-exponential, if and only if:
A random variable $X$ is $(h)$-exponential if and only if:
\begin{align}
    p(X < 0) &= 0 \\
    p(X > \gamma) < \hspace{1mm} & \frac{1}{\sigma_X} e^{-\gamma / \sigma_Y}, \hspace{1mm} \gamma \ge h
\end{align}
where $\sigma_Y$ is the standard deviation of $Y$.
\end{definition}


% ------------------------------------------------
\subsection{Guarantees}
% ------------------------------------------------

% \noindent Using these definitions, we obtain the following guarantees.

% ------------------------ alpha = 0 proof

\begin{lemma} \label{thm:max_bin_err}
% Let $b = 255$ be the number of quantization levels and let $y_{min} \triangleq \min_y(y)$ and $y_{max} \triangleq \max_y(y)$. Then $(\beta_m(y) - y)^2$ is at most $(\frac{ y_{max} - y_{min} }{ b })^2$.
% Let $p(Y < y_{min}) = 0$ and $p(Y > y_{max}) = 0$. Then $E_Y[|\hat{y} - y|]$ is at most $\frac{ y_{max} - y_{min} }{ b }$.
Let $p(Y < b_{min}) = 0$ and $p(Y > b_{max}) = 0$. Then $|\hat{y} - y| < \Delta$.
\end{lemma}

% \begin{proof} Suppose that $\alpha$ is fixed at $0$. In this case, $\Delta = (y_{max} - y_{min}) / B$. In the worst case, all of the probability density of $Y$ is concentrated at the maximum possible value that maps to a given bin; formally:
\begin{proof}
The error $|\hat{y} - y| > \eps$ can be decomposed according to (\ref{eq:decomposition}). By assumption, the first and last terms in this decomposition, wherein $Y$ clips, have probability 0. This leaves only:
\begin{align}
    |y - \hat{y}| = \Delta c(y)
\end{align}
where $0 \le c(y) < 1$. For any value of $c(y)$, $|y - \hat{y}| < \Delta$. Intuitively, this means that if the distribution isn't clipped, the worst quantization error is the width of a quantization bin.

% In the worst case, all of the probability density of $Y$ is concentrated at the maximum possible value that maps to a given bin. I.e.:
% \begin{align}
%     p(y) > 0 \iff \hspace{1mm} (b_{min} + i \Delta - \epsilon) \le y < (b_{min} + i \Delta), \hspace{1mm} i \in \{0,\ldots,B-1 \}
% \end{align}
% % where $\epsilon \approx 0$ is a positive constant. Then the quantization error for values within a given bin is $\Delta$. Since $Y$ falls entirely within $[b_{min}, b_{max}]$, the expected loss across all values of $Y$ is also $\Delta^2$. Since the loss for the learned value of $\alpha$ can be no worse than that when $\alpha = 0$ (since selecting $\alpha = 0$ is an option), the loss for the learned $\alpha$ is at most $\Delta^2 = (\frac{ y_{max} - y_{min} }{ b })^2$.
% where $\epsilon \approx 0$ is a positive constant. Then the quantization error for values within a given bin is $\Delta$. Since $Y$ falls entirely within $[b_{min}, b_{max}]$, the maximum quantization error for any value of $y$ is also $\Delta$.
\end{proof}

% ------------------------ exponential tails PAC bound


% So the idea here is that we define "(l, h)-exponential", which means that below l and above h, distro is described by exponential decaying from mean u, and then we show that for distros with this property, we get a sum of two squared exponentials
%   -

\begin{theorem}[Two-tailed generalization bound] \label{thm:pac_quant}
Let $Y$ be $(b_{min}, b_{max})$-exponential. Then:
\begin{align} \label{eq:bound_two_tailed}
    p(|y - \hat{y}| > \eps) <
                            % \frac{1}{\sigma_Y} \left(
                            %     e^{-(b_{max} + \eps - E[Y]) / \sigma_Y}
                            %     + e^{-(E[Y] - b_{min} - \eps) / \sigma_Y}
                            % \right)
                            \frac{1}{\sigma_Y} \left(
                                e^{-(b_{max}- E[Y]) / \sigma_Y}
                                + e^{-(E[Y] - b_{min}) / \sigma_Y}
                            \right)e^{-\eps / \sigma_Y}
\end{align}
for all $\eps > \Delta$.
% The loss $E_Y[(\hat{y} - y)^2]$ is at most $\Delta^2$ when evaluated on the training set.
\end{theorem}

\begin{proof}
% If $Y$ lies entirely within $[b_{min}, b_{max}]$, then $p(|y - \hat{y}| < \Delta) = 1$ by Lemma~\ref{thm:max_bin_err}. Consequently, the component of $Y$ that lies in this interval can contribute quantization error of at most $\Delta$. The remaining error comes from the components of $Y$ lying outside this interval. Using $\Delta$ as an upper bound on , and
Using the decomposition in (\ref{eq:decomposition}), we have:
\begin{align} \label{eq:decomp_proba}
    % p(|y - \hat{y}| > \eps) &= p(|y - b_{min}| > \eps)p(y \le b_{min}) \\
    %                         &+ p(|y - b_{max}| > \eps)p(y > b_{max}) \\
    %                         &+ p(|y - \floor*{ay - b}| > \eps)p(b_{min} < y \le b_{max})
    p(|y - \hat{y}| > \eps)
                            &= p(c(y)\Delta > \eps)p(b_{min} < y \le b_{max}) \nonumber \\
                            &+ p(b_{min} - y > \eps) \\
                            &+ p(y - b_{max} > \eps) \nonumber
\end{align}
The first term corresponds to $y$ being truncated within a bin, and the latter two correspond to $y$ clipping. Since $0 \le c(y) < 1$ and $p(b_{min} < y \le b_{max}) \le 1$, the first term can be bounded as:
\begin{align}
    p(c(y)\Delta > \eps)p(b_{min} < y \le b_{max}) < I\{\eps < \Delta\} \label{eq:bound_bin}
\end{align}
where $I\{\cdot\}$ is the indicator function. The latter two terms can be bounded using the fact that $Y$ is $(b_{min}, b_{max})$-exponential:
\begin{align}
    p(b_{min} - y > \eps) &= p(y < b_{min} - \eps) < \frac{1}{\sigma_Y} e^{-(E[Y] - b_{min} - \eps) / \sigma_Y} \\
    p(y - b_{max} > \eps) &= p(y > b_{max} + \eps) < \frac{1}{\sigma_Y} e^{-(b_{max} + \eps - E[Y]) / \sigma_Y} \label{eq:bound_bmax}
% \end{align}
% Similarly, for the second term:
% \begin{align}
\end{align}
Combining (\ref{eq:bound_bin})-(\ref{eq:bound_bmax}), we have:
\begin{align}
    p(|y - \hat{y}| > \eps) < I\{\eps < \Delta\} +
                            \frac{1}{\sigma_Y} \left(
                                e^{-(b_{max} + \eps - E[Y]) / \sigma_Y}
                                + e^{-(E[Y] - b_{min} - \eps) / \sigma_Y}
                            \right)
\end{align}
When $\eps \ge \Delta$, the first term is zero and we obtain (\ref{eq:bound_two_tailed}).
% \begin{align}
%     p(|y - \hat{y}| > \eps) <
%                             \frac{1}{\sigma_Y} \left(
%                                 e^{(b_{max} + \eps - E[Y]) / \sigma_Y}
%                                 + e^{(E[Y] - b_{min} - \eps) / \sigma_Y}
%                             \right)
% \end{align}
\end{proof}

% letting $\Delta \triangleq \frac{y_{max} - y_{min} }{ b }$ be the width of each bin,

For ease of understanding, it is helpful to consider the case wherein $b_{min}$ and $b_{max}$ are symmetric about the mean. When this holds, the bound of Theorem~\ref{thm:pac_quant} simplifies to the more concise expression of Lemma~\ref{thm:pac_quant_z}. This shows that the error probability decays exponentially with the number of standard deviations $b_{min}$ and $b_{max}$ are from the mean, as well as the size of $\eps$ (normalized by the standard deviation).

\begin{lemma}[Symmetric generalization bound] \label{thm:pac_quant_z}
Let $z$ be any scalar such that $Y$ is $(E[y] - z \sigma_Y, E[y] + z \sigma_Y)$-exponential. Then:
\begin{align}
    % p(|y - \hat{y}| > \epsilon) < TODO
    p(|y - \hat{y}| > \eps) < \frac{1}{\sigma_Y} 2e^{-(z + \eps / \sigma_Y)}
\end{align}
where $\eps > \Delta = 2z\sigma_Y / B$.
% The loss $E_Y[(\hat{y} - y)^2]$ is at most $\Delta^2$ when evaluated on the training set.
\end{lemma}

\begin{proof}
This follows immediately from Theorem~\ref{thm:pac_quant} using $b_{min} = E[y] - z \sigma_Y$, $b_{max} = E[y] + z \sigma_Y$. % Note that the constraint on $\eps$ corresponds to the bin width using these values of $b_{min}$ and $b_{max}$. % These values correspond to $\alpha = 0$. Since the learned value of $\alpha$ can be no worse than this (since selecting $\alpha = 0$ is an option), the loss for the learned $\alpha$ is at most TODO.
\end{proof}

The bound of \ref{thm:pac_quant} is effective when $Y$ is roughly symmetric (as is the case when quantizing dot products, and sometimes the case when quantizing $L_p$ distances), but less so when $Y$ is heavily skewed (as is often the case when quantizing $L_p$ distances). In the presence of severe skewness, $E[Y]$ is close to either $b_{min}$ or $b_{max}$, and so one of the two exponentials in parentheses approaches $1$. Theorem~\ref{thm:pac_quant_one_tail} describes a tighter bound for the case of right skew and a hard lower limit of $0$, since this is often distribution observed for $L_p$ distances. The corresponding bound for left skew and a hard upper limit is trivial so we omit it. Note that this theorem is useful only if $b_{min} \approx \Delta$, but this is commonly the case when the $L_p$ distances are highly skewed.

\begin{lemma}[One-tailed generalization bound] \label{thm:pac_quant_one_tail}
Let $Y$ be $(b_{min}, b_{max})$-exponential, with $p(Y < 0) = 0$. Then:
\begin{align} \label{eq:bound_one_tailed}
    p(|y - \hat{y}| > \eps) <
        \frac{1}{\sigma_Y} e^{-(b_{max} + \eps - E[Y]) / \sigma_Y}
\end{align}
for all $\eps > \max(\Delta, b_{min})$.
% The loss $E_Y[(\hat{y} - y)^2]$ is at most $\Delta^2$ when evaluated on the training set.
\end{lemma}

\begin{proof}
Using (\ref{eq:decomp_proba}) with the fact that $\eps > b_{min}$, we have:
\begin{align} \label{eq:decomp_proba}
    p(|y - \hat{y}| > \eps)
        % = p(c(y)\Delta > \eps)p(b_{min} < y \le b_{max}) + p(y - b_{max} > \eps)
                            &= p(c(y)\Delta > \eps)p(b_{min} < y \le b_{max}) \\
                            &+ p(y - b_{max} > \eps) \nonumber
\end{align}
Again applying the bounds from (\ref{eq:bound_bin}) and (\ref{eq:bound_bmax}), we obtain (\ref{eq:bound_one_tailed}).
% \begin{align} \label{eq:decomp_proba}
%     p(|y - \hat{y}| > \eps)
%                             &= p(c(y)\Delta > \eps)p(b_{min} < y \le b_{max}) \nonumber \\
%                             &+ p(y - b_{max} > \eps) \nonumber
% \end{align}
\end{proof}

% \begin{lemma} \label{thm:pac_quant_z}
% Let $\alpha$ be the scalar used to set $b_{min}$, $b_{max}$ and let $Y$ be $(b_{min}, b_{max})$-exponential. Then:
% \begin{align}
%     p(|y - \hat{y}| > \eps) <
%                             \frac{1}{\sigma_Y} \left(
%                                 e^{-(b_{max} + \eps - E[Y]) / \sigma_Y}
%                                 + e^{-(E[Y] - b_{min} - \eps) / \sigma_Y}
%                             \right)
% \end{align}
% for all $\eps > \Delta$.
% \end{lemma}

\section{Dot Product Error}
In this section, we bound the error in Bolt's approximate dot products. We also introduce a useful closed-form approximation that helps to explain the high performance of product quantization-based algorithms.



\section{Euclidean Distance Error}


\end{document}
