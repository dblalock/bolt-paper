%!TEX output_directory = aux

\documentclass{article}  % sysml

\usepackage{booktabs}
\usepackage{hyperref}
\newcommand{\theHalgorithm}{\arabic{algorithm}}
\usepackage{icml2020}

% \newcommand{\oursp}{\textsc{HashMul}\text{ }}
% \newcommand{\ours}{\textsc{HashMul}}
% \newcommand{\oursp}{\textsc{MADDNESS}\text{ }}
% \newcommand{\ours}{\textsc{MADDNESS}}
\newcommand{\oursp}{\textsc{Maddness}\text{ }}
\newcommand{\ours}{\textsc{Maddness}}
\newcommand{\oursHash}{\textsc{MaddnessHash}}

% \usepackage{flafter}  %

% \usepackage[sorting=ynt]{biblatex}
% \usepackage[sorting=ynt]{natbib}
% \usepackage[sort]{natbib}
% \DeclareSortingScheme{noneyear}{
%  \sort{\citeorder}
%  \sort{\field{year}}
% }

% \usepackage[sorting=ynt]{natbib}

%-------------------------------------------------------------- Includes

\usepackage{bbm}  % who knows?

\usepackage{amsmath}          % basic math
\usepackage{amssymb} 			    % math symbols
% \usepackage{amsthm}           % theorems
\usepackage{textcomp}

% \usepackage{float}            % make figures work
% \usepackage{cite}             % citations would be nice
\usepackage{url}              % better urls; magically keeping doc from breaking due to urls in refs

% \usepackage{tabu}
\usepackage{array}            % multiline table cells; somehow
\usepackage{tabularx}         % better tables
% \usepackage[table]{xcolor}    % colored table cells  % incompatible with sigconf
\usepackage{colortbl}
\newcolumntype{Y}{>{\centering\arraybackslash}X}	% centered column type for tabularx

% stuff for piecewise functions
\usepackage{mathtools}          %loads amsmath as well
\DeclarePairedDelimiter\Floor\lfloor\rfloor
\DeclarePairedDelimiter\Ceil\lceil\rceil

% \DeclareMathOperator*{\argmin}{arg\,min} % argmin
% \DeclareMathOperator*{\argmax}{arg\,max} % argmax
\DeclareMathOperator*{\argmin}{argmin} % argmin
\DeclareMathOperator*{\argmax}{argmax} % argmax

\usepackage{pbox}   % for trick to force linebreaks in table cells

\usepackage{setspace}

\usepackage{bm}  % bold math

\usepackage[dvipsnames]{xcolor}

% \usepackage{hyperref}

% \color{ForestGreen}

% \usepackage{setspace}

% \usepackage{booktabs} % acm recommended, and for pandas latex tables

%-------------------------------------------------------------- Algorithm setup

% commented out for sysml/icml

% \usepackage{algorithm}
% \usepackage{algorithmicx}
% \usepackage[noend]{algpseudocode} % I think this removes trailing "end {if,for,while}"

% % % \algnewcommand{\LineComment}[1]{\State \(\triangleright\) #1} % left-aligned comments
% % % \algnewcommand{\SideComment}[1]{\(//\) #1}
% \algnewcommand{\COMMENT}[2][.5\linewidth]{\leavevmode\hfill\makebox[#1][l]{//~#2}}
% \algnewcommand{\LineComment}[1]{\State \(//\) #1}	% left-aligned comments
% \algnewcommand\RETURN{\State \textbf{return} }


% sane comments when forced to use plain algorithmic instead of algorithmicx
% \renewcommand{\algorithmiccomment}[1]{\bgroup\hfill//~#1\egroup}
    % {\color{Gray}
\renewcommand{\algorithmiccomment}[1]{
    {\color{OliveGreen}
        \bgroup\hfill//~#1\egroup
    }
}
\newcommand{\LINECOMMENT}[1]{
    {\color{OliveGreen}
        //~#1
    }
}


%-------------------------------------------------------------- Figures setup

% \usepackage[pdftex]{graphicx}
\usepackage{graphicx}
\usepackage[space]{grffile}   % allow spaces in file names
% declare the path(s) where your graphic files are
\graphicspath{{../figs/}}
% and their extensions so you won't have to specify these
\DeclareGraphicsExtensions{.pdf,.jpeg,.jpg,.png}


%\textfloatsep: space between last top float or first bottom float and the text (default = 20.0pt plus 2.0pt minus 4.0pt).
%\intextsep : space left on top and bottom of an in-text float (default = 12.0pt plus 2.0pt minus 2.0pt).
\setlength{\textfloatsep}{4pt}
\setlength{\intextsep}{4pt}

% \usepackage{caption}
% \usepackage[font={small,it}]{caption}
\usepackage[font={bf}]{caption}
\setlength{\abovecaptionskip}{1pt} % less space between captions and figures
% \setlength{\abovecaptionskip}{-2pt} % less space between captions and figures
% \setlength{\abovecaptionskip}{-5pt}	% less space between captions and figures
% \setlength{\belowcaptionskip}{-13pt}  % less space below captions
% \setlength{\belowcaptionskip}{-10pt}  % less space below captions
\setlength{\belowcaptionskip}{-4pt}	% less space below captions

\usepackage{paralist}
\setdefaultleftmargin{10pt}{10pt}{}{}{}{}

% \usepackage{changepage}
% \usepackage{tabulary}

\usepackage{outlines}
\usepackage{enumitem}
\newcommand{\ItemSpacing}{0mm}
\newcommand{\ParSpacing}{0mm}
\setenumerate[1]{itemsep={\ItemSpacing},parsep={\ParSpacing},label=\arabic*.}
% \setenumerate[2]{itemsep={\ItemSpacing},parsep={\ParSpacing},label=\arabic*.}
\setenumerate[2]{itemsep={\ItemSpacing},parsep={\ParSpacing}}

% \usepackage[linewidth=1pt]{mdframed}
% \mdfsetup{frametitlealignment=\center, skipabove=0, innertopmargin=1mm,
% innerleftmargin=2mm, leftmargin=0mm, rightmargin=0mm}

%-------------------------------------------------------------- Miscellaneous setup

% \usepackage{enumitem}
% \setlist{nolistsep}
% \setlist{first=itemsep1em}
% \setlist[1]{labelindent=\parindent}

% \newlength\myindent
% \setlength\myindent{2em}
% \newcommand\bindent{%
%   \begingroup
%   \setlength{\itemindent}{\myindent}
%   \addtolength{\algorithmicindent}{\myindent}
% }
% \newcommand\eindent{\endgroup}

% remove unwanted space between paragraphs;
% it's set by the IEEE conference format, but no papers from this conference have it
% \parskip 0ex plus 0.2ex minus 0.1ex

% make vectors be bold instead of with arrows
% \renewcommand{\vec}[1]{\mathbf{#1}}
\renewcommand{\vec}[1]{\bm{#1}}
% add 'mat' command to make matrices bold
% \newcommand{\mat}[1]{\mathbf{#1}}
\newcommand{\mat}[1]{\bm{#1}}


\newcommand{\cs}[0]{\text{, }}

% \DeclarePairedDelimiter\ceil{\lceil}{\rceil}
% \DeclarePairedDelimiter\floor{\lfloor}{\rfloor}
\providecommand{\ceil}[1]{\left \lceil #1 \right \rceil }
\providecommand{\floor}[1]{\left \lfloor #1 \right \rfloor }

% \newtheorem{Definition}{Definition}[section]
% \newtheorem{Theorem}{Theorem}[section]
% \newtheorem{P}{Definition}[section]

% ------------------------ convenience commands
\newcommand\eps\varepsilon
\DeclareMathOperator{\infimum}{inf}
\renewcommand\inf\infty
\DeclareMathOperator{\erfc}{erfc}

\renewcommand{\c}{\vec{c}}
\newcommand{\q}{\vec{q}}
\renewcommand{\r}{\vec{r}}
\renewcommand{\v}{\vec{v}}
\newcommand{\vhat}{\hat{\vec{v}}}
\newcommand{\x}{\vec{x}}
\newcommand{\xhat}{\hat{\vec{x}}}
\newcommand{\y}{\vec{y}}
\newcommand{\yhat}{\hat{\vec{y}}}
\newcommand{\z}{\vec{z}}
\newcommand{\zhat}{\hat{\vec{z}}}

\newcommand{\lam}{\lambda}
\newcommand{\sig}{\sigma}
\newcommand{\Sig}{\Sigma}

\newcommand{\E}{\mathop{{}\mathbb{E}}}

\newcommand{\A}{\mat{A}}
\newcommand{\Ahat}{\mat{\hat{A}}}
\renewcommand{\a}{\vec{a}}
\newcommand{\ahat}{\vec{\hat{a}}}
\newcommand{\B}{\mat{B}}
\newcommand{\Bhat}{\mat{\hat{B}}}
\renewcommand{\b}{\vec{b}}
\newcommand{\bhat}{\vec{\hat{b}}}
\newcommand{\V}{\mat{V}}

\renewcommand{\H}{\mathcal{H}}
\newcommand{\R}{\mathbb{R}}
% \newcommand{\V}{\mathcal{V}}
\newcommand{\Xs}{\mathcal{X}}
% \newcommand{\Z}{\mathcal{Z}}
% \renewcommand{\S}{\mathcal{S}}
% \newcommand{\Nb}{\mathcal{N}_b}

\newcommand{\onehalf}{\frac{1}{2}}

\newcommand{\Xcal}{\mathcal{X}}
\newcommand{\Ycal}{\mathcal{Y}}

\newcommand{\pie}[1]{\frac{\pi}{#1}}
% \newcommand{\pitwo}{\frac{\pi}{2}}
% \newcommand{\pifour}{\frac{\pi}{4}}

% \newcommand{\norm}[1]{\left\lVert #1 \right\rVert}
\DeclarePairedDelimiter\abs{\lvert}{\rvert}%
\DeclarePairedDelimiter\norm{\lVert}{\rVert}%

\DeclareMathOperator{\Beta}{Beta}
\DeclareMathOperator{\Normal}{\mathcal{N}}
\DeclareMathOperator{\erf}{erf}
\DeclareMathOperator{\Var}{Var}

% \newcommand{\iid}{\stackrel{i.i.d.}{\sim}}
\newcommand{\iid}{\stackrel{iid}{\sim}}

% make *all* text 10pt
% \renewcommand{\footnotesize}{\normalsize}
% \renewcommand{\footnotesize}{\small}
% \renewcommand{\small}{\normalsize}

% ------------------------------------------------ stuff that might break things


% \usepackage{mathtools}
\usepackage{amsthm}           % theorems
\newtheorem{theorem}{Theorem}[section]
\newtheorem{lemma}{Lemma}[section]
\newtheorem{definition}{Definition}[section]
\newtheorem{corollary}{Corrolary}[section]


% optional custom title for header
\icmltitlerunning{Multiplying Matrices Without Multiplying}

\begin{document}

\twocolumn[
% ================================================================
\icmltitle{Multiplying Matrices Without Multiplying}
% ================================================================

\begin{icmlauthorlist}
\icmlauthor{Batman}{WayneEnterprises}
\end{icmlauthorlist}

\icmlaffiliation{WayneEnterprises}{Wayne Enterprises, Gotham, USA}
\icmlcorrespondingauthor{Batman}{batman@batman.batman}

\icmlkeywords{Vector Quantization, Approximate Algorithms, Matrix Multiplication}

\vskip 0.3in
] % end of icml 2 columnn

\printAffiliationsAndNotice{}

% ------------------------------------------------
\begin{abstract}
% ------------------------------------------------

Multiplying matrices is among the most fundamental and most computationally demanding operations in machine learning and scientific computing. Consequently, the task of efficiently approximating matrix products has received significant attention.

We introduce an approximate matrix multiplication algorithm that greatly outperforms existing methods. Experiments using hundreds of matrices from diverse domains show that it often runs $10\times$ faster than current methods at a given level of error, as well as $100\times$ faster than exact matrix multiplication. In the common case that one matrix is known ahead of time, our method also has the interesting property that it requires zero multiply-adds. These results suggest that a mixture of hashing, averaging, and byte shuffling—--the core operations of our method—--could be a more promising building block than the sparsified, factorized, and/or scalar quantized matrix products that have been the subject of hundreds of papers and enormous hardware investment in recent years.

\end{abstract}

% ================================================================
\section{Introduction} \label{sec:intro}
% ================================================================


% Get cost equation in here; ops are CRUD, and existing work focuses on extremely read-heavy workloads

% TODO how to frame this given that we actually have both 10x query speed, (almost) 10x encoding speed, and can do matmuls 5x faster than BLAS?

As datasets grow larger, so too do the costs of mining them. These costs include not only the space to store the data, but also the compute time to operate on it. These time costs can be decomposed into:
\begin{align}
    \texttt{Cost}_{\text{time}} = \texttt{Cost}_{\text{read}} + \texttt{Cost}_{\text{write}}
\end{align}
where $\texttt{Cost}_{\texttt{read}}$ is the time cost of reading data, and $\texttt{Cost}_{\texttt{write}}$ is the time cost of creating, updating, or deleting data.

Vector quantization methods presently enable significant reductions in $\texttt{Cost}_{\texttt{storage}}$ and $\texttt{Cost}_{\texttt{read}}$ on datasets of real-valued vectors. By replacing each vector with a learned approximation, these methods both decrease space usage and enable fast approximate scalar reductions, such as dot products and Euclidean distances. With as little as 8B per vector, these techniques can preserve distances and dot products with extremely high accuracy [][][].



 % but add overhead to Cost_{\texttt{write}}.


% When the dataset is fixed, vector quantization methods enable significant reductions in both of these costs. By replacing each vector with a learned approximation, these methods both reduce the space needed to store the vectors and enable fast approximate scalar reductions, such as dot products and Euclidean distances.

However, computing the approximation for a given vector can be time-consuming, adding greatly to $\texttt{Cost}_{\texttt{write}}$. The state-of-the-art method of [LSQ], for example, requires up to \textit{4ms} to encode a single $128$-dimensional vector. Other techniques are faster, but as we show experimentally, virtually all suffer from encoding times that yield significant overhead when data is being added or changed rapidly.

We describe a vector quantization method, Bolt, that greatly reduces both the time to encode vectors ($\texttt{Cost}_{\texttt{write}}$) and the time to compute scalar reductions over them ($\texttt{Cost}_{\texttt{read}}$). This makes quantization worthwhile even for fast-changing or fast-arriving data. Our key idea is to use much smaller quantization codebooks than similar techniques, which both facilitates finding the optimal encoding and allows scans over codes to be done in a vectorized manner.

Our contributions consist of:
\begin{enumerate}
\item A vector quantization algorithm that encodes vectors significantly faster than existing algorithms for a given level of representational fidelity.
\item A fast means of computing approximate similarities and distances using quantized vectors. Possible similarities and distances include dot products, cosine similarities, and distances in $L_p$ spaces, such as the Euclidean distance.
\item Theoretical and empirical evidence that the sufficient statistics used my dozens of recent vector quantization algorithms can be approximated with little or no loss of accuracy.
\item Theoretical analysis of both our approach and popular related approaches.
\end{enumerate}


% Many machine learning algorithms are built around some combination of matrix multiplications, dot products, and distance computations in vector spaces. Such algorithms include deep neural networks, k-means and hierarchical clustering, and linear regression, among many others.

% Lots of people have tried speeding up deep neural nets, mostly by pruning them post-hoc or adopting elementwise quantization / sparsity


% for ease of exposition, we will refer to computing ``similarities'' as shorthand for computing either dot products or euclidean distances throughout the remainder of this work, because typing ``dot products or distances'' is really verbose


% Contributions:

% A fast means of preprocessing vectors so as to minimize quantization loss. This method is compatible with most techniques in the rapidly-evolving area of vector quantization.

% A hardware-friendly means of computing approximate distances using quantized vectors.

% Theoretical analysis of our technique's effectiveness.

% Empirical demonstration that approximate distances suffice for many real-world applications.


% somewhere sneak in the fact that Bolt is an acronym for Based On Lookup Tables


\subsection{Problem Statement}

Formally, the problem our algorithm addresses is the following.

Let $\vec{q} \in \mathbb{R}^J$ be a \textit{query} vector and let $\mathcal{X} = \{\vec{x}_1,\ldots,\vec{x}_N\}, \vec{x}_i \in \mathbb{R}^J$ be a collection of \textit{database} vectors. Further let $d: \mathbb{R}^J \times \mathbb{R}^J \rightarrow \mathbb{R}$ be a distance or similarity function that can be written as:
\begin{align} \label{eq:distFuncForm}
        d(\vec{q}, \vec{x}) = f(\sum_{j=1}^J \delta(q_j, x_j))
\end{align}

% TODO this technically doesn't include how additive methods do ED;
%   wait, nvm, we're fine cuz they use the d_hat equation, not this one

where $f: \mathbb{R} \rightarrow \mathbb{R}$, $\delta: \mathbb{R}^J \times \mathbb{R}^J \rightarrow \mathbb{R}$. This includes both distances in $L_p$ spaces and dot products as special cases. In the former case, $\delta(q_j, x_j) = |q_j - x_j|^p$ and $f(r) = r^{(1/p)}$; in the latter case, $\delta(q_j, x_j) = q_j x_j$ and $f(r) = r$. For brevity, we will henceforth refer to $d$ as a distance function and its output as a distance, though our remarks apply to all functions of the above form unless noted otherwise.

Our task is to construct three functions $g: \mathbb{R}^J \rightarrow \mathcal{G}$, $h: \mathbb{R}^J \rightarrow \mathcal{H}$, and $\hat{d}: \mathcal{G} \times \mathcal{H} \rightarrow \mathbb{R}$ such that for a given approximation loss $\mathcal{L}$,
\begin{align}
    \mathcal{L} = E_{\vec{q},\vec{x}}[(d(\vec{q}, \vec{x}) - \hat{d}(g(\vec{q}), h(\vec{x})))^2]
    % d(\vec{q}, \vec{x}) \approx \hat{d}(g(\vec{q}), h(\vec{x}))
\end{align}
the computation time $T$,
\begin{align}
    T = T_g + T_h + T_d
\end{align}
is minimized, where $T_g$ is the time to encode received queries $\vec{q}$ using $g$\footnote{We cast the creation of query-specific lookup tables as encoding $\vec{q}$ rather than creating a new $\hat{d}$ (the typical interpretation in recent literature).}, $T_h$ the time to encode $\mathcal{X}$ using $h$, and $T_d$ the time to compute the approximate distances between the encoded queries and encoded database vectors. The relative contributions of each of these terms depends on how frequently $\mathcal{X}$ changes, so many of our experiments characterize each separately. % Because related work typically optimizes $T_g$ and $T_d$ at the expense of $T_h$, our differentiating assumption is that $T_h$ cannot be neglected. % This assumption is invalid when $\mathcal{X}$

% \footnote{We describe the creation of lookup tables for each query as encoding of the query, not creation of a query-specific $\hat{d}$ (the typical interpretation in recent literature).}

% . This is not to be confused with the \textit{symmetric} encoding used in some binary embedding methods.

There is a tradeoff between the value of the loss $\mathcal{L}$ and the time $T$, so multiple operating points are possible. In the extreme cases, $\mathcal{L}$ can be fixed at 0 by setting $g$ and $h$ to identity functions and setting $\hat{d} = d$. Similarly, $T$ can be set to $0$ by ignoring the data and estimating $d$ as a constant. The primary contribution of this work is therefore the introduction of $g$, $h$ and $\hat{d}$ functions that are significantly faster than those of existing work for a wide range of operating points.

% the loss:
% \begin{align}
%     \mathcal{L} = E_{\vec{q},\vec{x}}[(d(\vec{q}, \vec{x}) - \hat{d}(g(\vec{q}), h(\vec{x})))^2]
%     % \mathcal{L}[g, h, \hat{d}] = E_{q,x}[(d(q, x) - \hat{d}(g(q), h(x)))^2]
% \end{align}
% is minimized. Moreover, each of these functions should be as fast to compute as possible.

% Intuitively, the function $g$ encodes the query, $h$ encodes the database vectors, and $\hat{d}$ computes an approximate similarity or distance based on the encodings. In general, it is possible to drive the loss arbitrarily low by increasing the time and space usage of these functions. Indeed, the loss can be fixed at 0 by setting $g$ and $h$ to identity functions and setting $\hat{d} = d$. Consequently, our metric of interest is \textbf{computation time for a given loss}. The primary contribution of this work is the introduction of $g$, $h$ and $\hat{d}$ functions that are significantly faster than those of existing work for a given loss $\mathcal{L} > 0$.

% \begin{center}
% % \begin{tabu} to \linewidth [0]{ X[l] | X[c] | X[c] | X[c] | X[c]}
% \begin{tabu} to \textwidth [0]{ X[l] | X[c] | X[c] | X[c] | X[c]}
% % \caption{Performance of Vector Quantization Algorithms}
% \hline
%     & Data Encoding Speed & Query Encoding Speed & Search Speed & Compression \\
% \hline
%     Bolt (proposed) & Very High & Very High & Very High & Medium \\
%     % Binary Embedding & High & High & High & Very Low \\
%     PQ & High & High & High & Low \\
%     OPQ & High & High & High & Medium \\
%     RVQ & High & Medium & High & Medium \\
%     GRVQ, LSQ, CQ, AQ, OTQ & Low, Very Low & Medium & High & High \\
%     Raw Floats & N/A & N/A & Low & None \\
% \hline
% \end{tabu}
% \end{center}




\subsection{Assumptions}

Like other work [][][][], we assume that there is an initial offline phase during which the functions $g$ and $h$ may be learned. This phase contains a training dataset for $\mathcal{X}$ but not necessarily $\vec{q}$. Following this offline phase, there is an online phase wherein we are given database vectors $\vec{x}$ that must be encoded and query vectors $\vec{q}$ for which we must compute the distances to all of the database vectors received so far. Once a query is received, these distances must be computed with as little latency as possible. The vectors of $\mat{X}$ may be given all at once, or one at a time; they may also be modified or deleted, necessitating re-encoding or removal. This is in contrast to most existing work, which assumes that $\vec{x}$ vectors are all added at once before any queries are recieved [][][][].

In practice, one might require the distances between $\vec{q}$ and only some of the database vectors $\mathcal{X}$ (in particular, the $k$ closest vectors). This can be achieved using an indexing structure, such as an Inverted Multi-Index [][] or Locality-Sensitive Hashing hash tables [][], that allow inspection of only a fraction of $\mathcal{X}$. Such indexing is complementary to our work in that our approach could be used to accelerate the computation of distances to the subset of $\mathcal{X}$ that is inspected. Consequently, we assume that the task is to compute the distances to all vectors, noting that, in a production setting, ``all vectors'' for a given query might be a subset of a full database.

%     -we consider only dot products and euclidean distances throughout this work since they are by far the most common, but note that our approach could in principle be generalized to any similarity or distance of the above form.\footnote{Practically speaking, functions with a large range of values for \delta(q_j, x_j), such as distances in Lp space with p > 2, are unlikely to be approximated well.}




% ================================================================
% \section{Background and Related Work}
\section{Related Work} \label{sec:relatedWork}
% ================================================================



% % ------------------------------------------------
% \subsection{Exact Matrix Multiplication}
% % ------------------------------------------------


% ------------------------------------------------
\subsection{Approximate Matrix Multiplication}
% ------------------------------------------------

sketching

joint sketching

% ------------------------------------------------
\subsection{Structured Matrices}
% ------------------------------------------------

Winograd stuff (Toeplitz matrix)

ACDC, Adaptive Fastfood, FHT, etc

Sparsity, at least for stuff like Han2015

% ------------------------------------------------
\subsection{Vector Quantization}
% ------------------------------------------------

Our method is most similar to vector quantization methods developed for similarity search []. These methods transform the two matrices such that one becomes a collection of lookup tables and the other indices into those tables. Once transformed, the matrix multiplication \textit{per se} requires only table lookups. We adopt this same pattern, but replace the multiply-intensive transform with an efficient alternative. This replacement is enabled through our introduction of a fast and trainable family of quantization functions. We discuss these methods and how we differ from them in Sections~\ref{sec:background}~and~\ref{sec:method}.



% 
\subsection{Linear Approximation}

Traditional AMM methods construct matrices $\V_A, \V_B \in \R^{D \times d}, d \ll D$ such that
\begin{align}
    \A \B \approx (\A \V_A) (\V_B^\top \B).
\end{align}
There are many options for choosing $\V_A$ and $\V_B$, depending on what assumptions one makes.

The most common case studied is that in which $N$, $D$, and $M$ are large, but one has little or no prior knowledge about either matrix. One simple approach is to sketch each matrix individually. Most of these methods center on one of three basic approaches: deterministic sketching, random sketching, or random sampling. Deterministic sketching methods employ a fixed algorithm or projection matrix to obtain a low-rank approximation to the original matrix. The most prominent of these methods are the Frequent Directions algorithm \cite{liberty_simple_2012, ghashami_frequent_2016} and its many variations \cite{teng_fast_2019, francis_practical_2018, ye_frequent_2016, huang_near_2019, luo_robust_2019, francis_improvement_2018}. Random sketching methods typically design matrices with certain properties or structure, such as having entries composed of or multiplied by i.i.d. Rademacher random variables \cite{sarlos_improved_2006, kyrillidis_approximate_2014, pagh_compressed_2013} or having a particular sparse structure \cite{osnap}. Random sampling methods choose subsets of rows and/or columns, typically in accordance with some theoretically motivated sampling distribution \cite{drineas_fast_2006-1, drineas_fast_2006-2}.

One weakness of the above approaches is that they consider each matrix individually, instead of the two of them jointly. The first authors to formally address AMM by considering both matrices were \citet{drineas_fast_2006}, who sampled columns of $\A$ and rows of $\B$ according to a sampling distribution dependent upon both matrices. Later work by \citet{manne_fast_2014} reduces approximation of the matrices to an optimization problem, which is solved by steepest descent. \citet{mroueh_co-occuring_2016}, \citet{ye_frequent_2016}, and \citet{francis_improvement_2018} introduce variations of the Frequent Directions algorithm that take into account both matrices simultaneously.

% In the information retrieval community, approximate matrix multiplication is almost exclusively studied in the regime where $N, M \gg D$ and either $\A$ or $\B$ is either fixed ahead of time or changing slowly.
% Another relevant body of methods comes from the information retrieval literature.

\subsection{Nonlinear Approximation}

While they are not traditionally considered related work for AMM methods, there are also a number of relevant approaches from the information retrieval literature. These approaches consider the tasks of nearest neighbor and maximum inner product search, not approximate matrix multiplication \textit{per se}; however, these methods either
\begin{enumerate}
    \item work by reducing the problems to approximate matrix multiplication \cite{cq, aq, otq, sq, stackedQuantizers, bolt}, or
    \item could easily be applied to approximate matrix multiplication by replacing squared euclidean distance computations with dot product computations. \cite{pq, quckAdc, quickerAdc, pairq, grvq, opq, polysemous}
\end{enumerate}
Consequently, it is useful to cast them as AMM methods.

These methods almost exclusively consider the regime where $N, M \gg D$ and either $\A$ or $\B$ is fixed ahead of time (or at least changing slowly). These papers extend the traditional AMM formulation to include a nonlinear vector quantization function $g(\cdot)$ such that
\begin{align}
    \A \B \approx g(\A \V_A) (\V_B^\top \B).
\end{align}
The $g(\cdot)$ function binarizes and induces structured sparsity in the resulting matrix. Because all of the resulting coefficients of the matrix $g(\A \V_A)$ are either 0 or 1, this allows its product with $(\V_B^\top \B)$ to be computed using only additions, instead of multiply-adds. Moreover, if the sparsity is structured appropriately, dot products can be reduced to table lookups, with the indices into the tables given by the indices of the nonzeros. As several recent papers have observed \cite{bolt, quickAdc, quickerAdc}, these tables can be restricted in size and quantized such that the table lookups can be executed with CPU vector instructions.

We discuss the most relevant of these vector quantization-based approaches in more detail in section~\ref{sec:background}, and refer the reader to \cite{learningToHashSurvey, hashingSimilaritySurvey} for detailed surveys of this area. % For now, we merely note that most of the AMM-relevant effort in the past decade has been devoted to finding better $\V_A$ and $\V_B$ matrices.
% Because of the assumption that $N, M \gg D$, applying these linear transforms is nearly free

Our method is also built on vector quantization, but instead of applying $g(\cdot)$ to a transformed matrix $\A \V_A$, we directly transform $\A$ itself. Because we use a $g(\cdot)$ function that does not require any multiply-adds, our overall procedure requires multiply-adds only if $\V_B^\top \B$ cannot be precomputed. We also depart from existing vector-quantization based AMM through the addition of several more algorithmic enhancements, but defer discussion of these to section~\ref{sec:method}.

% There have been variations of this approach (see \cite{learningToHashSurvey, hashingSimilaritySurvey} for detailed surveys), with much of the effort aimed at developing better $\V_A$ and $\V_B$ matrices.





%================================================================
\vspace{-1.5mm}
\section{Background - Product Quantization} \label{sec:background}
\vspace{-.5mm}
%================================================================





Recall that our task is to construct functions $g(\cdot)$, $h(\cdot)$, and $f(\cdot)$ such that
\begin{align}
    \norm{f(g(\A), h(\B)) - \A\B}_F < \eps(\tau) \norm{\A\B}_F
\end{align}
for the smallest $\eps(\tau)$ possible.

In this section, we discuss the vector quantization (VQ) approach to approximate matrix multiplication. Because these methods approximate matrix products by approximating individual dot products between rows of $\A$ and columns of $\B$, we do so by examining how they approximate a single dot product $\a^\top \b$ between one row of $\A$ and one column of $\B$. For simplicity, we will also begin by focusing on Product Quantization (PQ) \cite{pq}, the classic algorithm on which most others are based.

The basic intuition behind PQ is that $\a^\top \b \approx \hat{\a}^\top \b$, where $\norm{\hat{\a} - \a}$ is small but $\hat{\a}$ has special structure allowing the product to be computed quickly. This structure consists of $\hat{\a}$ being formed by concatenating learned prototypes in disjoint subspaces; one obtains a speedup by precomputing the dot products between $\b$ and the prototypes once, and then reusing these values across many $\a$ vectors.

 % \cite{pq}, a classic approach to this problem that serves as a conceptual foundation for our own method. There are three basic steps behind PQ:
In somewhat more detail, PQ consists of the following:
\begin{enumerate}
    % \item asdf
    % \item $g(\A)$
    % \item Replacing each row of $\A$ with a prototype chosen from a predefined set.%
    \item \textbf{Prototype Learning} - In an initial, offline training phase,  cluster the rows of $\A$ (or a training set $\tilde{\A}$) using K-means to create prototypes. A separate K-means is run in each of $C$ disjoint subspaces to produce $C$ sets of $K$ protypes. %The prototypes consist of the cartesian product of prototypes within disjoint subspaces.
    % \item $g(\A)$ - Replace each row of $\A$ with the most similar prototype.
    % \item $h(\B)$ - Precompute the dot products between each column of $\B$ and each prototype.
    \item \textbf{Prototype Assignment}, $g(\a)$ - Determine the most similar prototype to $\a$ in each subspace. Store these assignments as integer indices using $C \log_2(K)$ bits.
    \item \textbf{Table Construction}, $h(\B)$ - Precompute the dot products between $\b$ and each prototype in each subspace. Store these partial dot products in $C$ lookup tables of size $K$.
    \item \textbf{Aggregation}, $f(\cdot,\cdot)$ - Use the indices and tables to \textit{lookup} the estimated partial $\a^\top \b$ in each subspace, then sum the results across all $C$ subspaces. %$A[i]^\top B[:, j]$\footnote{Because we will frequently need to refer to slices of rank-3 tensors and above, we employ Python-style slicing notation rather than traditional superscripts and subscripts.} rather than compute it with a series of multiply-adds.
\end{enumerate}

PQ is depicted for a single pair of vectors $\a$ and $\b$ in Figure~\ref{fig:pq}. We elaborate upon each of these steps below.

\begin{figure}[h]
\begin{center}
\includegraphics[width=\linewidth]{amm/pq}
\caption{Product Quantization, with prototype vectors already learned. The $g()$ function returns the index of the most similar prototype to the data vector a in each subspace. The $h(\cdot)$ function computes a lookup table of dot products between the query vector b and each protoype in each suspace. The aggregation function $f(\cdot,\cdot)$ sums the table entries in each subspace corresponding the the most similar prototype.}
% $\vec{a}$
\label{fig:pq}
\end{center}
\end{figure}

% ------------------------------------------------
\subsection{Prototype Learning}
% ------------------------------------------------

Let $\tilde{A} \in \R^{N \times D}$ be a training set, $K$ be a number of prototypes per subspace, $C$ be a number of subspaces, and $\{\mathcal{J}^{(c)}\}_{c=1}^C$ be the (mutually exclusive and collectively exhaustive) sets of indices associated with each subspace. The training-time task of PQ is to learn $C$ sets of prototypes $\mat{P}^{(c)} \in \R^{K \times |\mathcal{J}_c|}$ and assignments $\vec{z}^{(c)} \in \R^{N}$ such that:
\begin{align}
    \sum_{i=1}^N \sum_{c=1}^C \sum_{j\in \mathcal{J}_c} \tilde{\A}_{ij} - \mat{P}^{(c)}_{z^{(c)}},j
\end{align}
is minimized. It does this by running K-means separately in each subspace $\mathcal{J}$ and using the resulting centroids and assignments to populate $\mat{P}$ and $\vec{z}$.

% Perhaps the simplest means of constructing prototypes would be to run a clustering algorithm like K-means on the rows of $\A$. This would be effective, but would require using a large number of prototypes in order to have each row closely match its prototype. What would be preferable is some means of having a huge number of prototypes without having to pay so large a time and space cost.

% PQ achieves this goal by using as prototypes the cartesian product of prototypes within disjoint subspaces. Concretely, PQ runs K-means with $K$ centroids in each of $C$ disjoint (usually contiguous) sets of dimensions. This results in $K^C$ possible combinations of centroid assignments for each vector, with each unique combination corresponding to a unique overall prototype.

% vectors into disjoint ``subvectors'' (each corresponding to a unique set of dimensions) and runs k-means within each subspace.
%  If there are $K$ prototypes per subspace and $C$ subspaces, this results in $K^C$ possible combinations of prototype assignments for each vector. Viewed differently, this creates $K^C$ overall prototypes whose entries

% ------------------------------------------------
\subsection{Encoding Function - $g(\A)$}
% ------------------------------------------------

Given the learned prototypes, PQ replaces each row $\a$ of $\A$ with the concatenation of its $C$ K-means centroid assignments in each of the $C$ subspaces. Formally:
\begin{align} \label{eq:pqLoss}
    g(\a)^{(c)} \triangleq \argmin_k \sum_{j\in \mathcal{J}_c} (\a_j - \mat{P}^{(c)}_{k,j})^2
\end{align}
We will refer to the resulting sequence of indices as the \textit{encoding} of $\a$ and the set of $K$ centroids as a \textit{codebook}. This encoding function has complexity $\Theta(KD)$ for each $\a$, since there are $K$ centroids per subspace and the sum of the number of indices in all subspaces is $D$.

% ------------------------------------------------
\subsection{Table Construction - $h(\B)$}
% ------------------------------------------------

Using these same prototypes, PQ constructs a lookup table $h(\b)^{(c)} \in \R^K$ in each of the $C$ subspaces for each column $\b$ of $\B$, where
\begin{align}
    h(\b)^{(c)} \triangleq \sum_{j\in \mathcal{J}_c} \b_j \mat{P}^{(c)}_{k,j}
\end{align}
This has complexity $\Theta(KD)$ for each $\b$, since it is essentially the same as $g(\cdot)$ except without the $\argmin$ operation.

% ------------------------------------------------
\subsection{Aggregation - $f(\cdot,\cdot)$}
% ------------------------------------------------

Given the encoding of $\a$ and the lookup tables for $\b$, the product can be approximated as
\begin{align}
    \a^\top \b \approx \sum_{c=1}^C h(\b)^{(c)}_k, \text{ }k = g(\a)^{(c)}
\end{align}

This has complexity $\Theta(C)$ for each pair of vectors, or $\Theta(NMC)$ for the full matrix product $\A\B$. Since $C \ll D$, this provides a large speedup when $N$ and $M$ are large enough to amortize the costs of $g(\cdot)$ and $h(\cdot)$.

% ------------------------------------------------
\subsection{Complexity and Extensions}
% ------------------------------------------------

The total time complexity of PQ is equal to $\Theta(NDK + DKC + NMC)$. When $N$ and $M$ are large, the last term dominates and the complexity of PQ is essentially $\Theta(NMC)$.
Assuming one does not use Strassen's or other algorithms that are never used in practice in modern numerical libraries, full matrix multiplication has complexity $\Theta(NDM)$. Thus, PQ offers a speedup of $\Theta(\frac{D}{C})$.

To increase this ratio, many authors have attempted to construct more elaborate $g(\cdot)$ and $h(\cdot)$ functions. One line of work seeks to preprocess $\a$ and/or $\b$ with rotation matrices \cite{opq,cartesianKmeans,lopq}. Another line of work seeks to relax the restriction that the codebooks use disjoint subspaces; several authors \cite{aq,cq,otq,sq,grvq,stackedQuantizers} have observed that equation~\ref{eq:pqLoss} is a special case of
\begin{align} \label{eq:nonOrthogonal}
    \sum_{i=1}^N min_{\{z^{(1)},\ldots,z^{(C)}\}} \sum_{j=1}^D \left( \tilde{A}_{ij} - \sum_{c=1}^C \mat{P}^{(c)}_{z^{(c)},j} \right) ^2.
\end{align}
That is, all $C$ codebooks can contribute to approximating any dimension of $\a$, instead of only one codebook being responsible for it. This is strictly more expressive, but also slows down $g(\cdot)$ greatly and complicates training. There has been effort to reduce these burdens by carefully ordering update and assignment operations \cite{stackedQuantizers}, adding sparsity \cite{sq}, or structuring overlap between codebooks so that optimal assignments remain possible \cite{otq}. TODO sentence about optimizing LUTs directly.


%\footnote{This is assuming one does not uses Strassen's or other lower-complexity algorithms that are never used in practice in modern numerical libraries.} When $N$ and $M$ are large, the last term dominates and the complexity of PQ is essentially $\Theta(NDK + DKC + NMC)$. % We do not compare to Strassen's since we are concerned with CPU time on real-world matrices, not assymptotic complexity}.

% % ------------------------------------------------
% \subsection{Problem Formulation}
% % ------------------------------------------------

% Let $\A \in \R^{N \times D}$ and $\B \in \R^{D \times M}$ be two matrices, with $N \gg D, M$, and $M$ not significantly larger than $D$. Given a computation time budget $\tau$, our task is to
% construct three functions $g(\cdot)$, $h(\cdot)$, and $f(\cdot)$ such that
% \begin{align}
%     \norm{f(g(\A), h(\B)) - \A\B}_F < \eps(\tau) \norm{\A\B}_F
% \end{align}
% for the smallest error $\eps(\tau)$ possible.

% % . Moreover, the computation time $\tau(\eps)$ to run these three functions should be much smaller than the comuputation time $\tau_0$ to compute $\A\B$ directly.

% We assume the existence of a training set $\tilde{\A}$, whose rows are drawn from the same distribution as the rows of $\A$. This is a natural assumption in the case that rows of $\A$ represent examples in training data, or structured subsets thereof (such as patches of natural images). This assumption is common in the information retrieval literature \cite{bolt, pairq}.%, but not in theoretical work.

% % We also focus on the case in which $B$ has at most a few hundred columns--or even as few as 10. This is the range of sizes that one might obtain in a neural network, where each column of $B$ corresponds to one output neuron. It is also small enough that asymptotic complexity is a not a good characterization of how an algorithm will perform in practice.

% We also assume that the cost of computing $h(\B)$ negligible provided that $h(\B) \in O(MD)$. One sufficient condition for this is $\B$ being provided ahead of time (as is the case when $\B$ represents predefined model weights). A second is that $\B$ has sufficiently few columns compared to the number of rows in $\A$.%, which happens when $N$ is sufficiently large relative to $M * \frac{\tau(\eps)}{\tau(0)}$.

% % % ------------------------------------------------
% % \subsection{Existing approaches}
% % % ------------------------------------------------

% % Many existing approaches construct a matrix $\V \in \R^{D \times d}, d < D$ such that
% % \begin{align}
% %     \A \B \approx (\A \V) (\V^\top \B).
% % \end{align}
% % Often, $\V$ is sparse [] or has structure [] such that these projection operations are faster than a dense matrix multiply.


% % % ------------------------------------------------
% % \subsection{Problem Formulation}
% % % ------------------------------------------------

% % We consider three variations of the approximate matrix multiply problem. In all cases, let X = ...., <other variable definitions here>.

% % Maybe a table of assumptions of different methods as far as training data and preprocessing time. And/or which problem statement they try to solve.

% % Problem 1: No training data at all (most AMM work, ours with LUT computation + untrained encoding)

% % Problem 2: Training data for one matrix (ours with LUT computation)

% % Problem 3: One matrix fixed (VQ stuff with no optimizing wrt query, ours without learned quantization)

% % Problem 4: Training data for both matrices (VQ stuff that rotates using query distro, Bolt, ours with LUT computation + weighting errs)

% % Problem 5: Training data for one matrix, other matrix fixed (ours)

% % % this is not quite MECE because could have one matrix fixed and other completely unknown, in which case no LUT computation but also no trained quantizers. Probably just note that this possibility exists and our method could do it, but we ignore it for rest of the paper because

% % Note that this is maybe not a sufficient characterization because ours is also mostly just useful when one of the matrices is like really small.

% % Should probably just show ours, ours without learned quantization func, and ours with LUT computation at runtime. As a result, should only introduce these three problems.

% % % ------------------------------------------------
% % \subsection{Vector Quantization}
% % % ------------------------------------------------

% % With problem statement ourFrame stuff using the two encoding functions and one distance function.

% % First walk through basic VQ-based dot product, with img from poster.


%================================================================
% \vspace{-5mm}
\vspace{-3mm}
\section{Our Method} \label{sec:method}
\vspace{-.5mm}
%================================================================


Recall that our task is to construct functions $g(\cdot)$, $h(\cdot)$, and $f(\cdot)$ such that
\begin{align}
    \norm{f(g(\A), h(\B)) - \A\B}_F < \eps(\tau) \norm{\A\B}_F
\end{align}
$\eps(\tau)$ is minimized given a time constraint $\tau$.

% ------------------------------------------------
\subsection{Quantization Function - $g(\A)$}
% ------------------------------------------------

We begin by discussing

Basically just intuition for why it just splits on a bunch of dims. Then define a split formally. Then pseudocode (or probably just formula for applying them)

% ------------------------------------------------
\subsection{Learning Splits}
% ------------------------------------------------

Introduction to fact that we have two variations, one that requires training data and one that doesn't. Former is greedy and mention that we prove it's close to optimal.

Pseudocode for probably both, but maybe just the greedy one.

% ------------------------------------------------
\subsection{Table Construction - $h(\B)$}
% ------------------------------------------------

% ------------------------------------------------
\subsection{Lookup Scan - $f(\cdot,\cdot)$}
% ------------------------------------------------

If we didn't have pseudocode in background section, put it here. Break it down by encoding, LUT creation, and distance computation. Seems like we should probably already have explained these, so probably just text saying we kind of tie it all together and replace whatever lines in earlier pseudocode with hash-based encoding.




%================================================================
% \vspace{-2mm}
\section{Experiments} \label{sec:results}
% \vspace{-.5mm}
%================================================================


To assess Bolt's effectiveness, we implemented both it and the algorithms to which we compare it in C++, with wrappers in Python. All of our code and raw results are publicly available at [website]. This website also s contains many additional experiments and thorough documentation of both our code and experimental setups that we hope facilitates reproduction and extension of our work. All experiments use a single thread on a 2013 Macbook Pro with a 2.6GHz Intel Core i7-4960HQ processor. % This processor supports 32 simultaneous table lookups (c.f. Section~\ref{sec:boltVectorize}), which is fewer than the 64 available on more recent processors; this suggests that Bolt would be even faster on a more recent machine.

\subsection{Datasets}

Because there are no conditional branches in either Bolt or the comparison algorithms (when implemented efficiently), all running times are independent of the data and query distributions. Consequently, we report timing results primarily on random data. All timings are the best of 5 runs, averaged over 10 trials (i.e., the code is executed 50 times). We use the best in each trial, rather than average, since this is standard practice in performance benchmarking.

For assessing accuracy, we use several datasets widely used to benchmark Multi-Codebook Quantization (MCQ) algorithms:
\begin{itemize}
\item \textbf{Sift1M} [] --- 1 million 128-dimensional SIFT descriptors of images. Sift1M vectors tend to have high correlations among many dimensions, and so be highly compressible for algorithms that allow global rotations or do not employ space partitions. This dataset has a predefined query/train database/test database split, consisting of 10,000 query vectors, 100,000 training vectors, and 1 million testing vectors.
\item \textbf{Convnet1M} [] --- 1 million 128-dimensional Convnet descriptors of images. These vectors have some amount of correlation, but less so than Sift1M. It has a query/train/test split matching that of Sift1M.
\item \textbf{LabelMe22k} [] --- 22,000 960-dimensional GIST descriptors of images. Like Sift, it has a great deal of correlation between many dimensions. It only has a train/test split, so we follow [][] and use the 2,000-vector test set as the queries and the 20,000 vector training set as both the training and test databse.
\item \textbf{MNIST} [] --- 60,000 28x28-pixel greyscale images, flattened to 768-dimensional vectors. This dataset is sparse and has high correlations between various dimensions. Again following [] and [], we split it the same way as the LabelMe dataset.
\end{itemize}

Experiments on additional datasets are availble on the Bolt website [website].

\subsection{Comparison Algorithms}

Our comparison algorithms include MCQ methods that have high encoding speeds ($\ll 1$ms / vector on a CPU). If encoding speed is not a design consideration or is dominated by a need for maximal compression, we recommend using a method such as GRVQ [] or LSQ [] instead of Bolt\footnote{Although Bolt \textit{might} still be desirable for its high query speed even if encoding speed is not a consideration.}.

Since most research in this area has focused on improving accuracy for a given code length at the \textit{expense} of encoding time, only Product Quantization (PQ) [], Optimized Product Quantization (OPQ) [], and variations thereon such as [polysemous], [googleMips], [pairwiseQ] [LOPQ], [NOIMI] meet our inclusion criterion. Since [LOPQ], [NOIMI] and similar methods are coupled to an indexing structure, which is compatible with but orthogonal to our work, we do not compare to them. Moreover, the optimization of [googleMips] is essentially subsumed by that of [pairwise], which can itself by combined with OPQ by multiplying the rotation matrix by another matrix learned from the query distribution. As a result, the effective encoding and query speeds of each these methods are equal to those of OPQ. In short, by comparing to PQ and OPQ, we can effectively assess the performance of almost all fast-encoding MCQ algorithms, modulo the accuracy improvements afforded by different optimization approaches for OPQ.

We do not to compare to binary embedding methods as they are known to yield much lower accuracy for a given code length than MCQ methods [][][] and, as we show, are also slower in computing distances than Bolt. % This implies that there is no reason to prefer them to Bolt, unless one both required an even more extreme encoding speed than Bolt's ($\gg$ 2GB/s) and devised a binary embedding algorithm that could achieve this.

We have done our best to optimize the implementations of the comparison algorithms to the greatest extent possible, and find that our timings are far superior to those described in previous works. For example, [] reports encoding roughly 190,000 128-dimensional vectors per second with PQ, while our implementation encodes over 24 million per second. Indeed, we believe that the sheer extent to which PQ and related methods can be sped up is an interesting result in itself, though we do not explore it further.

TODO integrate pairQ mat + OPQ rotation mat for each Bolt subspace; then, just compare to OPQ with pairQ premultiply, since it's strictly better. Or maybe PQ, OPQ, PairQ using OPQ so we have 3 things instead of 2.

% \subsection{Comparison }

\subsection{Greater Speed than Binary Embedding}

As mentioned in Section~\ref{sec:relatedWork}, the current fastest method of obtaining approximate distances over compressed vectors is to embed them into Hamming space and use the \texttt{popcount} instruction to quickly compute the Hamming distances between them. Indeed, the speed of this approach is much of the motivation for the class of binary embedding methods (e.g., [][][][]), as well as more recent attempts to integrate hamming distances into product quantization [polysemous].

In shown in Figure~\ref{fig:bolt_vs_popcount}, Bolt can compute approximate distances (of any kind expressible via ~\ref{eq:distFuncForm}) faster than popcount can compute Hamming distances. Moreover, because Bolt can compute Hamming distances exactly (since the possible distances between pairs of 4 bits can be stored exactly in a 16B lookup table), it is both faster and more flexible than using popcount. It also has the benefit that it can use encoding lengths that are not multiples of 8B without loss of efficiency.

\begin{figure}[h]
\begin{center}
\label{fig:bolt_vs_popcount}
% \includegraphics[width=\linewidth, trim={0 2cm 0 0},clip]{moose0}
% \includegraphics[width=\linewidth, trim={0 1cm 0 0},clip]{moose0}
\includegraphics[width=\linewidth]{moose0}
\vspace*{-1mm}
\caption{Bolt can compute various distances, including Hamming distances, faster than binary embedding methods can compute Hamming distances (using the \texttt{popcount} instruction).}
\end{center}
\end{figure}

Above figure should use 8, 16, 32B. And ideally version of Bolt that immediately upcasts to uint16s, which is the most robust to overflows but might not be faster than popcount.

\subsection{Encoding Speed}

Bolt can encode both data and queries faster than any other Multi-Codebook Quantization (MCQ) method of which we are aware. As shown in Figure~\ref{fig:encoding_speeds}a, Bolt can encode data vectors at up to 2GB/s, while the fastest comparison, PQ, reaches at most 200MB/s. For perspective, Bolt's encoding rate is sufficient to encode the entire Sift1M dataset of 1 million vectors in 250ms, and the Sift1B dataset of 1 billion vectors in 250s. This rate is also much higher than that of high-speed (but general-purpose) compression algorithms such as Snappy [], which reports an encoding speed of 250MB/s.

Similarly, Bolt can compute the distance matrix constituting a query's encoding at up to 7 million queries/s, while PQ obtains only 500,000 queries/s Figure~\ref{fig:encoding_speeds}b. Both of these numbers are sufficiently high that encoding the query is unlikely to ever be a bottleneck in computing distances to it.

\begin{figure}[h]
\begin{center}
\label{fig:encoding_speeds}
% \includegraphics[width=\linewidth, trim={0 2cm 0 0},clip]{moose0}
% \includegraphics[width=\linewidth, trim={0 1cm 0 0},clip]{moose0}
\includegraphics[width=\linewidth]{moose1}
\vspace*{-1mm}
\caption{Bolt encodes both data vectors and query vectors significantly faster than existing algorithms.}
\end{center}
\end{figure}

Above needs to show times for 8B and 16B codes.
TODO vary query batch size? Because OPQ (and bolt if we add in rotations) will go faster with bigger batches


\subsection{Query Speed}

Much of the appeal of MCQ methods is that they allow fast computation of approximate distances and similarites directly on compressed data. We compared Bolt's speed in computing Euclidean distances from a batch of queries to each vector in a compressed dataset. We omit profiling of other distances and similarities since they only alter the computation of queries' distance matrices and therefore have nearly identical speeds. In all experiments, the number of compressed data vectors $N$ is fixed at $100,000$.

\begin{figure}[h]
\begin{center}
\label{fig:query_speeds}
\includegraphics[width=\linewidth]{moose2}
\vspace*{-1mm}
\caption{Bolt can compute the distances/similarities between a query $\vec{q}$ and the vectors of a compressed database $H$ up to $5\times$ faster than other MCQ algorithms. It is also faster than batched distance computations on floats using matrix multiplies up to a batch size of TODO.}
\end{center}
\end{figure}

% In addition to reducing the space consumption of a database of vectors $\mathcal{X}$, MCQ methods also allow direct computation of distances and similarities matching the form of~\ref{eq:distFuncForm}.


Use query batch sizes of {1, 32, 64, 128,...,512}; NBytes={8, 16, 32}; fix N at 100k. Compare Bolt, matmul, other MCQ methods.


\subsection{Nearest Neighbor Accuracy}

By far the most common assessment of MCQ and binary embedding algorithms' accuracy is their Recall@R on various benchmark datasets. The Recall@R is the fraction of the queries $\vec{q}$ for which the true nearest neighbor in Euclidean space is among the top $R$ points with smallest approximate distances to $\vec{q}$. As shown in Figure~\ref{fig:nn_acc}, Bolt yields lower accuracy for a given encoding length than other methods designed for maximal compression. And hopefully some comment about how it's at least better than PQ.

\begin{figure}[h]
\begin{center}
\label{fig:nn_acc}
\includegraphics[width=\linewidth]{moose3}
\vspace*{-1mm}
\caption{Bolt is less accurate in retrieving the nearest neighbor for a given encoding length than MCQ algorithms optimized for small encodings, but $<$ hopefully optimistic comment $>$.}
\end{center}
\end{figure}

Datasets: Sift1M, LabelMe, Convnet1M, MNIST; displayed in 2x2 grid of subplots
Within each plot, have 3 lines for each algo: dotted for 32B, solid for 16B, dashed for 8B.
Revert to table if time crunched.


% \begin{table}[h]
%   \caption{Recall@R for Bolt and other MCQ algorithms}
%   \label{tbl:nn_accuracy}
%   % \def\arraystretch{1.1}%  1 is the default, change whatever you need
%   \begin{tabularx}{\linewidth}{X|YYYY}
% \toprule
%             &  Recall@1 & Recall@10 & Recall@100 & Recall@1000
% \hline
%     Bolt 8B    & 0 & 0 & 0 & 0
% \bottomrule
% \end{tabularx}
% \end{table}

\subsection{Maximum Inner Product Search Accuracy}

Because dot products are perhaps even more common vector operations than Euclidian distance computations, we repeated the previous experiment for maximum inner product search (MIPS) instead of 1nn search (Figure~\ref{fig:mips_acc}).

\begin{figure}[h]
\begin{center}
\label{fig:mips_acc}
\includegraphics[width=\linewidth]{moose4}
\vspace*{-1mm}
\caption{Bolt is less accurate in retrieving the neighbor having maximum dot product with the query than MCQ algorithms optimized for small encodings. Again, $<$ hopefully optimistic comment $>$.}
\end{center}
\end{figure}

Might not get to this because we need to modify Bolt to allow int8s in LUT instead of just uint8s.

\subsection{Accuracy in Preserving Distances and Dot Products}

The above experiments characterize how well algorithms preserve distances and similarities to exceptionally similar points, but not whether distances and similarities in general tend to be preserved. To characterize this, we computed the correlations between the true Euclidean distances and approximate Euclidean distances from Bolt and other MCQ algorithms on the LabelMe22k and MNIST datasets (see [website] for results on other datasets).

\begin{figure}[h]
\begin{center}
\label{fig:corr_acc}
\includegraphics[width=\linewidth]{moose5}
\vspace*{-1mm}
\caption{Bolt Euclidean distances are highly correlated with true Euclidean distances, though slightly less so than the distances from other MCQ algorithms.}
\end{center}
\end{figure}

Maybe omit this because accuracy results make us look worse and it's not reported in any other MCQ paper.


\subsection{Case Study: K-Means Clustering}

In addition to accelerating standalone similarity searches, Bolt can be embedded within other algorithms to improve their speed. As an example, we show that using Bolt to compute the distances between centroids and data vectors can greatly accelerate k-means clustering of the full MNIST dataset with little or no loss in accuracy (Figure~\ref{fig:kmeans}). Accuracy is measured by mean squared Euclidean distance between vectors and their associated centroids. This distance is computed exactly, not with Bolt. Reported times are the for Bolt include initial encoding of the data.

\begin{figure}[h]
\begin{center}
\label{fig:kmeans}
\includegraphics[width=\linewidth]{moose6}
\vspace*{-1mm}
\caption{Using Bolt to compute distances within K-means decreases its runtime greatly without increasing its mean squared error (MSE). Standard errors across 10 runs are shown shaded.}
\end{center}
\end{figure}

Use K = 16, K = 256.

\subsection{Case Study: Power Iteration}

Backup if we don't have time for word embedding. Should compare time taken to find top k eigenvects + eigenvals, and just report cosine sims between former and raw values of latter in a table. Use gram schmidt to get eigenvectors after the first (think we have to decompress to do this--so this might not be faster at all).

\subsection{Case Study: Word Embedding}

Hope I have time to do this. Need to ask whether there's an obvious successor to word2vec other than Glove (which isn't amenable to our approach).



% show that it makes matrix-vector muls way faster for various matrix sizes
%     -also show approximation error
%         -which suggests getting lots of meaningful matrices, ideally fc layer weights
%     -prolly have to show both speed and err as a function of code length
% same for matrix-matrix; speed and acc
%     -is there any way we'll be better here? maybe for skinny mats...

% show that we can speed up kmeans a lot, and point out that this is orthogonal to other ppl's speedups based on pruning which ones you consider moving and/or using minibatches
%     -prolly pick a couple datasets and show it as a function of k

% a couple experiments on sift1m scan and mnist scan, prolly; maybe sift10m / deep10m also

% somewhere introduce the "cold scan" problem, wherein we also have to add in time to compress everything, but then we get a batch of queries
%     -if just 1 query, you're always worse off doing the encoding
%     -metric of interest is how many queries you have to do @k database vectors before it's faster than a matmul
%         -and you should also report the times for different query counts and db counts

%     -and prolly also the "warm scan" problem; you only have to encode some subset of the db (cuz it changed)

%     -or maybe the "hot scan" problem cuz data is hot; in contrast to fixed data, which is cold/frozen





%================================================================
% \vspace{-2.5mm}
\vspace{-.5mm}
\section{Discussion and Conclusion}
\vspace{-.5mm}
%================================================================

We introduce \ours, an algorithm that achieves up to a $10\times$ better speed-quality tradeoff than existing methods for the well-studied problem of approximate matrix multiplication, as measured on a large, diverse, and challenging set of real-world matrices.
% substantially larger than that used in any previous work.
% , as measured using a large, diverse, and challenging set of real-world matrices.
Our method also features theoretical guarantees, as well as a number of algorithmic contributions likely to be of more general interest. %While the focus of this paper is matrix multiplication, our results suggest that future methods similar to our own might hold promise for accelerating convolution, deep learning, and other workloads bottlenecked by linear transforms.

% \ours's efficacy for matrix multiplication suggests that similar methods based on vector quantization and byte shuffling may be promising for accelerating convolution, deep learning, and other workloads bottlenecked by linear transforms.

% \vspace{-.25mm}
The main limitation of our work is that we do not demonstrate (or claim) speedups on Google TPUs, recent NVIDIA GPUs, or other accelerators specialized for traditional matrix products. However, 1) such accelerators are a small minority of computational devices, and 2) our work does suggest a possible path for \textit{future} accelerators to be much more efficient. Namely, replacing the multiply-add hardware needed by current approaches with the multi\textit{plex}-add hardware needed for fast table lookups could both save many transistors and likely increase output quality.

% hardware acceleration for the hashing, averaging, and table lookup operations we employ could both increase quality of output and require fewer transistors than current multiplication-based approaches.

% than are needed by multipliers, and would (as we demonstrate) likely

% on which we rely would require fewer transistors than are needed by multipliers, and would (as we demonstrate) likely achieve better quality than current approaches.

%  simultanebe implemented with far fewer transistors than are needed by multipliers, and 2) could (as we demonstrate) approximate linear transforms at least as well traditional approaches like scalar quantization.

% \vspace{-.25mm}
Finally, while the focus of this paper is matrix multiplication and extending it beyond this will require further research, our results suggest that similar methods might hold promise for accelerating convolution, deep learning, and other workloads bottlenecked by linear transforms.

% In future work, our results suggest that methods similar to our own might hold promise for accelerating convolution, deep learning, and other workloads bottlenecked by linear transforms.

% In future work, our results

% that using methods similar to our own accelerating

 %---though realizing this potential will require a great deal of further research and engineering.

% Our results suggest that methods similar to our own could hold promise for accelerating convolution, deep learning, and other workloads bottlenecked by linear transforms, but we emphasize that such extensions will require significant further research and the present work claims superiority only for \textit{approximate matrix multiplication}.
% the present work only claims superiority for \textit{approximate matrix multiplication} as described in our problem formulation. In future work

% Our method's performance stems from 1) its replacement of the bottleneck in existing methods with a learned locality-sensitive hash function, 2) its optimization of a quantity that other methods cannot optimize without a large slowdown, and 3) its use of an inexact estimator for computing sums.
% In future work, we plan to specialize our method for convolutions, implement it on GPUs, and integrate it into neural networks.
% In future work, we plan to integrate our method into neural networks, specialize it for convolutions, and implement it on GPUs.

% ================================================================
% References
% ================================================================

% \IEEEtriggeratref{27} % trigger column break to make cols even
% \bibliographystyle{ACM-Reference-Format}
% \bibliographystyle{abbrev}
% \bibliographystyle{sysml2019}
\bibliographystyle{icml2020}
% \bibliography{prune,architectures,misc,understandDnn,classic,datasets,compress,science,metapapers}


% TODO uncomment

\bibliography{architectures,datasets,misc,sprintz,extract,bolt,backprop-alternatives,distillation,fast-dnn-runtime-stuff,fast-optimize,nets-theory,small-fast-arch,sparseAndPrune,scalarQuantize,vectorQuantize,combo,hashing,shrink+prune,binarize,ternary,automl,zero-shot,few-shot,amm,ammMore,lsh,adversarial,libs,math}
\clearpage
\newpage  % so we can cut the pdf
\appendix

% ================================================================
\section{Quantization and \oursHash} \label{sec:hashQuantize}
% ================================================================

The only subtlety in vectorizing our hash function is that one must execute line 4 using shuffle instructions such as \texttt{vpshufb} on x86, \texttt{vtbl} on ARM, or \texttt{vtbl} on PowerPC. In order to do this, the split values and scalars $x_{j^t}$ must be 8-bit integers. We quantize them by learning for each split index $j$ a pair of scalars $(\gamma_j, \delta_j)$, where
\begin{align}
    \delta_j &\triangleq \min_i \v^j_i \label{eq:offset} \\
    % \gamma_j \triangleq 2^l, l = \left \lfloor \text{log2} \left( \frac{255}{\max_i \v^j_i - \delta_j} \right) \right \rfloor
    \gamma_j &\triangleq 2^l, l = \floor{ \log_2 \left( \frac{255}{\max_i \v^j_i - \delta_j} \right) \label{eq:scale} }
\end{align}
% and
% \begin{align}
% %     &\min_i \v^j_i - \delta_j = 0 \\
% %     &\max_i \gamma_j(\v^j_i - \delta_j) \le 255
% \end{align}
% and $\gamma_j$ is a power of 2.
This restriction of $\gamma_j$ to powers of two allows one to quantize $x_{j^t}$ values with only shifts instead of multiplies. The $\vec{v}$ values can be quantized at the end of the training phase, while the $x_{j^t}$ values must be quantized within Algorithm~\ref{algo:ourEnc} before line 5.

% ================================================================
\vfill\break  % columnbreak
\section{Subroutines for \oursHash} \label{sec:optimalSplitVal}
% ================================================================

\begin{algorithm}[h]
\caption{Optimal Split Value Within a Bucket} \label{algo:optimalSplitVal}
\begin{algorithmic}[1]
    \STATE {\bfseries Input:} bucket $\mathcal{B}$, index $j$
    \STATE {$\mat{X} \leftarrow \texttt{as\_2d\_array}(\mathcal{B})$ }
    \STATE {$\mat{X}^{sort} = \texttt{sort\_rows\_based\_on\_col}(\mat{X} \text{, } j)$}
    \STATE {$\texttt{sses\_head} \leftarrow \texttt{cumulative\_sse}(\mat{X}^{sort}, \texttt{false}) $}
    \STATE {$\texttt{sses\_tail} \leftarrow \texttt{cumulative\_sse}(\mat{X}^{sort}, \texttt{true}) $}
    \STATE {$\texttt{losses} \leftarrow \texttt{sses\_head} $}
    \STATE {$\texttt{losses}_{1:N-1} \leftarrow \texttt{losses}_{1:N-1} + \texttt{sses\_tail}_{2:N} $}
    % \STATE {$ n^\ast \leftarrow \min_n \sum_{d=1}^D $}
    % \STATE {$ n^\ast \leftarrow \argmin_n \texttt{sses\_head}_n + \texttt{sses\_tail}_{n+1} $}
    \STATE {$ n^\ast \leftarrow \argmin_n \texttt{losses}_n $}
    \STATE{$ \textbf{return } (\mat{X}^{sort}_{n^\ast \text{, } j} + \mat{X}^{sort}_{n^\ast + 1 \text{, } j}) / 2 \text{, } \texttt{losses}_{n^\ast} $}
    % \STATE {$\texttt{sses\_tail} \leftarrow \texttt{reverse\_row\_order}(\texttt{cumsse\_cols}(\texttt{reverse\_row\_order}(\mat{X}_{sort})_) $}
    % \STATE {$\texttt{sses\_tail} \leftarrow \texttt{reverse\_row\_order}(\texttt{cumsse\_cols}(\texttt{reverse\_row\_order}(\mat{X}_{sort})_) $}

\end{algorithmic}
\end{algorithm}



\begin{algorithm}[h]
% \caption{Cumulative SSE Within Columns} \label{algo:cumSSE}
\caption{Cumulative SSE} \label{algo:cumSSE}
\begin{algorithmic}[1]
    \STATE {\bfseries Input:} 2D array $\mat{X}$, boolean \texttt{reverse}
    \STATE {$N, D \leftarrow \texttt{shape}(\mat{X})$}
    \IF {\texttt{reverse}}
        \STATE{$ \forall_i \texttt{ swap}(\mat{X}_{i,d}, \mat{X}_{N-i+1,d}) $}
    \ENDIF

    % \STATE {$\texttt{out} \leftarrow \texttt{empty}(N \text{, } D)$}
    \STATE {$\texttt{out} \leftarrow \texttt{empty}(N)$}
    \STATE {$\texttt{cumX} \leftarrow \texttt{empty}(D)$}
    \STATE {$\texttt{cumX2} \leftarrow \texttt{empty}(D)$}

    \LINECOMMENT{Initialize first row of output and cumulative values}
    \STATE{$\texttt{out}_{1} \leftarrow 0 $}
    \FOR{$d \leftarrow 1 \textbf{ to } D $}
        \STATE{$\texttt{cumX}_d \leftarrow X_{1, d} $}
        \STATE{$\texttt{cumX2}_d \leftarrow (X_{1, d})^2 $}
        % \STATE{$\texttt{out}_{1, d} \leftarrow 0 $}
    \ENDFOR

    \LINECOMMENT{Compute remaining output rows}
    \FOR{$n \leftarrow 2 \textbf{ to } N $}
        \STATE{$\texttt{out}_{n} \leftarrow 0 $}
        \FOR{$d \leftarrow 1 \textbf{ to } D $}
            \STATE{$\texttt{cumX}_d \leftarrow \texttt{cumX}_d + X_{1, d} $}
            \STATE{$\texttt{cumX2}_d \leftarrow \texttt{cumX2}_d + (X_{1, d})^2 $}
            % \STATE{$\texttt{meanX} \leftarrow \texttt{cumX}_d / n $}
            % \STATE{$\texttt{out}_{n, d} \leftarrow \texttt{cumX2}_d - (\texttt{cumX}_d \times \texttt{cumX}_d / n)$}
            \STATE{$\texttt{out}_{n} \leftarrow \texttt{out}_{n} + \texttt{cumX2}_d - (\texttt{cumX}_d \times \texttt{cumX}_d / n)$}
        \ENDFOR
    \ENDFOR
    \STATE{\textbf{return } \texttt{out}}
\end{algorithmic}
\end{algorithm}

% ================================================================
\vfill\break  % columnbreak
\section{Analysis of Aggregation Using Pairwise Averages} \label{sec:aggregateAnalysis}
% ================================================================


\begin{theorem}[Bias and Variance of Sum Estimator]
Let $\hat{s}(\x), \x \in \R^C, C \text{ \% } U = 0$ be the sum estimator described in section~\ref{sec:aggregate} and let $s(\x) \triangleq \sum_c x_c$. Further suppose that the low bits of any two scalars averaged are drawn from a Bernoulli(.5) distribution. Then $E[s(\x) - \hat{s}(\x)] = C \log_2(U) / 4$ and $E[(s(\x) - \hat{s}(\x))^2] = C \log_2(U) / 8$.
\end{theorem}

\begin{proof}
TODO
\end{proof}


% ================================================================
\vfill\break  % columnbreak
\section{Additional experimental details} \label{sec:experimentDetails}
% ================================================================

% ------------------------------------------------
\subsection{SparsePCA details}
% ------------------------------------------------

We took steps to ensure that SparsePCA's results were not hampered by insufficient hyperparameter tuning. First, for each matrix product, we tried a range of lambda values which we found to encompass the full gamut of nearly 0\% to nearly 100\% sparsity: $\lambda \in 2^i, i \in \{-5, -4, -3, -2, -1, 0, 1, 2, 3\}$. Second, because different sparsity patterns may yield different exeuction times, we report not times from the single matrix SparsePCA produces for a given ($d, \lambda$) pair, but the best times from any of ten random matrices of the same size and at most the same sparsity. Finally and most importantly, we plot only the pareto frontier of (speed, quality) pairs produced for a given matrix multiply. I.e., we let SparsePCA cherrypick its best results on each individual matrix multiply.

% ------------------------------------------------
\subsection{Additional Baselines}
% ------------------------------------------------

% ------------------------------------------------
\subsection{UCR Time Series Archive}
% ------------------------------------------------

Since different datasets have different test set sizes, all results are for a standardized test set size of 1000 rows. We wanted the length to be a multiple of 32 since existing methods operate best with sizes that are either powers of two or, failing that, multiples of large powers of two.

% since having a length that is a multiple of at least 8 is the best-case scenario for most existing methods, and 320 is a ``rounder'' number than 312.

% ------------------------------------------------
\subsection{Caltech101}
% ------------------------------------------------

We only extracted full windows---i.e., never past the edge of an image. We extracted the windows in CHW order, meaning that scalars from the same color channel were placed at contiguous indices. The ``first'' images are based on filename in lexicographic order.

We used pairs of filters because using a single filter would mean timing a matrix-vector product instead of a matrix-matrix product.

so that other methods could, theoretically, amortize their preprocessing costs and obtain a speedup. In practic


\end{document}
