%!TEX output_directory = aux

\documentclass{article}  % sysml

\usepackage{booktabs}
\usepackage{hyperref}
\newcommand{\theHalgorithm}{\arabic{algorithm}}
% \usepackage[accepted]{sysml2019}
% \usepackage{sysml2019}
\usepackage{icml2020}

% \newcommand{\oursp}{\textsc{HashMul}\text{ }}
% \newcommand{\ours}{\textsc{HashMul}}
% \newcommand{\oursp}{\textsc{MADDNESS}\text{ }}
% \newcommand{\ours}{\textsc{MADDNESS}}
\newcommand{\oursp}{\textsc{Maddness}\text{ }}
\newcommand{\ours}{\textsc{Maddness}}
\newcommand{\oursHash}{\textsc{MaddnessHash}}

% \newcommand{\mine}{\textsc{Sprintz}}
% \newcommand{\minesp}{\textsc{Sprintz}\text{ }}

% \newcommand{\npapers}{\text{83 }}
% \newcommand{\rawnpapers}{83}

% \usepackage{flafter}  %

% \usepackage[sorting=ynt]{biblatex}
% \usepackage[sorting=ynt]{natbib}
% \usepackage[sort]{natbib}
% \DeclareSortingScheme{noneyear}{
%  \sort{\citeorder}
%  \sort{\field{year}}
% }

% \usepackage[sorting=ynt]{natbib}

%-------------------------------------------------------------- Includes

\usepackage{amsmath}          % basic math
\usepackage{amssymb} 			    % math symbols
% \usepackage{amsthm}           % theorems
\usepackage{textcomp}

\usepackage{float}            % make figures work
\usepackage{cite}             % citations would be nice
\usepackage{url}              % better urls; magically keeping doc from breaking due to urls in refs

% \usepackage{tabu}
\usepackage{array}            % multiline table cells; somehow
\usepackage{tabularx}         % better tables
% \usepackage[table]{xcolor}    % colored table cells  % incompatible with sigconf
\usepackage{colortbl}
\newcolumntype{Y}{>{\centering\arraybackslash}X}	% centered column type for tabularx

% stuff for piecewise functions
\usepackage{mathtools}          %loads amsmath as well
\DeclarePairedDelimiter\Floor\lfloor\rfloor
\DeclarePairedDelimiter\Ceil\lceil\rceil

\DeclareMathOperator*{\argmin}{arg\,min} % argmin
\DeclareMathOperator*{\argmax}{arg\,max} % argmax

\usepackage{pbox}   % for trick to force linebreaks in table cells

\usepackage{setspace}
% \usepackage{setspace}

\usepackage{booktabs} % acm recommended, and for pandas latex tables

%-------------------------------------------------------------- Algorithm setup

\usepackage{algorithm}
\usepackage[noend]{algpseudocode} % I think this removes trailing "end {if,for,while}"

% \algnewcommand{\LineComment}[1]{\State \(\triangleright\) #1} % left-aligned comments
% \algnewcommand{\SideComment}[1]{\(//\) #1}
\algnewcommand{\COMMENT}[2][.5\linewidth]{\leavevmode\hfill\makebox[#1][l]{//~#2}}
\algnewcommand{\LineComment}[1]{\State \(//\) #1}	% left-aligned comments
\algnewcommand\RETURN{\State \textbf{return} }


%-------------------------------------------------------------- Figures setup

% \usepackage[pdftex]{graphicx}
\usepackage{graphicx}
\usepackage[space]{grffile}   % allow spaces in file names
% declare the path(s) where your graphic files are
\graphicspath{{../figs/}}
% and their extensions so you won't have to specify these
\DeclareGraphicsExtensions{.pdf,.jpeg,.jpg,.png}


%\textfloatsep: space between last top float or first bottom float and the text (default = 20.0pt plus 2.0pt minus 4.0pt).
%\intextsep : space left on top and bottom of an in-text float (default = 12.0pt plus 2.0pt minus 2.0pt).
\setlength{\textfloatsep}{4pt}
\setlength{\intextsep}{4pt}

% \usepackage{caption}
% \usepackage[font={small,it}]{caption}
\usepackage[font={bf}]{caption}
% \setlength{\abovecaptionskip}{-2pt} % less space between captions and figures
\setlength{\abovecaptionskip}{4pt}	% less space between captions and figures
% \setlength{\belowcaptionskip}{-13pt}  % less space below captions
\setlength{\belowcaptionskip}{-10pt}	% less space below captions

\usepackage{paralist}
\setdefaultleftmargin{10pt}{10pt}{}{}{}{}

% \usepackage{changepage}
% \usepackage{tabulary}

\usepackage{outlines}
\usepackage{enumitem}
\newcommand{\ItemSpacing}{0mm}
\newcommand{\ParSpacing}{0mm}
\setenumerate[1]{itemsep={\ItemSpacing},parsep={\ParSpacing},label=\arabic*.}
% \setenumerate[2]{itemsep={\ItemSpacing},parsep={\ParSpacing},label=\arabic*.}
\setenumerate[2]{itemsep={\ItemSpacing},parsep={\ParSpacing}}

\usepackage[linewidth=1pt]{mdframed}
\mdfsetup{frametitlealignment=\center, skipabove=0, innertopmargin=1mm,
innerleftmargin=2mm, leftmargin=0mm, rightmargin=0mm}

%-------------------------------------------------------------- Miscellaneous setup

\newlength\myindent
\setlength\myindent{2em}
\newcommand\bindent{%
  \begingroup
  \setlength{\itemindent}{\myindent}
  \addtolength{\algorithmicindent}{\myindent}
}
\newcommand\eindent{\endgroup}

% remove unwanted space between paragraphs;
% it's set by the IEEE conference format, but no papers from this conference have it
% \parskip 0ex plus 0.2ex minus 0.1ex

% make vectors be bold instead of with arrows
\renewcommand{\vec}[1]{\mathbf{#1}}
% add 'mat' command to make matrices bold
\newcommand{\mat}[1]{\mathbf{#1}}

\DeclarePairedDelimiter\ceil{\lceil}{\rceil}
\DeclarePairedDelimiter\floor{\lfloor}{\rfloor}

\newtheorem{Definition}{Definition}[section]

% make *all* text 10pt
% \renewcommand{\footnotesize}{\normalsize}
% \renewcommand{\footnotesize}{\small}
% \renewcommand{\small}{\normalsize}


% optional custom title for header
% \sysmltitlerunning{Multiplying Matrices Without Multiplying}
\icmltitlerunning{Multiplying Matrices Without Multiplying}

\begin{document}

\twocolumn[
% ================================================================
% \title{Have We Actually Learned Anything about \\ Pruning Neural Networks?}
% \sysmltitle{Multiplying Matrices Without Multiplying}
\icmltitle{Multiplying Matrices Without Multiplying}
% \title{Have We Actually Learned Anything about Pruning Neural Networks?}
% ================================================================

% \sysmltitlerunning{Submission and Formatting Instructions for SysML 2019}
% \begin{document}
% \twocolumn[
% \sysmltitle{Submission and Formatting Instructions for SysML 2019}

% \author{Davis W. Blalock}
% \affiliation{
%   \institution{Computer Science and Artificial \\ Intelligence Laboratory}
%   \institution{Massachusetts Institute of Technology}
% }
% \email{dblalock@mit.edu}

% \author{Jose Javier Gonzalez Ortiz}
% \affiliation{
%   \institution{Computer Science and Artificial \\ Intelligence Laboratory}
%   \institution{Massachusetts Institute of Technology}
% }
% \email{jjgo@mit.edu}

% \author{John V. Guttag}
% \affiliation{
%   \institution{Computer Science and Artificial \\ Intelligence Laboratory}
%   \institution{Massachusetts Institute of Technology}
% }
% \email{guttag@mit.edu}

% \begin{sysmlauthorlist}
\begin{icmlauthorlist}
% \sysmlauthor{Davis Blalock}{mit}
% \sysmlauthor{John Guttag}{mit}
\icmlauthor{Davis Blalock}{mit}
\icmlauthor{Tamara Broderick?}{mit}
\icmlauthor{John Guttag}{mit}
% \end{sysmlauthorlist}
\end{icmlauthorlist}

% \sysmlaffiliation{csail}{MIT CSAIL, Cambridge, MA, USA}
% \sysmlcorrespondingauthor{Davis Blalock}{dblalock@mit.edu}
\icmlaffiliation{csail}{MIT CSAIL, Cambridge, MA, USA}
\icmlcorrespondingauthor{Davis Blalock}{dblalock@mit.edu}

% apparently these only show up in pdf metadata, not document
% \sysmlkeywords{Approximate Algorithms, Matrix Multiplication}
\icmlkeywords{Approximate Algorithms, Matrix Multiplication}

\vskip 0.3in
] % end of icml 2 columnn

\printAffiliationsAndNotice{}

% ------------------------------------------------
\begin{abstract}
% ------------------------------------------------

Multiplying matrices is among the most fundamental and most computationally demanding operations in machine learning and numerical computing. Consequently, the task of efficiently approximating matrix products has received significant attention.

We introduce an approximate matrix multiplication algorithm that substantially outperforms existing methods. Experiments using hundreds of matrices from diverse domains show that it runs far faster than current approaches for a given amount of error, often obtaining more than a $10\times$ relative speedup. In the common case that one matrix is known ahead of time, our method also has the interesting property that it requires \textit{zero} multiply-add operations. The key idea behind our approach is to replace the expensive encoding step common in vector quantization methods with a learned locality-sensitive hash function.


\end{abstract}
% ]  % end of manual twocolumn for sysml

% \maketitle

% ================================================================
\section{Introduction} \label{sec:intro}
% ================================================================


Matrix multiplication is among the most fundamental subroutines used in machine learning and scientific computing. As a result, there has been a great deal of work on implementing high-speed matrix multiplication libraries \cite{pytorch,eigen,tensorflow}, designing custom hardware to accelerate multiplication of certain classes of matrices \cite{eie,eyeriss,scnn,tpu}, scaling matrix multiplication across many machines \cite{distributedCoded, shortDot, entangledPolynomial, matmulCommunicationBounds}, and designing efficient Approximate Matrix Multiplication (AMM) algorithms under various assumptions and problem settings \cite{drineas_fast_2006,manne_fast_2014,ye_frequent_2016,mroueh_co-occuring_2016,bolt}.

We focus on the AMM task under the assumptions that the matrices are tall, relatively dense, and resident in a single machine's memory. In this setting, the primary challenge is minimization of the amount of CPU time required to approximate linear operations with a given level of fidelity.
%not reduction of disk or network usage [], efficient coordination between distributed workers [], maximization of a space-distortion tradeoff [], or reduction of asymptotic complexity []. Instead, it is minimization of the amount of CPU time required to approximate linear operations with a given level of fidelity.

This setting arises naturally in machine learning and data mining when one has a data matrix $\mat{A}$ whose rows are samples and a linear operator $\mat{B}$ one wishes to apply to these samples. $\mat{B}$ could be a linear classifier, linear regressor, or an embedding matrix, among other possibilities.

As a concrete example, consider the task of approximating a softmax classifier trained to predict image labels given embeddings derived from a neural network. Here, the rows of $\A$ are the embeddings for each image, and the columns of $\B$ are the weight vectors for each class. Classification is performed by computing the product $\A\B$ and taking the argmax within each row of the result.
In Figure~\ref{fig:fig1}, we see the results of approximating $\A\B$ using our method and its best-performing rivals \cite{hashjl, sparsePCA} on the CIFAR-10 and CIFAR-100 datasets.
\vspace{1mm}
\begin{figure}[h]
\begin{center}
\includegraphics[width=\linewidth]{amm/fig1}
\caption{Our method achieves a dramatically better speed-accuracy tradeoff than existing methods when approximating two linear classifiers.}
\label{fig:fig1}
\end{center}
\end{figure}
\vspace{-1mm}
% Each method yields a curve whose points are specific speedups and associated levels of classification accuracy. More speedup and more accuracy (up and to the right) is better. While we defer detailed discussion to Section~\ref{sec:results}, it is clear that our proposed approach significantly outperforms alternatives.

% In addition to achieving strong empirical performance, our method also has an

% In addition to achieving strong empirical performance,
Our method represents a significant methodological departure from most traditional approaches to this problem. Traditional AMM methods construct matrices $\V_A, \V_B \in \R^{D \times d}, d \ll D$ such that
\begin{align}
    \A \B \approx (\A \V_A) (\V_B^\top \B).
\end{align}
Often, $\V_A$ and $\V_B$ are sparse, embody some sort of sampling scheme, or have other structure such that these projection operations are faster than a dense matrix multiply. In short, these methods use linear functions to preprocess $\A$ and $\B$ and reduce the problem to exact matrix multiplication in a lower-dimensional space.
% reduce the AMM problem to exact matrix multiplication in a lower-dimensional space using linear functions to preprocess $\A$ and $\B$.

Our proposed method, \ours\footnote{Multiply-ADDitioN-lESS}, instead employs a \textit{nonlinear} preprocessing function and reduces the problem to table lookups. Moreover, in the case that $\B$ is known ahead of time---which happens when applying a trained linear model to new data, among other situations---\oursp does not require any multiply-add operations.
% This includes binary \texttt{XNOR} and \texttt{popcount} operations \cite{xnornet,dorefanet}, as well as the summation of shifted and masked scalars to approximate multiplication \cite{hackyQuasiMultiplies}. I.e., we are not merely approximating scalar multiplies at the bit manipulation level, but approximating matrix products at the algorithm level.

% An extension of this method common in the information retrieval literature is to define a nonlinear function $g(\cdot)$ such that
% \begin{align}
%     \A \B \approx g(\A \V_A) (\V_B^\top \B).
% \end{align}
% The $g(\cdot)$ function typically binarizes and induces structured sparsity in the resulting matrix. This allows the product of this matrix and $(\V_B^\top \B)$ to be computed using only additions, instead of multiply-adds. Moreover, if the sparsity is structured appropriately, dot products can be reduced to table lookups, with the indices into the tables given by the indices of the nonzeros.




% Our proposed method, \ours, instead preprocesses $\A$ and $\B$ with \textit{nonlinear} functions and reduce the problem to table lookups.


% These methods may also attempt to reduce the cost of each multiply-add by reducing the number of bits used to store each scalar. This works b



% Existing approaches to this task attempt to reduce the CPU time by performing fewer multiply-add operations or by reducing the cost of each such operation. The former typically entails projecting $\A$ and $\B$ into a lower dimensional space before multiplying them, and the latter typically entails reducing the number of bits used to store each scalar element.
% % This typically entails constructing a matrix $\V \in \R^{D \times d}, d < D$ such that
% % \begin{align}
% %     \A \B \approx (\A \V) (\V^\top \B).
% % \end{align}
% % Often, $\V$ is sparse or has structure such that these projection operations are faster than a dense matrix multiply. These methods may also attempt to reduce the cost of each multiply-add by reducing the number of bits used to store each scalar.

% % Instead of performing dimensionality or bitwidth reduction,
% In contrast, we introduce \ours, a method that can eliminate the multiply-add operations entirely when given time to preprocess $\B$. It does this by precomputing certain statistics about $\B$ and replacing the multiply-adds with table lookups. Crucially, these table lookups are not merely a different implementation of multiplication---as we discuss in section~\ref{sec:method}, they implement nonlinear functions. % that cannot be characterized as an algebraic ring.

Our method is most closely related to vector quantization methods used for similarity search (e.g., \cite{bolt,quickAdc,quickerAdc,pq,opq}). However, instead of using an expensive quantization function that requires many multiply-adds, we introduce a family of quantization functions that require no multiply-adds. %, including binary \texttt{XNOR} and \texttt{popcount} operations (though excluding implementation-level multiplies to compute memory addresses).
% TODO something about multiplication taking a lot of transistors vs table lookups.

Our contributions can be summarized as follows:
\vspace{-3mm}
\begin{itemize}\itemsep-1mm
    \item An efficient family of vector quantization functions that can encode over 100GB of data per second in a single CPU thread.
    % \item A provably good procedure for learning one of these functions from training data, in addition to other theoretical analysis.
    \item A high-speed summation algorithm for low-bitwidth integers that avoids upcasting, saturation, and overflow.
    \item An algorithm based on these functions for approximate matrix multiplication. Experiments across hundreds of diverse matrices demonstrate that this algorithm significantly outperforms existing alternatives.
\end{itemize}
\vspace{-3mm}

% ------------------------------------------------
\subsection{Problem Formulation} \label{sec:problemStatement}
% ------------------------------------------------

Let $\A \in \R^{N \times D}$ and $\B \in \R^{D \times M}$ be two matrices, with $N \gg D, M$, and $M$ not significantly larger than $D$. Given a computation time budget $\tau$, our task is to
construct three functions $g(\cdot)$, $h(\cdot)$, and $f(\cdot)$, along with constants $\alpha$ and $\beta$, such that
\begin{align} \label{eq:objective}
    \norm{\alpha f(g(\A), h(\B)) + \mat{\beta} - \A\B}_F < \eps(\tau) \norm{\A\B}_F
\end{align}
for the smallest error $\eps(\tau)$ possible. The constants $\alpha$ and $\beta$ are separated from $f(\cdot,\cdot)$ so that $f(\cdot,\cdot)$ can produce low-bitwidth outputs (e.g., in the range $[0, 255]$) even when the entries of $\A\B$ do not fall in this range.

% it produce low-bitwidth outputs

% The scalar $\alpha$ and bias matrix $\mat{\beta}$ must be known constants. These constants are separated from $f(\cdot,\cdot)$ so that it produce low-bitwidth outputs.

% , and are separated from $f(\cdot)$ so that it can produce low-bitwidth outputs that may not capture the range of values in $\A\B$. %Furthermore, $\alpha$ must be a power of $2$ so that multiplication can be replaced with bit shifting.\footnote{}
% to allow $f(\cdot)$ to

% . Moreover, the computation time $\tau(\eps)$ to run these three functions should be much smaller than the comuputation time $\tau_0$ to compute $\A\B$ directly.

We assume the existence of a training set $\tilde{\A}$, whose rows are drawn from the same distribution as the rows of $\A$. This is a natural assumption in the case that rows of $\A$ represent examples in training data, or structured subsets thereof (such as patches of images). This assumption is common in the information retrieval literature \cite{bolt,pairq,quip}, but is a significant departure from most theoretical work.

% We also assume that the cost of computing $h(\B)$ negligible provided that $h(\B) \in O(MD)$. One sufficient condition for this is $\B$ being provided ahead of time (as is the case when $\B$ represents fixed model weights). A second is that $\B$ has sufficiently few columns compared to the number of rows in $\A$ (i.e., $M \ll N$).


% ================================================================

% These methods transform the two matrices such that one becomes a collection of lookup tables and the other indices into those tables. Once transformed, the matrix multiplication \textit{per se} requires only table lookups. We adopt this same pattern, but replace the multiply-intensive transform with an efficient alternative. This replacement is enabled through our introduction of a fast and trainable family of quantization functions.

%  which already compute matrix products \textit{per se} with no multiplies. However, they require many multiplies in

% our key insight is that the quantization codebooks used by these methods


% The basic insight behind our method is that vector quantization can be carried

% In machine learning specifically, it is often the bottleneck in deep neural networks, which have emerged as an indispensible class of models for many tasks.



% multiplications [] or reduces the cost of each multiplication [], with the most extreme case of the latter being multiplication of binary values [].

% Most existing approaches to this task project $\mat{A}$ and $\mat{B}$ into a lower-dimensional space and then perform an exact matrix multiplication in this reduced space []. Concretely

% Most existing approaches to this task project $\mat{A}$ and $\mat{B}$

 % They may also reduce the number of bits used to store elements of $\A$ and $\B$ or


% We formalize these notions in the following section, but this corresponds to the common scenario in machine learning wherein one has a matrix whose

% Tall and (relatively) dense arise naturally in machine learning,
% real-world assumptions, such as matrices
% We focus on approximate multiplication of tall, dense matrices that already reside in memory.
% More specifically, consider two matrices $\mat{A} \in \R^{N \times D}$ and $\mat{B} \in \R^{D \times M}$, and their product $\mat{C} \triangleq \mat{A}\mat{B}$. We consider the setting in which $N \gg D, M$, and $M$ is not significantly larger than $D$. Furthermore, we focus on the typical real-world case in which $N$, $D$, and $M$ are all small enough that asymptotic complexity is not an adequate characterization of speed.
% Such matrices arise naturally in machine learning, when $A$ is a matrix of data and $B$ is a matrix of model weights or some other sort of linear projection. % It is also common in this setting to have a matrix of training data $\mat{\tilde{A}}$ whose rows share the same distribution as those of $\mat{A}$, and we assume access to such a matrix.

% By tall, we are referring to the typical real-world case in which asymptotic complexity is a poor definition of speed.

% In contrast to most existing AMM work, we also consider various levels of prior knowledge about the distributions from which the matrices are drawn. In the least informed case, nothing is known about either matrix until the instant they are provided. In the most informed case, one matrix is known ahead of time and the other has rows drawn from a distribution from which we have numerous i.i.d. samples.

% Provided that one matrix is known ahead of time, our method requires zero multiplication operations, aside from possible pointer arithmetic at the implementation level.
%This is in contrast to existing work, which either reduces the number of multiplications [] or reduces the cost of each multiplication [], with the most extreme case of the latter being multiplication of binary values [].


% ================================================================
% \section{Background and Related Work}
\section{Related Work} \label{sec:relatedWork}
% ================================================================


Because our approach draws on ideas from randomized algorithms, approximate matrix multiplication, vector quantization, and other fields, the body of work related to our own is vast.
Here, we provide only a high-level overview, and refer the interested reader to
% \cite{learningToHashSurvey, hashingSimilaritySurvey, isvd} for more detailed surveys.
surveys of this area \cite{learningToHashSurvey, hashingSimilaritySurvey, isvd} for more information.
We also defer discussion of related vector quantization methods to the following sections, since it is easier to appreciate how they differ from our own method once our method has been introduced.

\vspace{-1mm}
\subsection{Linear Approximation}
% \vspace{-.5mm}
Most AMM methods work by projecting $\A$ and $\B$ into lower-dimensional spaces and then performing an exact matrix multiply.
%  $\V_A, \V_B \in \R^{D \times d}, d \ll D$ such that
% % \begin{align}
% %     \A \B \approx (\A \V_A) (\V_B^\top \B).
% % \end{align}
% There are many options for choosing $\V_A$ and $\V_B$
%, depending on what assumptions one makes. The most common case studied is that in which $N$, $D$, and $M$ are large, but one has little or no prior knowledge about either matrix.
% One simple approach is to choose them such that $(\A \V_A)$ and $(\V_B^\top \B)$ preserve certain structure in $\A$ and $\B$---i.e. to sketch each matrix independently.
One simple option for choosing the projection matrices is to use matrix sketching algorithms. The most prominent deterministic matrix sketching methods are the Frequent Directions algorithm \cite{liberty_simple_2012, ghashami_frequent_2016} and its many variations \cite{teng_fast_2019, francis_practical_2018, ye_frequent_2016, huang_near_2019, luo_robust_2019, francis_improvement_2018}. There are also many randomized sketching methods \cite{sarlos_improved_2006, kyrillidis_approximate_2014, pagh_compressed_2013, hashjl,osnap} and sampling methods \cite{drineas_fast_2006-1, drineas_fast_2006-2}.

A weakness of matrix sketching methods in the context of matrix multiplication is that they consider each matrix in isolation. To exploit information about both matrices simultaneously, \citet{drineas_fast_2006} sample columns of $\A$ and rows of $\B$ according to a sampling distribution dependent upon both matrices. Later work by \citet{manne_fast_2014} reduces approximation of the matrices to an optimization problem, which is solved by steepest descent. \citet{mroueh_co-occuring_2016}, \citet{ye_frequent_2016}, and \citet{francis_improvement_2018} introduce variations of the Frequent Directions algorithm that take into account both matrices.

All of the above methods differ from our own not merely in specifics, but also in problem formulation. These methods all assume that there is no training set $\tilde{\A}$ and nearly all focus on large matrices, where provably reduced asymptotic complexity for a given level of error is the goal. % We assume smaller matrices and the presence of a training set.

\vspace{-1mm}
\subsection{Hashing to Avoid Linear Operations}
% \vspace{-.5mm}
In the neural network acceleration literature, there have been several efforts to accelerate dense linear layers using some form of hashing \cite{springScalable,slide,wtaSoftmax,googleWtaCvpr,hashnet}. These methods differ from our own in the hash functions chosen, in not exploiting a training set, and in the overall goal of the algorithm. While we seek to approximate the entire output matrix, these methods seek to either sample outputs \cite{springScalable,slide}, approximate only the largest outputs \cite{wtaSoftmax,googleWtaCvpr}, or implement a fixed, sparse linear operator \cite{hashnet}.



% 
\subsection{Linear Approximation}

Traditional AMM methods construct matrices $\V_A, \V_B \in \R^{D \times d}, d \ll D$ such that
\begin{align}
    \A \B \approx (\A \V_A) (\V_B^\top \B).
\end{align}
There are many options for choosing $\V_A$ and $\V_B$, depending on what assumptions one makes.

The most common case studied is that in which $N$, $D$, and $M$ are large, but one has little or no prior knowledge about either matrix. One simple approach is to sketch each matrix individually. Most of these methods center on one of three basic approaches: deterministic sketching, random sketching, or random sampling. Deterministic sketching methods employ a fixed algorithm or projection matrix to obtain a low-rank approximation to the original matrix. The most prominent of these methods are the Frequent Directions algorithm \cite{liberty_simple_2012, ghashami_frequent_2016} and its many variations \cite{teng_fast_2019, francis_practical_2018, ye_frequent_2016, huang_near_2019, luo_robust_2019, francis_improvement_2018}. Random sketching methods typically design matrices with certain properties or structure, such as having entries composed of or multiplied by i.i.d. Rademacher random variables \cite{sarlos_improved_2006, kyrillidis_approximate_2014, pagh_compressed_2013} or having a particular sparse structure \cite{hashjl,osnap}. Random sampling methods choose subsets of rows and/or columns, typically in accordance with some theoretically motivated sampling distribution \cite{drineas_fast_2006-1, drineas_fast_2006-2}.

One weakness of the above approaches is that they consider each matrix individually, instead of the two of them jointly. The first authors to formally address AMM by considering both matrices were \citet{drineas_fast_2006}, who sampled columns of $\A$ and rows of $\B$ according to a sampling distribution dependent upon both matrices. Later work by \citet{manne_fast_2014} reduces approximation of the matrices to an optimization problem, which is solved by steepest descent. \citet{mroueh_co-occuring_2016}, \citet{ye_frequent_2016}, and \citet{francis_improvement_2018} introduce variations of the Frequent Directions algorithm that take into account both matrices simultaneously.

% In the information retrieval community, approximate matrix multiplication is almost exclusively studied in the regime where $N, M \gg D$ and either $\A$ or $\B$ is either fixed ahead of time or changing slowly.
% Another relevant body of methods comes from the information retrieval literature.

\subsection{Vector Quantization}

While they are not traditionally considered related work for AMM methods, there are also a number of relevant approaches from the information retrieval literature. These approaches consider the tasks of nearest neighbor and maximum inner product search, not approximate matrix multiplication \textit{per se}; however, these methods either
\begin{enumerate}
    \item work by reducing the problems to approximate matrix multiplication \cite{cq, aq, otq, scq, stackedQuantizers, bolt}, or
    \item could easily be applied to approximate matrix multiplication by replacing squared euclidean distance computations with dot product computations. \cite{pq, quickAdc, quickerAdc, pairq, grvq, opq, polysemous}
\end{enumerate}
Consequently, it is useful to cast them as AMM methods.

These methods almost exclusively consider the regime where $N, M \gg D$ and either $\A$ or $\B$ is fixed ahead of time (or at least changing slowly). These papers extend the traditional AMM formulation to include a nonlinear vector quantization function $g(\cdot)$ such that
\begin{align}
    \A \B \approx g(\A \V_A) (\V_B^\top \B).
\end{align}
The $g(\cdot)$ function binarizes and induces structured sparsity in the resulting matrix. Because all of the resulting coefficients of the matrix $g(\A \V_A)$ are either 0 or 1, this allows its product with $(\V_B^\top \B)$ to be computed using only additions, instead of multiply-adds. Moreover, if the sparsity is structured appropriately, dot products can be reduced to table lookups, with the indices into the tables given by the indices of the nonzeros. As several recent papers have observed \cite{bolt, quickAdc, quickerAdc}, these tables can be restricted in size and quantized such that the table lookups can be executed with CPU vector instructions.

We discuss the most relevant of these vector quantization-based approaches in more detail in Section~\ref{sec:background}, and refer the reader to \cite{learningToHashSurvey, hashingSimilaritySurvey} for detailed surveys of this area. % For now, we merely note that most of the AMM-relevant effort in the past decade has been devoted to finding better $\V_A$ and $\V_B$ matrices.
% Because of the assumption that $N, M \gg D$, applying these linear transforms is nearly free

Our method is also built on vector quantization, but instead of applying $g(\cdot)$ to a transformed matrix $\A \V_A$, we directly transform $\A$ itself. Because we use a $g(\cdot)$ function that does not require any multiply-adds, our overall procedure requires multiply-adds only if $\V_B^\top \B$ cannot be precomputed. We also depart from existing vector-quantization based AMM through the addition of several more algorithmic enhancements, but defer discussion of these to Section~\ref{sec:method}.


\subsection{Hashing to Avoid Linear Operators}

TODO talk about Ryan Spring's DNNs + Hashing stuff somewhere \cite{springScalable,slide}. Also other hashing methods to avoid computing most of the outputs \cite{wtaSoftmax,googleWtaCvpr}. Also talk about idea of using LSH for clustering being an existing thing in the differential privacy literature \cite{kmeansLshDp2017, kmeansLshDp2018, kmeansLshDp2020}.

% There have been variations of this approach (see \cite{learningToHashSurvey, hashingSimilaritySurvey} for detailed surveys), with much of the effort aimed at developing better $\V_A$ and $\V_B$ matrices.





%================================================================
\vspace{-1.5mm}
\section{Background - Product Quantization} \label{sec:background}
%================================================================


% Recall that our task is to construct functions $g(\cdot)$, $h(\cdot)$, and $f(\cdot)$ such that
% \begin{align}
%     \norm{f(g(\A), h(\B)) - \A\B}_F < \eps(\tau) \norm{\A\B}_F
% \end{align}
% for the smallest $\eps(\tau)$ possible.

% In this section, we discuss the vector quantization (VQ) approach to approximate matrix multiplication. Because these methods approximate matrix products by approximating individual dot products between rows of $\A$ and columns of $\B$, we do so by examining how they approximate a single dot product $\a^\top \b$ between one row of $\A$ and one column of $\B$. For simplicity, we will also begin by focusing on Product Quantization (PQ) \cite{pq}, the classic algorithm on which most others are based.

To lay the groundwork for our own method, we begin by reviewing Product Quantization (PQ) \cite{pq}. PQ is a classic vector quantization algorithm for approximating inner products and Euclidean distances and serves as the basis for nearly all vector quantization methods similar to our own. % This not only lays necessary groundwork for explaining our own method, but also affords the opportunity to discuss

% to both lay the groundwork  the classic vector quantization algorithm for approximating inner products on which most others are based.

The basic intuition behind PQ is that $\a^\top \b \approx \hat{\a}^\top \b$, where $\norm{\hat{\a} - \a}$ is small but $\hat{\a}$ has special structure allowing the product to be computed quickly. This structure consists of $\hat{\a}$ being formed by concatenating learned prototypes in disjoint subspaces; one obtains a speedup by precomputing the dot products between $\b$ and the prototypes once, and then reusing these values across many $\a$ vectors. The $\a$ vectors here are the (transposed) rows of $\A$ and the $\b$ vectors are the columns of $\B$.

 % \cite{pq}, a classic approach to this problem that serves as a conceptual foundation for our own method. There are three basic steps behind PQ:
In somewhat more detail, PQ consists of the following:
\vspace{-3mm}
% \begin{enumerate}\itemsep.5mm
\begin{enumerate}\itemsep1.5mm
    % \item asdf
    % \item $g(\A)$
    % \item Replacing each row of $\A$ with a prototype chosen from a predefined set.%
    \item \textbf{Prototype Learning} - In an initial, offline training phase,  cluster the rows of $\A$ (or a training set $\tilde{\A}$) using K-means to create prototypes. A separate K-means is run in each of $C$ disjoint subspaces to produce $C$ sets of $K$ prototypes. %The prototypes consist of the cartesian product of prototypes within disjoint subspaces.
    % \item $g(\A)$ - Replace each row of $\A$ with the most similar prototype.
    % \item $h(\B)$ - Precompute the dot products between each column of $\B$ and each prototype.
    \item \textbf{Encoding Function}, $g(\a)$ - Determine the most similar prototype to $\a$ in each subspace. Store these assignments as integer indices using $C \log_2(K)$ bits.
    \item \textbf{Table Construction}, $h(\B)$ - Precompute the dot products between $\b$ and each prototype in each subspace. Store these partial dot products in $C$ lookup tables of size $K$.
    \item \textbf{Aggregation}, $f(\cdot,\cdot)$ - Use the indices and tables to \textit{lookup} the estimated partial $\a^\top \b$ in each subspace, then sum the results across all $C$ subspaces. %$A[i]^\top B[:, j]$\footnote{Because we will frequently need to refer to slices of rank-3 tensors and above, we employ Python-style slicing notation rather than traditional superscripts and subscripts.} rather than compute it with a series of multiply-adds.
\end{enumerate}
\vspace{-3mm}
PQ is depicted for a single pair of vectors $\a$ and $\b$ in Figure~\ref{fig:pq}. We elaborate upon each of these steps below.
% \vspace{-2mm}

% ------------------------------------------------
% \subsection{Prototype Learning}
\vspace{-2mm}
\paragraph{Prototype Learning:}
% ------------------------------------------------

Let $\tilde{A} \in \R^{N \times D}$ be a training set, $K$ be a number of prototypes per subspace, $C$ be a number of subspaces, and $\{\mathcal{J}^{(c)}\}_{c=1}^C$ be the mutually exclusive and collectively exhaustive sets of indices associated with each subspace. The training-time task of PQ is to learn $C$ sets of prototypes $\mat{P}^{(c)} \in \R^{K \times |\mathcal{J}^{(c)}|}$ and assignments $\vec{z}^{(c)} \in \R^{N}$ such that:
\vspace{-2mm}
\begin{align}
    \sum_{i=1}^N \sum_{c=1}^C \sum_{j\in \mathcal{J}^{(c)}} \left( \tilde{\A}_{ij} - \mat{P}^{(c)}_{z^{(c)},j} \right)^2
\vspace*{-.5mm}
\end{align}
is minimized. It does this by running K-means separately in each subspace $\mathcal{J}^{(c)}$ and using the resulting centroids and assignments to populate $\mat{P}^{(c)}$ and $\vec{z}^{(c)}$.

\begin{figure}[t]
\begin{center}
\includegraphics[width=\linewidth]{amm/pq}
\caption{Product Quantization. The $g(\cdot)$ function returns the index of the most similar prototype to the data vector a in each subspace. The $h(\cdot)$ function computes a lookup table of dot products between the query vector b and each protoype in each subspace. The aggregation function $f(\cdot,\cdot)$ sums the table entries corresponding to each index.}
% $\vec{a}$
\label{fig:pq}
\vspace*{2mm}
\end{center}
\end{figure}


% Perhaps the simplest means of constructing prototypes would be to run a clustering algorithm like K-means on the rows of $\A$. This would be effective, but would require using a large number of prototypes in order to have each row closely match its prototype. What would be preferable is some means of having a huge number of prototypes without having to pay so large a time and space cost.

% PQ achieves this goal by using as prototypes the cartesian product of prototypes within disjoint subspaces. Concretely, PQ runs K-means with $K$ centroids in each of $C$ disjoint (usually contiguous) sets of dimensions. This results in $K^C$ possible combinations of centroid assignments for each vector, with each unique combination corresponding to a unique overall prototype.

% vectors into disjoint ``subvectors'' (each corresponding to a unique set of dimensions) and runs k-means within each subspace.
%  If there are $K$ prototypes per subspace and $C$ subspaces, this results in $K^C$ possible combinations of prototype assignments for each vector. Viewed differently, this creates $K^C$ overall prototypes whose entries

% ------------------------------------------------
% \subsection{Encoding Function - $g(\A)$}
% \vspace{-2.5mm}
% \vspace{144pt}
% \vspace{288pt}
\paragraph{Encoding Function, $\bm{g(\A)}$:}
% ------------------------------------------------

Given the learned prototypes, PQ replaces each row $\a$ of $\A$ with the concatenation of its $C$ K-means centroid assignments in each of the $C$ subspaces. Formally:
\vspace{-.5mm}
\begin{align} \label{eq:pqLoss}
    g^{(c)}(\a) \triangleq \argmin_k \sum_{j\in \mathcal{J}^{(c)}} \left( \a_j - \mat{P}^{(c)}_{k,j} \right)^2.
\vspace*{-.5mm}
\end{align}
We will refer to the resulting sequence of indices as the \textit{encoding} of $\a$ and the set of $K$ centroids as a \textit{codebook}. For convenience, we will also refer to the vector $\a^{(c)} \triangleq \langle a_j \rangle, j \in \mathcal{J}^{(c)}$ as the \textit{subvector} of $\a$ in subspace $c$. %This encoding function has complexity $\Theta(KD)$ for each $\a$, since there are $K$ centroids per subspace and the sum of the number of indices in all subspaces is $D$.

% ------------------------------------------------
% \subsection{Table Construction - $h(\B)$}
% \vspace{-2.5mm}
\paragraph{Table Construction, $\bm{h(\B)}$:}
% ------------------------------------------------

Using these same prototypes, PQ constructs a lookup table $h^{(c)}(\b) \in \R^K$ in each of the $C$ subspaces for each column $\b$ of $\B$, where
% \vspace{-.5mm}
\begin{align}
    h^{(c)}(\b)_k \triangleq \sum_{j\in \mathcal{J}^{(c)}} \b_j \mat{P}^{(c)}_{k,j}.
\end{align}
Existing work has shown that setting $K = 16$ and quantizing the lookup tables to 8 bits can offer enormous speedups compared to larger $K$ and/or floating-point tables \cite{bolt, quickAdc, quickerAdc}. This is because 16 1-byte entries can be stored in a SIMD register, allowing 16 or more table lookups to be performed in parallel using a byte shuffle instruction. Since the table entries naturally occupy more than 8 bits even for 8-bit data, some means of quantizing these entries is necessary. This can easily be done by subtracting off the minimum entry in each table and linearly rescaling such that the maximum entry in any table is at most 255. Ignoring rounding error, this affine transform is invertible, and is reflected by the constants $\alpha$ and $\mat{\beta}$ in equation~\ref{eq:objective}. See Appendix~\ref{sec:lutQuantize} for further details.

% \vspace{-.5mm}
%This has complexity $\Theta(KD)$ for each $\b$, since it is essentially the same as $g(\cdot)$ except without the $\argmin$ operation.

% ------------------------------------------------
% \subsection{Aggregation - $f(\cdot,\cdot)$}
% \vspace{-2mm}
\paragraph{Aggregation, $\bm{f(\cdot,\cdot)}$:}
% ------------------------------------------------

Given the encoding of $\a$ and the lookup tables for $\b$, the product can be approximated as
% \vspace*{-2mm}
\begin{align}
    \a^\top \b = \sum_{c=1}^C \a^{(c)\top} \b^{(c)} \approx \sum_{c=1}^C h^{(c)}(\b)_k, \text{ }k = g^{(c)}(\a).
\end{align}
% \vspace{-.5mm}

%This has complexity $\Theta(C)$ for each pair of vectors, or $\Theta(NMC)$ for the full matrix product $\A\B$. Since $C \ll D$, this provides a large speedup when $N$ and $M$ are large enough to amortize the costs of $g(\cdot)$ and $h(\cdot)$.

% % ------------------------------------------------
% \subsection{Complexity and Extensions}
% % ------------------------------------------------

% The total time complexity of PQ is equal to $\Theta(NDK + DKC + NMC)$. When $N$ and $M$ are large, the last term dominates and the complexity of PQ is essentially $\Theta(NMC)$.
% Assuming one does not use Strassen's or other algorithms that are never used in practice in modern numerical libraries, full matrix multiplication has complexity $\Theta(NDM)$. Thus, PQ offers a speedup of $\Theta(\frac{D}{C})$.

% To increase this ratio, many authors have attempted to construct more elaborate $g(\cdot)$ and $h(\cdot)$ functions. One line of work seeks to preprocess $\a$ and/or $\b$ with rotation matrices \cite{opq,cartesianKmeans,lopq}. Another line of work seeks to relax the restriction that the codebooks use disjoint subspaces; several authors \cite{aq,cq,otq,sq,grvq,stackedQuantizers} have observed that equation~\ref{eq:pqLoss} is a special case of
% \begin{align} \label{eq:nonOrthogonal}
%     \sum_{i=1}^N min_{\{z^{(1)},\ldots,z^{(C)}\}} \sum_{j=1}^D \left( \tilde{A}_{ij} - \sum_{c=1}^C \mat{P}^{(c)}_{z^{(c)},j} \right) ^2.
% \end{align}
% That is, all $C$ codebooks can contribute to approximating any dimension of $\a$, instead of only one codebook being responsible for it. This is strictly more expressive, but also slows down $g(\cdot)$ greatly and complicates training. There has been effort to reduce these burdens by carefully ordering update and assignment operations \cite{stackedQuantizers}, adding sparsity \cite{sq}, or structuring overlap between codebooks so that optimal assignments remain possible \cite{otq}. TODO sentence about optimizing LUTs directly.


%================================================================
\vspace{-7mm}
\section{Our Method} \label{sec:method}
\vspace{-1mm}
%================================================================


As mentioned in the problem statement, our goal is to construct a distance function $\hat{d}$ and two encoding functions $g$ and $h$ such that $\hat{d}(g(\vec{q}), h(\vec{x})) \approx d(\vec{q}, \vec{x})$ for some ``true'' distance function $d$. To explain how we do this, we first begin with a review of Product Quantization [], and then describe how our method differs.

\subsection{Background: Product Quantization}

Recall that, by assumption, the function $d$ can be written as:
\begin{align*}
        d(\vec{q}, \vec{x}) = f(\sum_{j=1}^J d_j(q_j, x_j))
\end{align*}
where $f: \mathbb{R} \rightarrow \mathbb{R}$, $d_j: \mathbb{R}^J \times \mathbb{R}^J \rightarrow \mathbb{R}$. Now, suppose one has a partition $\mathcal{P} = \{p_1,\ldots,p_M \}$ of the indices $j$, such that $i \ne j \implies p_i \cap p_j = \emptyset$ and $\bigcup_m p_m = \{1,\ldots,D\}$. The argument to $f$ can then be written as:
\begin{align}
        \sum_{m=1}^M \sum_{j \in p_m} d_j(q_j, x_j)
            = \sum_{m=1}^M d_m(\vec{q}_m, \vec{x}_m)
\end{align}
where $\vec{q}_m$ and $\vec{x}_m$ are the vectors formed by gathering the indices of $\vec{q}$ and $\vec{x}$ at the indices $j \in p_m$, and $d_m$ sums the relevant $d_j$ functions. The idea of Product Quantization (PQ) is to replace each $x_m$ with one vector $\vec{c}_{m}$ from a \textit{codebook} set $\mathcal{C}_m$ of possibilities. That is: % (\vec{q_m}, \vec{x}) = \sum_{j = 1}^|p_m d_j(q_j, x_j)$.
\begin{align} \label{eq:pqDistNoLut}
        \sum_{m=1}^M d_m(\vec{q}_m, \vec{x}_m) \approx \sum_{m=1}^M d_m(\vec{q}_m, \vec{c}_{m})
\end{align}
This allows one to store only the identity of the codebook vector chosen, instead of the elements of the original vector $\vec{x}_m$. More formally, the PQ encoding function $h(\vec{x})$ for a given set of codebooks $\mathcal{C} = \{\mathcal{C}_1,\ldots,\mathcal{C}_C\}, \mathcal{C}_m = \{\vec{c}_{1m},\ldots,\vec{c}_{Cm}\}$ is:
\begin{align}
    h(\vec{x}) = \langle i_1,\ldots,i_M \rangle,  i_m = \argmin_i d(\vec{c}_{im}, \vec{x_m})
\end{align}

Using a fixed set of codebooks also enables construction of a fast query encoding $g$ and distance approximation $\hat{d}$. Specifically, let the query encoding space $\mathcal{G}$ be $R^{M \times C}$ and define $\mat{D} = g(\vec{q})$ as: % $g: \mathbb{R}^J \rightarrow R^{M \times C}$ as:
\begin{align}
    \mat{D}_{im} \triangleq d_m(\vec{q}_m, \vec{c}_{im})
\end{align}
Then we can rewrite the approximate distance on the right hand side of ~\ref{eq:pqDistNoLut} as:
\begin{align} \label{eq:pqDist}
        \sum_{m=1}^M \mat{D}_{im}, i = h(x)_m
\end{align}
In other words, the distance can be reduced to a sum of precomputed distances between $\vec{q_m}$ and the codebook vectors $\vec{c}_{im}$ used to approximate $\vec{x}$. Finally, by reintroducing $f$, one can now define:
\begin{align} \label{eq:pq_dhat}
    \hat{d}(g(\vec{q}), h(\vec{x})) \triangleq f(\sum_{m=1}^M \mat{D}_{im}, i = h(x)_m)
\end{align}
If $M \ll D$ and $C \ll |\mathcal{X}|$, then computation of $\hat{d}$ is much faster than computation of $d$ given the $g(\vec{q})$ matrix $\mat{D}$ and data encodings $\mathcal{H} = \{h(\vec{x}), \vec{x} \in \mathcal{X} \}$.

The total computational cost of product quantization is $\Theta(CJ)$ to encode each $\vec{x}$, $\Theta(CJ)$ to encode each query $\vec{q}$, and $\Theta(M)$ to compute the approximate distance between a given $\vec{q}$ and $\vec{x}$. Because queries must be encoded before distance computations can be performed, this means that the cost of computing the distances to the $N$ database vectors $\mathcal{X}$ when a query is received is $\Theta(CJ) + \Theta(NM)$. Lastly, since codebooks are learned using k-means clustering, the time to learn the codebook vectors is $O(CNJT)$, where $T$ is the number of k-means iterations. In all works of which we are aware, $C$ is set to $256$ so that each element of $h(\vec{x})$ can be encoded as one byte. Further, the subspaces $p_m$ are the contiguous blocks of $J/M$ dimensions, possibly after a random permutation.

In certain cases, Product Quantization is nearly an optimal encoding scheme. Specifically, under the assumptions that 1) $\vec{x}_m \sim MVN(\vec{\mu}_m, \Sigma_m)$, 2) $Pr[\vec{x}] = \prod_m Pr[\vec{x}_m]$; and 3) $\forall_m |\Sigma_m| = \textit{const}$, PQ achieves the information-theoretic lower bound on code length for a given quantization error [OPQ]. In words, this means that PQ encoding is optimal if $x$ is drawn from a multivariate gaussian and the subspaces $p_m$ are independent and have the same variance.

In practice, however, most datasets are not Gaussian and their subspaces are neither independent nor homoscedastic. Consequently, many works have generalized PQ to capture relationships across subspaces or decrease the dependenies between them [][][][]. One work of particular note is Optimized Product Quantization (OPQ). OPQ exploits the fact that Euclidean distances and dot products are invariant to rotatations by learning a rotation that reduces the PQ quantization error.

Talk about OPQ more here if we have time to plug in our modified version of it.

\subsection{Bolt}

Bolt is similar to Optimized Product Quantization but differs in three key ways:
\begin{enumerate}
\item It constrains the rotation matrix such that vectors can be rotated more quickly.
\item It uses far fewer centroids.
\item It uses an approximate query encoding distance matrix $\mat{D}$.
\end{enumerate}

Change (1) directly increases the speeds of $g$ and $h$, as well as training, while (2) and (3) work together to increase the speeds of $g$, $h$, and $\hat{d}$.

\subsubsection{Constrained Rotation Matrix}

Write this if we end up using this approach. Otherwise remove change 1. We're just gonna make the rotation matrix block diagonal so we can run OPQ and in disjoint subspaces and apply several small rotations instead of one big, slow one.

\subsubsection{Fewer Centroids and Approximate Distances}

Change (2), reducing the size of codebooks $C$, directly reduces the constant factor associated with both query and data encoding. Although this is novel insofar as virtually all other papers have taken for granted the value $C = 256$ to the best of our knowledge, this is a simple change of parameter setting and we do not discuss it further.

More importantly, changes (2) and (3) together allow accelerated computation of $\hat{d}$. Specifically, if we fix $C = 16$, each of the $M$ codes in the data encoding $h(\vec{x})$ can be expressed using 4 bits. At the same time, if we quantize each entry of $\mat{D}$ to a single byte, instead of a 32-bit or 64-bit float, the rows of $\mat{D}$ can be fit in 16 bytes. This means that each $i \in h(\vec{x})$ is a 4-bit index into a 16-byte array. On virtually all personal computers, servers, CUDA-enabled GPUs, tablets, and smart phones, lookups into 16B arrays can be vectorized; i.e., some number $V$ of them (typically 16, 32, or 64) can be performed by the processor simultaneously\footnote{The relevant instructions are \texttt{vpshufb} on x86 [], \texttt{vtbl} on ARM [], \texttt{vperm} on PowerPC [], and \texttt{\_\_shfl} on CUDA [], among others on less popular architectures.}. Moreover, on CPUs, the entire column of $\mat{D}$ in question can be stored in a SIMD register, obviating the need to access L1 cache. In theory then, for $V$ simultaneous lookups, this affords over a $V$-fold speedup in the computation of $\hat{d}$. The speedup is lower in practice---see Section~\ref{sec:results}).

It is worth noting that [] used a similar vectorization strategy to accelerate PQ distance computations. However, their method requires hundreds of thousands or millions of encodings to be sorted lexicographically and stored contiguously ahead of time, as well as scanned through serially. This is untenable when the data is rapidly changing or when using an indexing structure, which would split the data into far smaller partitions. Their approach also requires a second refinement pass of non-vectorized PQ distance computations for encodings that might correspond to nearest neighbors or are otherwise of interest, since it computes only a lower bound on the PQ approximate distance. Probably move this to related work. % In practice, this theoretical speedup is not attained because of the loads, stores, and other instructions necessary to scan through the data, but a factor of $5$ or more is common (c.f. Section~\ref{sec:results}).

Unfortunately, it is not obvious how to quantize the elements of the distance matrix $\mat{D}$ in a manner that captures the full range of distances in the matrix. Or maybe it is; I don't have a formal experiment showing this, particularly when you upcast to uint16s as soon as you do the lookup. Either way, write a paragraph about how we end up doing it.

Maybe pseudocode for the query encoding and/or dist computations here? I think the latter would just add confusion because we have to use a weird partially-column-major storage layout that I'd rather not walk people through.

Worth switching from D as dimensionality to J or something? The goal is to get the distance matrix $\mat{D}$ to instead be $\mat{D}$. Also to decouple $d$ as a distance function from $D$ which has little to do with it.

% modern x86 processors [vpshufb], ARM processors [vtbl], PowerPC processors [vperm], and CUDA GPUs [__shfl], which together comprise the overwhelming majority of personal computers, servers, and smart phones, lookups into 16B arrays can be vectorized; i.e., 16 or, more commonly, 32 of them can be carried out simultaneously.

% Product quantization can yield high accuracy, but depends upon the characteristics of the data. In particular, because it quantizes subspaces independently, it cannot exploit mutual information between variables in different subspaces. When there is little or none of this information, however,



 % When there is a great deal of this information, it requires longer encodings (larger $M$) than

% with each vector in the codebook corresponding to one centroid. To facilitate exposition, we will henceforth refer to these vectors as \textit{centroids}, and the indices $i$ identifying them as \textit{codes}.

% Computing If the set $\mathcal{X}$ is fixed, then the data encodings can be precomputed.

% TODO: index in above shouldn't be i; it should be $\hat{x}_m$






% TODO put this back in when we transition to OPQ
% Moreover, under the assumptions that 1) $\vec{x}_m \sim MVN(\vec{\mu}_m, \Sigma_m)$, 2) $Pr[\vec{x}] = \prod_m Pr[\vec{x}_m]$; and 3) $\forall_m |\Sigma_m| = const$, this formulation achieves the information-theoretic lower bound on code length for a given squared error [OPQ]. In words, this means that PQ encoding is optimal if $x$ is a multivariate gaussian and the subspaces $p_m$ are independent and have the same variance.



%  1) $x \sim MVN(\mu, \Sigma)$, and therefore $x_m \sim MVN(\mu_m, \Sigma_m)$; 2) $\Sigma_{ij} = 0$ for all $i, j$ in different subspaces $p_m$; and 3) $\forall_m |\Sigma_m| = const$, where $\Sigma_m$ is the covariance within subspace $p_m$, this formulation achieves the information-theoretic lower bound on code length for a given squared error [OPQ].

% overview
%     -as mentioned in the problem statement, our goal is to construct a distance (or similarity) function $\hat{d}(q, x)$ that approximates some true distance (or similarity) function d(q, x).

%     -Further recall that $\hat{d}$ need not operate on the original spaces Q and X, but can instead use transformed spaces $G = g(Q)$ and $H = h(X)$, so that its signature is instead $\hat{d}: G x H \rightarrow R$.

%     -our method entails offline learning of the functions $G$ and $H$ in a manner similar to existing Multi-Codebook Quantization (MCQ) techniques, and online computation of the distance

%     -we first describe the function $h(.)$, which quantizes the vectors X.

%     -we then describe the functions $g(.)$, which computes sufficient statistics about the vector q

%     -we finally describe the function $\hat{d}(.,.)$, which returns a distance based on $g(q)$ and $h(x)$.

% Quantization scheme
%     -product quantization background
%     -our novel permutation method

% Distance computation
%     -asymetric distance computation background





% %================================================================
% \section{Theoretical Analysis}
% %================================================================

% 
% ------------------------------------------------
\subsection{Generalization Error of Quantization}
% ------------------------------------------------

Basically just stating results from Bachem 2017.

% ------------------------------------------------
\subsection{Concentration of Errors}
% ------------------------------------------------

Mention Bolt results.

statement of generalized azuma's and generalized mcdiarmid's (probably based on lipshitz continuity). Defer proofs to appendix.

Apply this to errs from different subspaces, and some kind of numerical experiments to show that it's reasonable (and maybe even that independence is reasonable).




%================================================================
\section{Experiments} \label{sec:results}
%================================================================


To assess \ours's effectiveness in practice, we implemented both it and existing algorithms in C++ and Python. All of our code and raw numerical results are publicly available, along with additional experiments omitted from this section due to space constraints.\footnote{https://smarturl.it/\ours}. In order to avoid implementation bias, we built upon the source code provided by \citep{bolt}\footnote{https://github.com/dblalock/bolt}, which has the most in common with our method and includes highly tuned implementations of many algorithms to which we compare. All experiments use a single thread on a 2013 Macbook Pro with a 2.6GHz Intel Core i7-4960HQ processor. Unless stated otherwise, all timing results use five trials, with each trial reporting the lowest time it took the code to execute in 20 runs. We use the best, rather than average, since this is standard practice in performance benchmarking and is robust to the purely additive noise introduced by competing CPU tasks.

Because nearly all existing work on approximate matrix multiplication either focuses on special cases that do not satisfy our problem definition \cite{quickerAdc, pq, opq} or synthetic matrices, there is not a clear set of benchmark matrix multiply tasks to use. We therefore propose a collection of tasks that we believe are both reproducible and representative of many real-world matrices. To the best of our knowledge, our experiments use over an order of magnitude more matrices than any previous study.% These tasks include multiplication of hundreds of real-world matrices, which is, to the best of our knowledge, orders of magnitude more than have been used in any previous study.

% ------------------------------------------------
\subsection{Methods Tested}
% ------------------------------------------------
Recall that most baselines take the form of selecting a matrix $\V \in \R^{D \times d}, d < D$ such that $\A \B \approx (\A \V) (\V^\top \B)$. Here $d$ is a free parameter that adjusts the quality vs speed tradeoff. We therefore characterize these methods by how they set $\V$.

\begin{itemize}\itemsep0em
    \item PCA. Set $\V$ equal to the top principal components of the training matrix $\tilde{\A}$.
    \item SparsePCA \cite{sparsePCA}. Set $\V = \argmin_{\V} \min_{\mat{U}} \frac{1}{N_{train}} \frac{1}{2} \norm{ \tilde{\A} - \mat{U}\mat{V} }^2_F + \lambda \norm{\V}_1$. This is not the only dictionary learning formulation referred to as SparsePCA \cite{spcaSurvey1,spcaSurvey2}, but it is reasonably representative and is the only one with support in any major Python machine learning library.%\footnote{https://scikit-learn.org/stable/modules/generated/sklearn.decomposition.SparsePCA.html}.

    Because SparsePCA requires the tuning of both $d$ and $\lambda$ for each matrix, we allowed it tune these parameters \textit{on the test set} to ensure that insufficient hyperparameter tuning did not hamper its performance. See Appendix~\ref{sec:experimentDetails} for more information.

    % First, for any given $d$ value and level of sparsity, we report
    % For each matrix product, we tried $\lambda \in 2^i, i \in \{-5, -4, -3, -2, -1, 0, 1, 2, 3\}$ for each matrix product. To ensure that our results err on the side of optimism (and also speed up our experiments), we report the best
    % selected the lambda \textit{post-hoc}---i.e., we cherrypicked the best results on the test set rather than cross-validating,
    % \item FastJL \cite{fastjl}. Note that we don't time the fast hadamard transform step.
    \item HashJL \cite{hashjl}. $\V$ is zero except for a $\pm 1$ in each row, with both sign and position chosen uniformly at random.
    % \item OSNAP \cite{osnap}. The OSNAP sketch is essentially a collection of $s$ HashJL sketches, each of dimensionality $d / s$. Following \cite{iSVD}, we set $s = 4$. We also tried other values of $s$ and found that other values did not perform significantly better (except $s=1$, which reduces to HashJL).
    \item RandGauss \cite{lshOrig,E2LSH}. The entries of $\V$ are drawn i.i.d. from $\mathcal{N}(0, D^{-1})$.
    % \item OrthoGauss \cite{superbitLSH}. The entries of $\V$ initialized as in RandGauss, and then $\V$ is set to the $\mat{Q}$ matrix in a QR decomposition of a RandGauss matrix.
    % \item Rademacher \cite{rademacherJL}. The entries of $\V$ are i.i.d. Rademacher random variables scaled by $\frac{1}{\sqrt{D}}$. TODO is that the right scale?
    \item Bolt \cite{bolt}. As discussed in Section~\ref{sec:relatedWork} Bolt is the most similar to our own method. It is not optimized for small $M$, however, effectively performing a preliminary matrix product with $M=16$ as a preprocessing step.
    \item Exact Multiplication. We simply compute the matrix product $\A\B$ using a modern BLAS implementation. We also implemented our own matrix product function specialized for tall, skinny matrices. In all cases, we report the timings based on the better of these two functions for a given matrix product.
\end{itemize}

In addition to these baselines, we test two variations of our method:
\begin{itemize}
    \item \ours. This is the algorithm described in Section~\ref{sec:method}.
    \item \ours-PQ. This is a handicapped version of \oursp without the prototype optimization step. The disparity in performance between \oursp and \ours-PQ is the gain from optimizing the prototypes.
\end{itemize}

We also compared to many additional methods (see Appendix~\ref{sec:experimentDetails}), but omit their results since they were not competitive with the listed baselines.

% ------------------------------------------------
\subsection{How fast is \ours?}
% ------------------------------------------------

We begin by profiling the raw speed of our method and the most similar baselines. In Figure~\ref{fig:encodeSpeed}, we time the $g(\A)$ functions for $\A$ matrices with $2^{15}$ rows and varying number of columns $D$. Following \cite{bolt}, we also vary the size of the row encodings, profiling 8, 16, and 32 bytes per row. \oursp is not only faster than existing methods, but faster than the machine's memory bandwidth. This is possible because it only reads $O(C)$ columns, and $C$ can be lower than $D$.

\begin{figure}[h]
\begin{center}
\includegraphics[width=\linewidth]{amm/encode_speed}
\caption{\oursp encodes the matrix $\A$ orders of magnitude more quickly than existing vector quantization methods.}
\label{fig:encodeSpeed}
\end{center}
\end{figure}

We also profile the speed of our aggregation function $f(\cdot, \cdot)$ using the same baselines as \citet{bolt}. Though less dramatic a speedup than for the encoding function, we see that our average-based aggregation is significantly faster than the upcasting-based method of Bolt, its nearest rival.

\begin{figure}[h]
\begin{center}
\includegraphics[width=\linewidth]{amm/scan_speed}
\caption{Given the encoded matrices, \oursp computes the approximate output twice as fast as the fastest existing method.}
\label{fig:scanSpeed}
\end{center}
\end{figure}

% ------------------------------------------------
\subsection{Softmax Classifier}
% ------------------------------------------------

As described in section~\ref{sec:intro}, we approximated linear classifiers on the widely used CIFAR-10 and CIFAR-100 datasets \cite{cifarDsets}. The classifiers use as input features the 512-dimensional activations of an open-source, VGG-like neural networks trained on each dataset \cite{cifarVgg}. The matrices $\A$ are the $10000 \times 512$-dimensional floating point activations for the full test sets, and the matrices $\B$ are each original network's final dense layer. The $50000 \times 512$-dimensional activations from the training set served as the training matrices $\tilde{\A}$.

As shown in Figure~\ref{fig:cifar}, \oursp significantly outperforms all existing methods, achieving near-perfect accuracy more than an order of magnitude faster than brute force.

\begin{figure}[h]
\begin{center}
\includegraphics[width=\linewidth]{amm/cifar_Speedup_Accuracy}
\caption{\oursp achieves a far better speed-accuracy tradeoff than any existing method when approximating two softmax classifiers.}
\label{fig:cifar}
\end{center}
\end{figure}

% ------------------------------------------------
\subsection{Kernel-based classification}
% ------------------------------------------------

To assess the efficacy of our method on a larger and more diverse set of datasets than simply CIFAR-10 and CIFAR-100, we trained kernel classifiers on the datasets from the UCR Time Series Archive \cite{UCRArchive2018}. In addition to being a large, public corpus of over a hundred datasets from a huge variety of different domains, it also has the advantage that it can be used to produce matrix multiplication tasks of a fixed size. This is necessary for meaningful comparison of performance across datasets. We constructed training and test matrices $\tilde{\A}$ and $\A$ by resampling each time series in each dataset's train and test set to a length of $320$ (the closest multiple of 32 to the median length of 310). We obtained the matrix $\B$ for each dataset by running Stochastic Neighbor Compression \cite{snc} on the training set with an RBF kernel of bandwidth one. We set the number of returned neighbors to 128 (results with 64 and 256 were similar), yielding a $\B$ matrix of size $320 \times 128$. See Appendix~\ref{sec:experimentDetails} for additional details.

This is not the state-of-the-art means of classifying time series, but it does yield fixed-sized matrices and is representative of several modern techniques for constructing highly efficient classifiers \cite{snc,dsnc,bnc,protonn}. This setup also complements the CIFAR results by being an extremely difficult task, since Stochastic Neighbor Compression has already optimized the classifer to avoid redundancy. % since the distances between time series are often far smaller than their euclidean lengths; this property means that even small amounts of approximation error are sufficient to change predictions. It is also difficult because the $\B$ matrix has been optimized to avoid redundancy, so any acceleration

As shown in Figure~\ref{fig:ucr}, \oursp is significantly faster at a given level of accuracy. A counter-intuitive result, however is that optimization of the prototypes is occasionally worse than not optimizing them (see the red line dipping below the blue one in the lower subplot). Since the optimization strictly increases the expressive power, we believe that this is a product of overfitting and could be corrected were we to tune the ridge regression's parameter (instead of always using $\lambda = 1$). This subplot also reveals that the most stringent degredation requirements can sometimes only be satisfied by PCA at a low speedup, since this is the only curve close to $1$.
% , at high accuracy preservation thresholds,  % However as indicated by \oursp never approaching a fraction of 1 on the bottom subplot, \oursp does not consistently preserve the full accuracy. This suggests that, for difficult classification problems in which almost no accuracy can be sacrificed, \outsp is often not the best choice. The only methods that reliably preserve nearly the full original accuracy are PCA with a small speedup and, with certain parameter settings, Bolt.

\begin{figure}[h]
\begin{center}
\includegraphics[width=\linewidth]{amm/ucr2_Speedup_Relative Accuracy_rbf}
\caption{Fraction of UCR datasets for which each method preserves a given fraction of the original accuracy, versus the method's degree of speedup. \oursp enables much greater speedups for a given level of accuracy degredation}
\label{fig:ucr}
\end{center}
\end{figure}

% ------------------------------------------------
\subsection{Image Processing}
% ------------------------------------------------

To test the extreme limits of \ours, we benchmarked the various techniques' ability to apply small filters to images. As representative examples we chose $3 \times 3$ Sobel filters and $5 \times 5$ gaussian filters. The former are common high-pass filters and the latter are common low-pass filters. We took the first 10 images from the first 50 classes of the Caltech101 dataset as the training set, and the first 10 images from the remaining 51 classes as the test set. We constructed the $\A$ matrices by extracting each patch of each image in

To allow meaningful speed comparisons across images, we resized and center cropped each image to $224 \times 224$ as commonly done in image classification pipelines \cite{resNet,resnet2,densenet}. We then extracted sliding windows of the appropriate size and used each (flattened) window as one row of $\tilde{\A}$ or $\A$. We similarly flattened the filters, with each set of coefficients forming one column of $\B$. In both cases, $\B$ has two columns---this is because using a single filter would mean timing a matrix-vector product instead of a matrix-matrix product. Moreover, because Sobel filters are often used in horizontal and vertical pairings, and Gaussian filters are often used together to perform difference-of-Gaussians transforms. See Appendix~\ref{sec:experimentDetails} for additional details.

% \frac{1}{NM}
We report the normalized mean-squared error (NMSE), which is defined as $\norm{\mat{\hat{C}}_{i,j} - \A\B}^2_F / \norm{\A\B}^2_F$, where $\hat{\mat{C}}$ is the method's approximation of $\A\B$. An NMSE of 0 is perfect and an NMSE of 1 corresponds to always predicting 0.

In Figure~\ref{fig:caltech}, we see that it is only \oursp that offers any advantage over exact matrix products. This is likely due to the fact that two columns afford almost no time to amortize preprocessing costs for $\A$; indeed, rival vector quantization methods cannot logically do less worth than brute force in this setting, and dense linear methods can only save work by embedding rows of $\A$ in one-dimensional space.

\oursp performs much worse on the high-pass filters (top) than the low-pass filters (bottom). This is likely because the former produce outputs with variance that is orders of magnitude lower than that of the original image; this means that the NMSE denominator, $\norm{\A\B}^2_F$ is extremely small.% relative to $\norm{\A}_F$ and $\norm{\B}_F$.

% a method must preserve nearly all the information about the image to get $\norm{\mat{\hat{C}}_{i,j} - \A\B}^2_F$

% Notes
%   -CHW order
%   -resized and center cropped to 224x224 as is common in deep learning
%   -stride of (2, 2) on training set because

\begin{figure}[h]
\begin{center}
\includegraphics[width=\linewidth]{amm/caltech_Speedup_1 - NMSE}
\caption{Despite having only $M=2$ columns in the matrix $\B$, \oursp still achieves a significant speedup with reasonable accuracy. It is, however, much more effective for approximating a 5x5 low-pass filter (bottom) than a 3x3 high-pass filter (top). Methods that did not simultanesouly perform better than chance and provide any speedup over exact matrix multiplication are not shown.}
\label{fig:caltech}
\end{center}
\end{figure}

% % ------------------------------------------------
% \subsection{Summary}
% % ------------------------------------------------

% Our method without \textit{any} hyperparameter tuning outperforms its nearest rival with all its hyperameters tuned \textit{on the test set}.



%================================================================
\section{Conclusion}
%================================================================

We introduce an approximate matrix multiplication algorithm that achieves a significantly better speed-quality tradeoff than existing methods, as measured on hundreds of real-world matrices from various domains. Our method's performance stems from 1) its replacement of the bottleneck in existing methods with a learned locality-sensitive hash function, 2) its optimization of a quantity that other methods cannot optimize without a large slowdown, and 3) its use of an inexact estimator for computing sums. In future work, we plan to integrate our method into neural networks and specialize it for convolutions.

% ================================================================
% References
% ================================================================

% \IEEEtriggeratref{27} % trigger column break to make cols even
% \bibliographystyle{ACM-Reference-Format}
% \bibliographystyle{abbrev}
% \bibliographystyle{sysml2019}
\bibliographystyle{icml2020}
% \bibliography{prune,architectures,misc,understandDnn,classic,datasets,compress,science,metapapers}


% TODO uncomment

\bibliography{architectures,datasets,misc,sprintz,extract,bolt,backprop-alternatives,distillation,fast-dnn-runtime-stuff,fast-optimize,nets-theory,small-fast-arch,sparseAndPrune,scalarQuantize,vectorQuantize,combo,hashing,shrink+prune,binarize,ternary,automl,zero-shot,few-shot,amm,ammMore,lsh,adversarial}
\clearpage
\newpage  % so we can cut the pdf
\appendix

% ================================================================
\section{Quantizing Lookup Tables} \label{sec:lutQuantize}
% ===============================================================

% Existing work has shown that setting $K = 16$ and quantizing the lookup tables for a given column of $\B$ to 8 bits can offer enormous speedups compared to larger $K$ and/or floating-point tables \cite{bolt, quickAdc, quickerAdc}. This is because 16 1-byte entries can be stored in a SIMD register, allowing 16 or more table lookups to be performed in parallel in a single instruction.

Since the lookup table entries naturally occupy more than 8 bits even for 8-bit data (since products of 8-bit values require 16 bits), some means of quantizing these entries is necessary to enable vectorization. Unfortunately, existing quantization methods are not applicable to our problem setting. The scheme of \citet{bolt} requires knowledge of $\B$ at training time, while the scheme of \citet{quickAdc} and \citet{quickerAdc} is only applicable for nearest-neighbor search. We instead use the following approach, where $\mat{T} \in \R^{M \times C \times K}$ is the tensor of lookup tables for all $M$ columns of $\B$, $\mat{T}^q$ is the quantized version of $\mat{T}$, $\vec{\delta} \in \R^C$ is a vector of table-specific offsets, and $\alpha^{-1}$ is an overall scale factor:
\begin{align}
    \vec{\delta}_c &\triangleq \min_{m,k} \text{ } \mat{T}_{m,c,k} \\
    \alpha^{-1} \triangleq 2^l \cs l &= \max_c \floor{ \log_2 \left( \frac{255}{
            \max_{m,k} (\mat{T}_{m,c,k} - \delta_{c})
        } \right)}. \\
    \mat{T}^q_{m,c,k} &\triangleq \alpha^{-1}(\mat{T}_{m,c,k} - \delta_{c})
\end{align}
This is similar to equations~\ref{eq:offset} and \ref{eq:scale}, but with the scale factor pooled across all codebooks instead of unique to each input column. The $\alpha$ used here is the same as that in equation~\ref{eq:objective}, and the matrix $\beta$ in equation~\ref{eq:objective} has entries equal to $\sum_c \vec{\delta}_c$ (plus the debiasing constant from our averaging-based aggregation).

% ================================================================
\section{Quantization and \oursHash} \label{sec:hashQuantize}
% ================================================================

The use of at most 16 leaves is so that the resulting codes use 4 bits. This allows the use of these same shuffle instructions to accelerate the table lookups as in \citet{bolt}.

The only subtlety in vectorizing our hash function is that one must execute line 4 using shuffle instructions such as \texttt{vpshufb} on x86, \texttt{vtbl} on ARM, or \texttt{vtbl} on PowerPC. In order to do this, the split values and scalars $x_{j^t}$ must be 8-bit integers. We quantize them by learning for each split index $j$ a pair of scalars $(\gamma_j, \delta_j)$, where
\begin{align}
    \delta_j &\triangleq \min_i \v^j_i \label{eq:offset} \\
    % \gamma_j \triangleq 2^l, l = \left \lfloor \text{log2} \left( \frac{255}{\max_i \v^j_i - \delta_j} \right) \right \rfloor
    \gamma_j &\triangleq 2^l, l = \floor{ \log_2 \left( \frac{255}{\max_i \v^j_i - \delta_j} \right) \label{eq:scale} }
\end{align}
% and
% \begin{align}
% %     &\min_i \v^j_i - \delta_j = 0 \\
% %     &\max_i \gamma_j(\v^j_i - \delta_j) \le 255
% \end{align}
% and $\gamma_j$ is a power of 2.
This restriction of $\gamma_j$ to powers of two allows one to quantize $x_{j^t}$ values with only shifts instead of multiplies. The $\vec{v}$ values can be quantized at the end of the training phase, while the $x_{j^t}$ values must be quantized within Algorithm~\ref{algo:ourEnc} before line 5.

% ================================================================
\vfill\break  % columnbreak
\section{Subroutines for Training \oursHash} \label{sec:optimalSplitVal}
% ================================================================

The \texttt{optimal\_split\_threshold} algorithm (Algorithm~\ref{algo:optimalSplitVal}) finds the best threshold at which to split a bucket within a given dimension. To do this, it uses the \texttt{cumulative\_sse} function (Algorithm \ref{algo:cumSSE}) to help evaluate the loss associated with the resulting child buckets.

These algorithms exploit the fact that the sum of squared errors can be computed using only the sum of values and sum of squared values, both of which can be updated in $O(1)$ time when a vector is moved from one side of the split to the other.

% ------------------------------------------------ optimalSplitVal

\begin{algorithm}[h]
\caption{Optimal Split Threshold Within a Bucket} \label{algo:optimalSplitVal}
\begin{algorithmic}[1]
    \STATE {\bfseries Input:} bucket $\mathcal{B}$, index $j$
    \STATE {$\mat{X} \leftarrow \texttt{as\_2d\_array}(\mathcal{B})$ }
    \STATE {$\mat{X}^{sort} = \texttt{sort\_rows\_based\_on\_col}(\mat{X} \text{, } j)$}
    \STATE {$\texttt{sses\_head} \leftarrow \texttt{cumulative\_sse}(\mat{X}^{sort}, \texttt{false}) $}
    \STATE {$\texttt{sses\_tail} \leftarrow \texttt{cumulative\_sse}(\mat{X}^{sort}, \texttt{true}) $}
    \STATE {$\texttt{losses} \leftarrow \texttt{sses\_head} $}
    \STATE {$\texttt{losses}_{1:N-1} \leftarrow \texttt{losses}_{1:N-1} + \texttt{sses\_tail}_{2:N} $}
    % \STATE {$ n^\ast \leftarrow \min_n \sum_{d=1}^D $}
    % \STATE {$ n^\ast \leftarrow \argmin_n \texttt{sses\_head}_n + \texttt{sses\_tail}_{n+1} $}
    \STATE {$ n^\ast \leftarrow \argmin_n \texttt{losses}_n $}
    \STATE{$ \textbf{return } (\mat{X}^{sort}_{n^\ast \text{, } j} + \mat{X}^{sort}_{n^\ast + 1 \text{, } j}) / 2 \text{, } \texttt{losses}_{n^\ast} $}
    % \STATE {$\texttt{sses\_tail} \leftarrow \texttt{reverse\_row\_order}(\texttt{cumsse\_cols}(\texttt{reverse\_row\_order}(\mat{X}_{sort})_) $}
    % \STATE {$\texttt{sses\_tail} \leftarrow \texttt{reverse\_row\_order}(\texttt{cumsse\_cols}(\texttt{reverse\_row\_order}(\mat{X}_{sort})_) $}

\end{algorithmic}
\end{algorithm}

% ------------------------------------------------ cumSSE

\begin{algorithm}[h]
% \caption{Cumulative SSE Within Columns} \label{algo:cumSSE}
\caption{Cumulative SSE} \label{algo:cumSSE}
\begin{algorithmic}[1]
    \STATE {\bfseries Input:} 2D array $\mat{X}$, boolean \texttt{reverse}
    \STATE {$N, D \leftarrow \texttt{shape}(\mat{X})$}
    \IF {\texttt{reverse}}
        \STATE{$ \forall_i \texttt{ swap}(\mat{X}_{i,d}, \mat{X}_{N-i+1,d}) $}
    \ENDIF

    % \STATE {$\texttt{out} \leftarrow \texttt{empty}(N \text{, } D)$}
    \STATE {$\texttt{out} \leftarrow \texttt{empty}(N)$}
    \STATE {$\texttt{cumX} \leftarrow \texttt{empty}(D)$}
    \STATE {$\texttt{cumX2} \leftarrow \texttt{empty}(D)$}

    \LINECOMMENT{Initialize first row of output and cumulative values}
    \STATE{$\texttt{out}_{1} \leftarrow 0 $}
    \FOR{$d \leftarrow 1 \textbf{ to } D $}
        \STATE{$\texttt{cumX}_d \leftarrow X_{1, d} $}
        \STATE{$\texttt{cumX2}_d \leftarrow (X_{1, d})^2 $}
        % \STATE{$\texttt{out}_{1, d} \leftarrow 0 $}
    \ENDFOR

    \LINECOMMENT{Compute remaining output rows}
    \FOR{$n \leftarrow 2 \textbf{ to } N $}
        \STATE{$\texttt{out}_{n} \leftarrow 0 $}
        \FOR{$d \leftarrow 1 \textbf{ to } D $}
            \STATE{$\texttt{cumX}_d \leftarrow \texttt{cumX}_d + X_{1, d} $}
            \STATE{$\texttt{cumX2}_d \leftarrow \texttt{cumX2}_d + (X_{1, d})^2 $}
            % \STATE{$\texttt{meanX} \leftarrow \texttt{cumX}_d / n $}
            % \STATE{$\texttt{out}_{n, d} \leftarrow \texttt{cumX2}_d - (\texttt{cumX}_d \times \texttt{cumX}_d / n)$}
            \STATE{$\texttt{out}_{n} \leftarrow \texttt{out}_{n} + \texttt{cumX2}_d - (\texttt{cumX}_d \times \texttt{cumX}_d / n)$}
        \ENDFOR
    \ENDFOR
    \STATE{\textbf{return } \texttt{out}}
\end{algorithmic}
\end{algorithm}

% ================================================================
% \vfill\break  % columnbreak
\clearpage
\section{Analysis of Aggregation Using Pairwise Averages} \label{sec:aggregateAnalysis}
% ================================================================


\begin{definition}[Averaging Integer Sum Estimator]
Let $\x \in \{0, 1\}^C, C \text{ \% } U = 0, U = 2^p, p \ge 0$. The Averaging Integer Sum Estimator (AISE) $\hat{s}(\x)$ is defined as:
\begin{align}
    % s(\x) &\triangleq \sum_{k=1}^{C / U} s_U(\x_{((k-1)*U+1):((k-1)*U+1+U)}) \\
    \hat{s}(\x) &\triangleq \sum_{k=1}^{C / U} \hat{s}_U(\x_{i_k:j_k}) \\
    \hat{s}_U(\x) &\triangleq
        \begin{cases}
            x_1 & \x \in \R^1 \\
            % \lfloor \frac{1}{2}(x_1 + x_2 + 1) \rfloor & \x \in \R^2 \\
             % \lfloor \frac{1}{2}(s_U(\x_{:\text{len}(x)/2}) + s_U(\x_{\text{len}(x)/2:}) + 1) \rfloor & \text{otherwise}
             \lfloor \frac{1}{2}(\hat{s}_U(\x_{left}) + \hat{s}_U(\x_{right}) + 1) \rfloor & \text{otherwise}
       \end{cases}
\end{align}
where $i_k = (k-1) \cdot U+1, j_k = i_U + U$ and $\x_{left}$ and $\x_{right}$ denote vectors formed by taking the initial and final $D/2$ indices of a given $\x \in \R^D$.
\end{definition}

\begin{definition}[Pairwise Integer Sum and Sum Estimator]
For integers $a$ and $b$, define
\begin{align}
    s(a, b) &\triangleq a + b \\
    \hat{s}(a, b) &\triangleq 2 \mu(a, b)
\end{align}
where $\mu(a, b) \triangleq \lfloor \frac{1}{2}(a + b + 1) \rfloor$.
\end{definition}

\begin{lemma}[Bias when averaging one pair] \label{thm:avgBias}
Consider two scalars $a$ and $b$, with $a, b \iid$ Bernoulli(.5). Define $\eps(a, b) \triangleq \hat{s}(a, b) - s(a, b)$. Then
\[
    E[\eps(a, b)] = \frac{1}{2}
\]
\end{lemma}
\begin{proof} The proof follows immediately from considering the four equiprobable realizations of the pair $a, b$. In the cases $(0, 0)$ and $(1, 1)$, $2\mu(a, b) = s(a, b)$. In the cases $(0, 1)$ and $(1, 0)$, $2 \mu(a, b) = 2$, while $s(a, b) = 1$.
% : $(0, 0)$, $(0, 1)$, $(1, 0)$, $(1, 1)$
\end{proof}




% \begin{lemma}[Bias when averaging one pair] \label{thm:avgBias}
% Consider two scalars $a$ and $b$, $a, b \iid Bernoulli(.5)$. Then
% \[
%     E[\mu(x, y)] = \frac{1}{2}
% \]
% \end{lemma}
% \begin{proof} The proof follows immediately from considering the four equiprobable realizations of the pair $a, b$. In the cases $(0, 0)$ and $(1, 1)$, $\mu(a, b) = 0$. In the cases $(0, 1)$ and $(1, 0)$, $\mu(a, b) = 1$.
% % : $(0, 0)$, $(0, 1)$, $(1, 0)$, $(1, 1)$
% \end{proof}

\begin{lemma}[Variance of error when averaging one pair]
Consider two scalars $a$ and $b$, $a, b \iid$ Bernoulli(.5). Then
\[
    % E[\mu(a, b)^2] - E[\mu(x, y)]^2 = \frac{1}{4}
    E[\eps(a, b)^2] - E[\eps(x, y)]^2 = \frac{1}{4}
\]
\end{lemma}
\begin{proof} Using Lemma~\ref{thm:avgBias}, the above can be rewritten as:
\[
    % E[\mu(a, b)^2] = \frac{1}{2}
    E[\eps(a, b)^2] = \frac{1}{2}
\]
The proof then follows by again considering the four equiprobable cases as in Lemma~\ref{thm:avgBias}. In the cases $(0, 0)$ and $(1, 1)$, $\eps(a, b)^2 = 0$. In the cases $(0, 1)$ and $(1, 0)$, $(2 \hat{s}(a, b) - s(a, b))^2 = (2 - 1)^2 = 1$.
\end{proof}

% \begin{lemma}[Bias Recursive Step]
% % When applying the function $\mu$ recursively,
% \[
%     % E[\mu(\mu(a, b), \mu(c, d))] = 1
%     % E[\mu(\mu(a, b), \mu(c, d))] = 2 (E[\mu(a, b)] + E[\mu(c, d)])
%     % E[\eps(\hat{s}(a, b), \hat{s}(c, d))] = .5 + E[\eps(a, b)] + E[\eps(c, d)]
%     E[\eps(\mu(a, b), \mu{s}(c, d))] = .5 + E[\eps(a, b)] + E[\eps(c, d)]
% \]
% \end{lemma}
% \begin{proof}This can equivalently be formulated as
% \begin{align}
%     % E[\mu(\mu(a, b), \mu(c, d))] = 2 E[\mu(\mu(a, b), \mu(c, d))]
%     E[\mu(\mu(a, b), \mu(c, d))] = 2 E[\mu(a, b)] + E[\mu(c, d)]
% \end{align}
% \end{proof}



% \begin{lemma}[Bias and error variance of AISE within a subspace] \label{thm:avgSubspace}
\begin{lemma}[Bias of AISE within a subspace] \label{thm:avgSubspace}
% Let $\hat{s}(\x), \x \in \R^C, C \text{ \% } U = 0$ be the sum estimator described in section~\ref{sec:aggregate} and let $s(\x) \triangleq \sum_c x_c$.
% Let $s(\x)$ be defined as
Suppose that the scalar elements $x_i$ of $\vec{x}$ are drawn from independent Bernoulli(.5) distributions. Then
\begin{align}
    E[s_U(\x) - \hat{s}_U(\x)] &= U \log_2(U) / 4 \\
    % Var[s_U(\x) - \hat{s}_U(\x)] &= U \log_2(U) / 8.
\end{align}
\end{lemma}
\begin{proof}
Observe that the computation graph can be cast as a balanced binary tree with $U$ leaves and each parent equal to the integer average of its children. Consider the bias introduced at each level $t$ of the tree, where $t=0$ corresponds to the leaves and $t = \log_2(U)$ corresponds to the root. The expected error $E[\xi(t, n)]$ introduced at a node $n$ in level $t > 0$ is given by:
\begin{align}
    E[\xi(t, n)] = \frac{1}{2} \cdot 2^{t - 1}
\end{align}
where the $\frac{1}{2}$ follows from Lemma~\ref{thm:avgBias} and the scale $2^{t - 1}$ is the number of leaf nodes to which the bias is effectively applied. E.g., adding 1 to the estimated average of four leaf nodes would increase the estimated sum by four. Since there are $U \cdot 2^{-t}$ nodes per level, this means that the total expected error introduced at level $t$ is $\frac{1}{2} \cdot 2^{t - 1} \cdot 2^{-t} = \frac{1}{4}$. Summing from $t = 1$ to $t = \log_2(U)$ completes the proof of the expected error. Note that $t=0$ is omitted since the leaf nodes are not the result of averaging operations and so introduce no error.

% Identical reasoning can be used to derive the variance of the error, since the variances from different nodes add.
\end{proof}


% \begin{theorem}[Bias and error variance of AISE]
\begin{theorem}[Bias of AISE] \label{thm:overallBias}
% Let $\hat{s}(\x), \x \in \R^C, C \text{ \% } U = 0$ be the sum estimator described in section~\ref{sec:aggregate} and let $s(\x) \triangleq \sum_c x_c$.
% Let $s(\x)$ be defined as
Suppose that the scalar elements $x_i$ of $\vec{x}$ are drawn from independent Bernoulli(.5) distributions. Then
\begin{align}
    E[s(\x) - \hat{s}(\x)] &= C \log_2(U) / 4 \\
    % Var[s(\x) - \hat{s}(\x)] &= C \log_2(U) / 8.
\end{align}
\end{theorem}
\begin{proof}
This follows immediately from Lemma~\ref{thm:avgSubspace}, the fact that the overall sum is estimated within each of $C / U$ subspaces of size $U$, and assumption that the errors in each subspace are independent.
\end{proof}

We also verified Theorem~\ref{thm:overallBias} numerically by summing large numbers of integers drawn uniformly from the interval $0,\ldots,255$.

Note that the assumption the the elements are independent is not especially strong in reality. This is because this section focuses on the effects on the least significant bits (which are the ones affected by each averaging operation), and the least significant bit does tend to be nearly uniformly random in a great deal of real-world data.

% ================================================================
\vfill\break  % columnbreak
\section{Additional experimental details} \label{sec:experimentDetails}
% ================================================================

% ------------------------------------------------
\subsection{Choice of Matrix Multiplication Tasks}
% ------------------------------------------------

Because nearly all existing work on approximate matrix multiplication either focuses on special cases that do not satisfy our problem definition \cite{quickerAdc, pq, opq} or synthetic matrices, there is not a clear set of benchmark matrix multiply tasks to use. We therefore propose a collection of tasks that we believe are both reproducible and representative of many real-world matrices. To the best of our knowledge, our experiments use over an order of magnitude more matrices than any previous study.

% ------------------------------------------------
\subsection{Choice of Single Threaded Benchmarks}
% ------------------------------------------------

Given the ubiquity of GPUs and multicore CPUs, it may not be obvious why single-threaded experiments are desirable. There are a number of reasons we restrict our focus to CPUs and the single threaded case:
% We only implemented our algorithm on the CPU for simplicity and
\begin{itemize}
    \item To enable fair comparisons to existing work, particularly the nearest rival, Bolt \cite{bolt}.
    % \item Existing work, particularly our method's nearest rival \cite{bolt}, only uses a single thread.
    \item To facilitate fair comparisons to our work by future authors---single-threaded experiments are much easier to reproduce and extend than multithreaded ones.
    \item Matrix multiplication is embarrassingly parallel with respect to rows of the input and columns of the output. There is therefore nothing ``interesting'' about how our method parallelizes relative to any other; all methods reduce to a single-threaded kernel that can easily be applied to disjoint submatrices. While we certainly could spend the considerable time required to construct and debug multicore benchmarks, this would be unlikely to yield any useful insights.
    \item Parallelization in modern numerical libraries is often managed at a higher level than the lowest-level subroutines. For example, the authors of FBGEMM \cite{fbgemm} state: \textit{``Internally, FBGEMM is intentionally designed not to create any threads. Usually, such a library is intended to be used as a backend by deep learning frameworks, such as PyTorch and Caffe2, that create and manage their own threads.''}\footnote{https://engineering.fb.com/ml-applications/fbgemm/} I.e., a multithreaded library calls into single-threaded subroutines (such as a matrix multiplication function); it is this single-threaded subroutine where we make contributions, and therefore where we focus our experimental efforts. Independent of common practices in modern libraries, this pattern is also the only sensible strategy for small matrices, like many of those we consider---the overhead of launching and joining threads is extremely unlikely to be worth it for sufficiently small matrices. We could perhaps characterize where this breakpoint is, but this is a hardware-specific result that has little to do with our contributions.
    % \item There are many subtle issues around where data ``originates'' that
    \item While training of deep neural networks is typically done on GPUs or other accelerators, trained models (including, but not limited to, neural networks) are commonly deployed on smartphones with just CPUs and/or graphics acceleration that is no better than the CPU \cite{fbAtEdge}. And since most of the billions of smartphones in the world tend to be low end or old, the need to deploy models on CPUs (including those with few cores) is unlikely to change for many years.
    \item Creating, benchmarking, and analyzing a performant implementation of our method for GPUs would require a great deal of engineering work. We plan to create such an implementation in the future, but believe that the many empirical and theoretical results we currently have are more than adequate proof
    of concept and already worth sharing with the community. % Moreover, given both the rapid pace of change in accelerator hardware and diversity of existing hardware, even results on GPUs would quickly
    % \item While it would require a great deal of engineering work to produce a performant GPU version of our method, experiments on CPUs are sufficient proof of concept to suggest
    % \item Our basic ideas can be applied to GPUs or other devices. The only CPU-specific constraint informing our algorithm is the need to have splits at the same level of a tree all use the same index. We of course cannot know precisely how well our ideas work on any given piece of hardware absent direct testing of a high-performance implementation. We therefore plan to create a GPU implementation of our method in future work.
\end{itemize}

% ,

% ------------------------------------------------
\subsection{SparsePCA details}
% ------------------------------------------------

We took steps to ensure that SparsePCA's results were not hampered by insufficient hyperparameter tuning. First, for each matrix product, we tried a range of lambda values which we found to encompass the full gamut of nearly 0\% to nearly 100\% sparsity: $\lambda \in 2^i, i \in \{-5, -4, -3, -2, -1, 0, 1, 2, 3\}$. Second, because different sparsity patterns may yield different exeuction times, we report not times from the single matrix SparsePCA produces for a given ($d, \lambda$) pair, but the best times from any of ten random matrices of the same size and at most the same sparsity. Finally and most importantly, we plot only the Pareto frontier of (speed, quality) pairs produced for a given matrix multiply. I.e., we let SparsePCA cherrypick its best results on each individual matrix multiply.

% ------------------------------------------------
\subsection{Exact Matrix Multiplication}
% ------------------------------------------------

We also implemented our own matrix product function specialized for tall, skinny matrices. In all cases, we report the timings based on the faster of this function and Eigen's \cite{eigen} matrix multiply function for a given matrix product.

% ------------------------------------------------
\subsection{Additional Baselines}
% ------------------------------------------------

We also tested Frequent Directions / Fast frequent directions \cite{liberty_simple_2012, ghashami_frequent_2016, isvd}, many variations of the sampling method of \cite{drineas_fast_2006}, projection using orthogonalized gaussian random matrices \cite{superbitLSH}, projection using matrices of scaled i.i.d. rademacher random variables \cite{rademacherJL}, projection using orthonormalized matrices of rademacher random variables, the co-occuring directions sketch \cite{mroueh_co-occuring_2016}, OSNAP \cite{osnap}, Product Quantization \cite{pq}, and Optimized Product Quanization \cite{opq}.

The poor performance of many of these methods is unsurprising in our setting. Given that we have access to a training set on which to learn the true principal components, the Eckart-Young-Mirsky theorem \cite{eckartYoungMirskyThm} indicates that PCA should outperform any other individual matrix sketching method employing dense projection matrices, at least in the limit of infinite training data. Also, since PQ and OPQ use 256 dense centroids (except in the Bolt / QuickerADC variations), it is also impossible for them to perform well when $\min(D, M)$ is not significantly larger than 256.

% ------------------------------------------------
\subsection{UCR Time Series Archive}
% ------------------------------------------------

We omitted datasets with fewer than 128 training examples, since it is not possible for Stochastic Neighbor Compression to draw 128 samples without replacement in this case.

In addition to being a large, public corpus of over a hundred datasets from a huge variety of different domains, the UCR Time Series Archive also has the advantage that it can be used to produce matrix multiplication tasks of a fixed size. This is necessary for meaningful comparison of speed vs accuracy tradeoffs across datasets. We constructed training and test matrices $\tilde{\A}$ and $\A$ by resampling each time series in each dataset's train and test set to a length of $320$ (the closest multiple of 32 to the median length of 310). We obtained the matrix $\B$ for each dataset by running Stochastic Neighbor Compression \cite{snc} on the training set with an RBF kernel of bandwidth one. We set the number of returned neighbors to 128 (results with 64 and 256 were similar), yielding a $\B$ matrix of size $320 \times 128$. %See Appendix~\ref{sec:experimentDetails} for additional details.
Since different datasets have different test set sizes, all results are for a standardized test set size of 1000 rows. We wanted the length to be a multiple of 32 since existing methods operate best with sizes that are either powers of two or, failing that, multiples of large powers of two.

We approximate Euclidean distances using the identity $\norm{\vec{x} - \vec{y}}_2^2 = \norm{\vec{x}}_2^2 - 2\vec{x}^\top \vec{y} + \norm{\vec{y}}_2^2$. We approximate only the inner products $\vec{x}^\top \vec{y}$, since $\norm{\vec{y}}_2^2$ can be precomputed for fixed exemplars $\vec{y}$ and $\norm{\vec{x}}_2^2$ doesn't affect the class prediction since it is constant across all exemplars for a given input $\vec{x}$.

% since having a length that is a multiple of at least 8 is the best-case scenario for most existing methods, and 320 is a ``rounder'' number than 312.

% ------------------------------------------------
\subsection{Caltech101}
% ------------------------------------------------

We only extracted valid windows---i.e., never past the edge of an image. We extracted the windows in CHW order, meaning that scalars from the same color channel were placed at contiguous indices. The ``first'' images are based on filename in lexicographic order.

We used pairs of filters because using a single filter would mean timing a matrix-vector product instead of a matrix-matrix product.

To allow meaningful speed comparisons across images, we resized and center cropped each image to $224 \times 224$ as commonly done in image classification pipelines \cite{resNet,resnet2,densenet}. We then extracted sliding windows of the appropriate size and used each (flattened) window as one row of $\tilde{\A}$ or $\A$. We similarly flattened the filters, with each set of coefficients forming one column of $\B$. In both cases, $\B$ has two columns---this is because using a single filter would mean timing a matrix-vector product instead of a matrix-matrix product. Two columns also made sense since Sobel filters are often used in horizontal and vertical pairings, and Gaussian filters are often used together to perform difference-of-Gaussians transforms.

Even though the RGB values at each position are naturally unsigned 8-bit integers, we allowed all rival methods to operate on them as 32-bit floating point, without including the conversion when timing them. Because it only requires checking whether values are above a threshold, \oursp can operate on 8-bit data directly.

% ------------------------------------------------
\subsection{Additional Results}
% ------------------------------------------------

In Section~\ref{sec:results}, we showed the classification accuracy as a function of wall time for the CIFAR-10 and CIFAR-100 softmax classifiers, as well as on the UCR datasets. In Figure~\ref{fig:cifarNMSE} and Figure~\ref{fig:ucrNMSE}, we instead show normalized mean squared error versus time. In Figure~\ref{fig:cifarOps} and Figure~\ref{fig:caltechOps}, we show accuracy or NMSE vs number of operations performed, where one operation is either one multiply-add or one table lookup, depending on the method. The first two figures illustrate that NMSE is closely related to classification accuracy, but with imperfect NMSE still yielding excellent accuracy in many cases. The second two figures show that our method's superior results are not merely caused by the use of faster CPU instructions, but also by the use of fewer basic operations at the algorithm level.

\begin{figure}[h]
\begin{center}
\includegraphics[width=\linewidth]{amm/cifar_Speedup_1 - NMSE}
\caption{\oursp achieves a far better speed versus squared error tradeoff than any existing method when approximating two softmax classifiers. These results parallel the speed vs classification accuracy results, except that the addition of our ridge regression is much more beneficial on CIFAR-100.}
\label{fig:cifarNMSE}
\end{center}
\end{figure}

\begin{figure}[h]
\begin{center}
\includegraphics[width=\linewidth]{amm/ucr2_Speedup_1 - NMSE_rbf}
\caption{\oursp achieves the lowest squared error at high speedups on the UCR datasets. These results parallel the speed vs classification accuracy results.}
\label{fig:ucrNMSE}
\end{center}
\end{figure}

\begin{figure}[h]
\begin{center}
\includegraphics[width=\linewidth]{amm/cifar_ops_Accuracy}
\caption{\oursp achieves the best speed versus accuracy tradeoff on the CIFAR datasets of any method even when speed is measured as number of operations instead of wall time. Note that fewer operations with a high accuracy (up and to the left) is better.}
\label{fig:cifarOps}
\end{center}
\end{figure}

\begin{figure}[h]
\begin{center}
\includegraphics[width=\linewidth]{amm/caltech_ops_1 - NMSE.pdf}
\caption{\oursp still achieves the best speed versus squared error tradeoff on the image processing tasks when speed is measured as number of operations instead of wall time.}
\label{fig:caltechOps}
\end{center}
\end{figure}


% ================================================================
% \vfill\break  % columnbreak
% \vfill
\clearpage
\section{Proof of Generalization Guarantee} \label{sec:maddnessMath}
% ================================================================

In this section, we prove Theorem~\ref{thm:maddness}, restated below for convenience.

\begin{theorem*}[Generalization Error of \ours] \label{thm:maddness}
Let $\Dcal$ be a probability distribution over $\R^D$ and suppose that \oursp is trained on a matrix $\tilde{\A} \in R^{N \times D}$ whose rows are drawn independently from $\Dcal$ and with maximum singular value bounded by $\sigma_A$. Let $C$ be the number of codebooks used by \oursp and $\lambda > 0$ the regularization parameter used in the ridge regression step. %Then for any vector $\b$, any vector $\a \sim \Dcal$, and any $0 < \delta < 1$, we have with probability at least $1 - \delta$ that
Then for any $\b \in \R^D$, any $\a \sim \Dcal$, and any $0 < \delta < 1$, we have with probability at least $1 - \delta$ that
 % Then for all vectors $\b$ and any vector $\a \sim \Dcal$, we have with probability at least $1 - \delta$, $0 < \delta < 1$,
\begin{align*}
    \begin{split}
    \E_{\Dcal}[&\Lcal(\a, \b)] \le \E_{\tilde{\A}}[\Lcal(\a, \b)] + \\
    &\frac{C \sigma_A \norm{\b}_2}{2 \sqrt{\lambda}} \left(
        \frac{1}{256} +
        \frac{
            8 +
            \sqrt{
                % C (4\ceil{\log_2(D)} + 256) \log{2} -\log{\delta}
                \nu(C, D, \delta)
            }
        }{\sqrt{2n}}
    \right)
    \end{split}
% \end{equation}
\end{align*}
% where
% where $\Lcal(\a, \b) \triangleq |\a^\top \b - \alpha f(g(\a), h(\b)) - \mat{\beta}|$ (c.f., Equation~\ref{eq:objective}).
where $\Lcal(\a, \b) \triangleq |\a^\top \b - \alpha f(g(\a), h(\b)) - \mat{\beta}|$, $\alpha$ is the scale used for quantizing the lookup tables, $\mat{\beta}$ is the constants used in quantizing the lookup tables plus the debiasing constant of Section~\ref{sec:aggregate}, and
\begin{align*}
    \nu(C, D, \delta) \triangleq C (4\ceil{\log_2(D)} + 256) \log{2} -\log{\delta}.
\end{align*}
% \begin{align}
%     \begin{split}
%     \E_{\Dcal}[ \Lcal(\a, \b)] \le \E_{\tilde{\A}}[\Lcal(\a, \b)] +
%     \frac{C \sigma_A \norm{\b}_2}{2 \sqrt{\lambda}} \bigg(
%         \frac{1}{256} + \\
%         \frac{
%             8 +
%             \sqrt{
%                 % C ((4 + \frac{2}{C})\ceil{\log_2(D)} + 284) \log{2} -\log{\delta}
%                 C (4\ceil{\log_2(D)} + 256) \log{2} -\log{\delta}
%             }
%         }{\sqrt{2n}}
%     % \right)
%     \bigg)
%     \end{split}
% \end{align}
% where $\Lcal(\a, \b) \triangleq |\a^\top \b - \alpha f(g(\a), h(\b)) - \mat{\beta}|$, $\alpha$ is the scale used for quantizing the lookup tables, and $\mat{\beta}$ is the constants used in quantizing the lookup tables plus the debiasing constant of Section~\ref{sec:aggregate}.

\end{theorem*}

The proof relies on the observation that \ours's training procedure can be decomposed into two sequential subroutines: \texttt{Maddness-Build-Tree}, which learns the function $g(\a)$ by constructing a binary decision tree, and \texttt{Maddness-Regress}, which learns the function $h(\b)$ by optimizing a prototype matrix $\P$ such that $g(\tilde{\A}) \P \approx \tilde{\A}$.
% regressing the output $\mat{G} \triangleq g(\tilde{\A})$ onto the.
%prototype matrix $\P$ used by using ridge regression conditioned on the output of \texttt{Maddness-Build-Tree} on the training matrix $\tilde{\A}$.
% We will prove an overall generalization guarantee for \oursp by first providing a guarantee for \texttt{Maddness-Regress} for a fixed \texttt{Maddness-Build-Tree} hypothesis, and then union bounding over the hypothesis space for \texttt{Maddness-Build-Tree}.
This observation allows us to prove~\ref{thm:maddness} by first providing a guarantee for \\ \texttt{Maddness-Regress} for a fixed \texttt{Maddness-Build-Tree} hypothesis, and then union bounding over the hypothesis space for \texttt{Maddness-Build-Tree}. Bounding the size of the hypothesis space is straightforward (Lemma~\ref{lemma:ntrees}), so the bulk of this section focuses on providing a guarantee for \texttt{Maddness-Regress}. We must also prove a bound on the loss contributed by quantizing the lookup tables array $\P^\top \b$. % There is also a need to bound the errors introduced by quantizing the lookup tables $\P^\top \b$, but this error is small and.

\begin{lemma}[Number of Hypotheses for \texttt{Maddness-Build-Tree}] \label{lemma:ntrees}
Let $C$ be the number of codebooks used by \oursp and let $D$ be the number of columns in the matrix $\tilde{\A}$ on which \oursp is trained. Then there are at most $2^{C (4\ceil{\log_2(D)} + 256)}$ unique trees that \texttt{Maddness-Build-Tree} can generate.
\end{lemma}
\begin{proof}
\texttt{Maddness-Build-Tree} learns four sets of parameters for each of the $C$ trees it produces: split indices, split offsets, split scales, and split values.

There are four split indices per tree because there is one for each of the tree's four levels. Each index requires $\ceil{\log_2(D)}$ bits to store, so the split indices require a total of $4 \ceil{\log_2(D)}$ bits per tree. For each split index, there is one split offset and scale, used to map floating point data in an arbitrary range to the interval $[0, 255]$ to match up with the 8-bit split values.

The offsets require at most 25 bits each for 32-bit floating point data, since the low 7 bits can be dropped without affecting the post-scaling quantized output. The scales are constrained to be powers of two, and so require at most 9 bits for non-subnormal 32-bit floating point inputs (which have one sign bit and eight exponent bits). The offsets and scales together therefore contribute $4 (25 + 9) = 136$ bits per tree.

There are $15$ split values because there is one for the root of each tree, then two for the second level, four for the third, and eight for the fourth. Each split value is stored using eight bits, so each tree requires $15 \cdot 8 = 120$ bits for split values. The total number of bits used for all trees is therefore $C(4 \ceil{\log_2(D)} + 256)$. Note that the constant 256 being a power of two is just an accident of floating point formats. The claimed hypothesis count follows from the number of expressible hypotheses being at most $2$ to the power of the largest number of bits used to store any hypothesis.
% Finally, it is necessary to encode the number of bits used to store split indices, $\ceil{\log_2(D)}$; assuming $D$ can be stored in 32 bits, this is at most 5 bits.
\end{proof}

% \begin{theorem}[[Bartlett and Mendelson 2002]] let $\Fcal$ be a class of functions, let $\Scal$ be a set of $n$ samples drawn i.i.d. from some distribution $\Dcal$, and let $\Lcal(f), f \in \Fcal$ be a loss function with Lipschitz constant $l$. Then for any $\delta, 0 < \delta < 1$, it holds for all $f \in \Fcal$ with probability at least $1 - \delta$ that
% \begin{align}
%     \E_\Dcal[\Lcal{f}] \le \E_S[\Lcal(f)] + 2 l \Rcal_n(\Fcal) + l \sqrt{\frac{\log(1 / \delta)}{2n}}
% \end{align}
% \end{theorem}

We now turn our attention to bounding the errors of the regression component of training. Our strategy for doing so is to bound the largest singular value of the learned matrix of prototypes $\P$. Given such a bound, the norms of both $g(\a)^\top \P$ and $\P^\top \b$ can be bounded.

% and $\Y \in \R^{D \times M}
\begin{lemma}[Regularized Pseudoinverse Operator Norm Bound] \label{lemma:pinvBound}
Let $\X \in \R^{N \times D}$
% $\lambda$
% $\mat{I}$
% $\W$
be an arbitrary matrix with finite elements. Then every singular value $\sigma_i$ of the matrix $\Z \triangleq (\X^\top \X + \lambda \mat{I})^{-1} \X^\top$, $\lambda > 0$ is at most $\frac{1}{2\sqrt{\lambda}}$.
\end{lemma}
\begin{proof} Let $\U \Sigm \Vt$ be the singular value decomposition of $\X$. Then we have
\begin{align}
    \Z &= (\X^\top \X + \lambda \I)^{-1} \X^\top   \\
    &= (\V \Sigm \Ut \U \Sigm \Vt + \lambda \I)^{-1} \V \Sigm \Ut \\
    &= (\V \Sigm^2 \Vt + \lambda \I)^{-1} \V \Sigm \Ut \\
    &= (\V \Sigm^2 \Vt + \V \lambda \I \Vt)^{-1} \V \Sigm \Ut \label{step:wrapI} \\
    &= (\V \Sigm_\lambda \Vt)^{-1} \V \Sigm \Ut \\
    &= \V \Sigm_\lambda^{-1} \Vt \V \Sigm \Ut \\
    &= \V \Sigm_\lambda^{-1} \Sigm \Ut \\
    &= \V \Sigm^\prime \Ut
\end{align}
where $\Sigm_\lambda \triangleq \Sigm^2 + \lambda \I$ and $\Sigm^\prime \triangleq (\Sigm^2 + \lambda \I)^{-1} \Sig$. Step \ref{step:wrapI} follows from the equality $\V \lambda \I \Vt = \lambda \V \Vt = \lambda \I$. Because the matrices $\V$ and $\Ut$ are orthonormal and $\Sigm^\prime$ is diagonal, the singular values of $\Z$ are equal to the diagonal entries of $\Sigm^\prime$. Each entry $\sigma_i^\prime$ is equal to
\begin{align}
    \sigma_i^\prime = \frac{\sigma_i}{\sigma_i^2 + \lambda}.
\end{align}
This expression attains its maximal value of $\frac{1}{2\sqrt{\lambda}}$ when $\sigma_i^2 = \lambda$.
\end{proof}

\begin{lemma}[Ridge Regression Singular Value Bound] \label{lemma:ridgeBound}
Let $\X \in \R^{N \times D}$ and $\Y \in \R^{D \times M}$ be arbitrary matrices and let $\W \triangleq (\X^\top \X + \lambda \mat{I})^{-1} \X^\top \Y$, $\lambda > 0$ be the ridge regression weight matrix. Then $\norm{\W}_\inf \le \frac{\norm{\Y}_\inf}{2 \sqrt{\lambda}}$, where $\norm{\cdot}_\inf$ denotes the largest singular value.
\end{lemma}
\begin{proof}
Observe that $\W = \Z \Y$, where $\Z \triangleq (\X^\top \X + \lambda \mat{I})^{-1} \X^\top$. Then by applying Lemma~\ref{lemma:pinvBound} and recalling that Schatten norms are submultiplicative, we have
\begin{align}
    \norm{\W}_\inf \le \norm{\Z}_\inf \norm{\Y}_\inf \le \frac{\norm{\Y}_\inf}{2\sqrt{\lambda}}.
\end{align}
\end{proof}

\begin{lemma}[Bound on \oursp Embedding Norm] \label{lemma:embedNorm}
Let $\g = g(\a)$ be the encoding of an arbitrary vector $\a$ using $C$ codebooks and let $\P$ be the prototype matrix learned by \oursp using training matrix $\tilde{\A}$ with ridge regression parameter $\lambda > 0$. Then
\begin{align}
    \norm{\g^\top \P}_2 \le \frac{C}{2 \sqrt{\lambda}} \norm{\tilde{\A}}_{\inf}
\end{align}
where $\norm{\tilde{\A}}_{\inf}$ denotes the largest singular value of $\tilde{\A}$.
\end{lemma}
\begin{proof} We have
\begin{align}
    \norm{\g^\top \P}_2 &\le \norm{\g}_2 \norm{\P}_\inf \\
    &= C \norm{\P}_\inf \\
    &\le \frac{C}{2 \sqrt{\lambda}} \norm{\tilde{\A}}_{\inf}.
\end{align}
The first step follows from Cauchy-Schwarz. The second follows from $\g$ being zero except for exactly $C$ ones. The last is an application of Lemma~\ref{lemma:ridgeBound}.
\end{proof}

\begin{lemma}[Maximum Table Quantization Loss] \label{lemma:lutQuantBound}
Let $\ahat = g(\a)^\top\P$, where $g(\cdot)$ and $P$ are trained using $C$ codebooks and ridge regression penalty $\lambda > 0$ on a matrix $\tilde{\A}$ with maximum singular value at most $\sigma_A$, and $\a \in \R^D$ is an arbitrary vector. Then for any vector $\b \in \R^D$, $|\ahat^\top \b - \yhat| < \frac{C \sigma_A \norm{\b}_2}{512 \sqrt{\lambda}} $, where %$\yhat \triangleq g(\a)^\top (\P^\top \b)$.
$\yhat \triangleq \alpha g(\a)^\top g(\b) + \mat{\beta}$ is \ours's approximation to $\a^\top \b$, with $\alpha$ and $\mat{\beta}$ the scale and offsets used to quantize the lookup tables.
%defined in Equation~\ref{eq:objective} except without the averaging debias constant added to $\mat{\beta}$.
\end{lemma}
\begin{proof} If \oursp had infinite-precision lookup tables, $\yhat$ would exactly equal $\ahat^\top \b$. We therefore need only bound the error introduced by the quantization. By Lemma~\ref{lemma:embedNorm}, $\norm{\ahat}_2 \le \frac{C \sigma_A}{2 \sqrt{\lambda}} $. This implies that
\begin{align}
    \norm{\ahat^\top \b} \le \frac{C \sigma_A \norm{\b}_2}{2 \sqrt{\lambda}}
\end{align}
and therefore
\begin{align}
    \frac{-C \sigma_A \norm{\b}_2}{2 \sqrt{\lambda}} \le \ahat^\top \b \le \frac{C \sigma_A \norm{\b}_2}{2 \sqrt{\lambda}}.
\end{align}
For each of the $C$ codebooks, this means that the value to be quantized lies in the interval $[\frac{-\sigma_A \norm{\b}_2}{2 \sqrt{\lambda}}, \frac{\sigma_A \norm{\b}_2}{2 \sqrt{\lambda}}]$ of width $\frac{\sigma_A \norm{\b}_2}{\sqrt{\lambda}}$.
Because \oursp quantizes the lookup tables such that largest and smallest entries for any row of $\P$ are linearly mapped to $255.5$ and $-0.5$,\footnote{We use $255.5$ and $-0.5$ rather than $255$ and $0$ because the latter only guarantees that a point is within $1 / 510$ of the interval width, not $1 / 512$. This is not an important choice and either option would be fine.} respectively, the worst-case quantization error is when the quantized value lies exactly between two quantization levels.% (as opposed to lying outside the linearly rescaled interval).
We therefore need to compute the largest possible gap between a value and its quantization. Using 256 quantization levels, the largest possible gap is $1 / (256 / .5) = 1/512$ of the interval width. Multiplying by the above interval width yields a maximum quantization error for a given codebook of $\frac{\sigma_A \norm{\b}_2}{512 \sqrt{\lambda}}$.
 % Using the worst-case interval above and observing that 256 quantization levels imply a maximum quantization error of $1 / (256 / .5) = 1/512$ of the interval width gives us a maximum error for each codebook of $\frac{\sigma_A \norm{\b}_2}{512 \sqrt{\lambda}}$.
Because the errors in each subspace may not agree in sign, their sum is an upper bound on the overall quantization error.
\end{proof}

At this point, we have all of the pieces necessary to prove a generalization guarantee for \texttt{Maddness-Regress} save one: a theorem linking the norms of the various vectors and matrices involved to a probabilistic guarantee. Kakade et al. \cite{kakadeLinear} provide such a gaurantee, based on Rademacher complexity \cite{rademacherOrig}.

\begin{theorem}[\cite{kakadeLinear}, Corollary 5] \label{thm:linearGeneralize}
Let $\Fcal = \{\w^\top \x : \norm{\w}_2 \le W \}$ be the class of linear functions with bounded $L_2$ norms, let $\Scal$ be a set of $n$ samples drawn i.i.d. from some distribution $\Dcal$ over the $L_2$ ball of radius $X$,
% such that $\x \sim \Dcal \implies \norm{\x}_2 \le X$ almost surely for some $X$
and let $\Lcal(f), f \in \Fcal$ be a loss function with Lipschitz constant $L$. Then for any $0 < \delta < 1$, it holds with probability at least $1 - \delta$  over the sample $\Scal$ that
\begin{align}
    \E_\Dcal[\Lcal(f)] \le \E_S[\Lcal(f)] + \frac{L X W}{\sqrt{2n}} \left( 8 + \sqrt{-log(\delta)} \right) .
    % l \Rcal_n(\Fcal) + l \sqrt{\frac{\log(1 / \delta)}{2n}}
\end{align}
\end{theorem}

We can now obtain our desired guarantee for the regression step.

% \begin{lemma}[Rademacher Complexity and Maximum Loss of \texttt{Maddness-Regress}].
\begin{lemma}[Generalization Error of \texttt{Maddness-Regress}] \label{lemma:regGeneralize}
% $\Dcal: \R^{N \times D} \rightarrow \R_+$
% \tilde{\A} \sim \Dcal,

% Let $\tilde{\A} \in \R^{N \times D}$ be the matrix on which \oursp is trained. Let $\G = g(\tilde{\A}) \in \R^{N \times 16C}$ be the encoding of $\tilde{\A}$ using a fixed (data-independent) set of $C$ trees output by \\
% \texttt{Maddness-Build-Tree}; let $\sigma_A$ be an upper bound on the largest singular value of any matrix $\A$ drawn from $\Dcal$ with $N$ rows; let $\lambda > 0$ be the regularization parameter of the ridge regression used by \texttt{Maddness-Regress}; and let $\Lcal(\a, \b) \triangle |\a^\top \b - f(g(\a), h(\b))|$. Then for any row $\a$ of any matrix $\A \sim \Dcal$, and any vector $\b, \norm{\b}_2 \le L_{\b}$, it holds with probability at least $1 - \delta$, $0 < \delta < 1$ over the training matrix $\tilde{\A}$ that:
Let $\Dcal$ be a probability distribution over $\R^D$ and suppose that \oursp is trained on a matrix $\tilde{\A} \in R^{N \times D}$ whose rows are drawn independently from $\Dcal$ and with maximum singular value bounded by $\sigma_A$. Let $C$ be the number of codebooks used by \oursp and $\lambda > 0$ the regularization parameter used in the ridge regression step. Further let $g(\a)$ be a fixed (data-independent) function and $\Lcal(\a, \b) \triangleq |\a^\top \b - f(g(\a), h(\b))|$. Then for all vectors $\b$, any vector $\a \sim \Dcal$, and any $0 < \delta < 1$, we have with probability at least $1 - \delta$ that
\begin{align} \label{eq:regressGeneralize}
    \begin{split}
    \E_{\Dcal}[ & \Lcal(\a, \b)] \le \E_{\tilde{\A}}[\Lcal(\a, \b)] + \frac{C \sigma_A \norm{\b}_2}{512 \sqrt{\lambda}} + \\
    &\frac{C \sigma_A \norm{\b}_2}{2\sqrt{2n \lambda}} \left( 8 + \sqrt{-log(\delta)} \right).
    \end{split}
\end{align}
\end{lemma}

\begin{proof}
The output of \texttt{Maddness-Regress} can be decomposed into
\begin{align}
    \yhat \triangleq f(g(\a), h(\b)) = \g^\top \P \b + \eps + \zeta
\end{align}
where $\g = g(\a)$, $\P$ is the matrix of prototypes, $\eps$ is data-independent noise from the averaging process\footnote{We continue to make the assumption that the least significant bits of the lookup table entries are independent Bernoulli(0.5) random variables, which is nearly true in practice. Even if this assumption does not hold, this noise does not contribute to the generalization gap unless it differs between train and test sets.}, and $\zeta$ is noise from quantizing the lookup table entries. By Lemma~\ref{lemma:lutQuantBound}, $\zeta \le \frac{C \sigma_A \norm{\b}_2}{512 \sqrt{\lambda}}$ (accounting for the second term in Equation~\ref{eq:regressGeneralize}). We therefore need only obtain a guarantee for $|\g^\top \P \b - \a^\top \b|$. Defining $\w \triangleq \P \b$, we see that \texttt{Maddness-Regress} is a linear model, and therefore subject to Theorem~\ref{thm:linearGeneralize}. Given an upper bound on the Lipschitz constant of the loss, a bound on the $L_2$ norm of $\g$, and a bound on the $L_2$ norm of $\w$, we can apply this theorem. The Lipschitz constant for the absolute loss is 1. The $L_2$ norm of $\g$ is exactly $C$. The $L_2$ norm of $\w$ can be bounded as
\begin{align}
    \norm{\w}_2 &= \norm{\P \b}_2 \le \norm{\P}_\inf \norm{\b}_2
    \le \frac{\sigma_A \norm{\b}_2 }{2\sqrt{\lambda}}
\end{align}
using Lemma~\ref{lemma:ridgeBound}.
\end{proof}

Using this lemma, the proof of Theorem~\ref{thm:maddness} is immediate; we begin with Lemma~\ref{lemma:regGeneralize} and simply union bound over all $2^{C (4\ceil{\log_2(D)} + 120)}$ hypotheses from Lemma~\ref{lemma:ntrees}.




\end{document}
