
Recall that our task is to construct functions $g(\cdot)$, $h(\cdot)$, and $f(\cdot)$ such that
\begin{align}
    \norm{f(g(\A), h(\B)) - \A\B}_F < \eps(\tau) \norm{\A\B}_F
\end{align}
$\eps(\tau)$ is minimized given a time constraint $\tau$.

% ------------------------------------------------
\subsection{Quantization Function - $g(\A)$}
% ------------------------------------------------

We begin by discussing

Basically just intuition for why it just splits on a bunch of dims. Then define a split formally. Then pseudocode (or probably just formula for applying them)

% ------------------------------------------------
\subsection{Learning Splits}
% ------------------------------------------------

Introduction to fact that we have two variations, one that requires training data and one that doesn't. Former is greedy and mention that we prove it's close to optimal.

Pseudocode for probably both, but maybe just the greedy one.

% ------------------------------------------------
\subsection{Table Construction - $h(\B)$}
% ------------------------------------------------

% ------------------------------------------------
\subsection{Lookup Scan - $f(\cdot,\cdot)$}
% ------------------------------------------------

If we didn't have pseudocode in background section, put it here. Break it down by encoding, LUT creation, and distance computation. Seems like we should probably already have explained these, so probably just text saying we kind of tie it all together and replace whatever lines in earlier pseudocode with hash-based encoding.


